\chapter{What is a Single Board Computer}
	<TODO Chapter What is a Single Board Computer : NOT DONE>

\section{Parts of a Single Board Computer (SBC)}
	<TODO Section Parts of a Single Board Computer (SBC) : NOT DONE>

\subsection{Microprocessor(s)}
	<TODO Subsection Microprocessor(s) : NOT DONE>

\subsection{Memory}
	<TODO Subsection Memory : NOT DONE>

\subsection{Input and Output}
	<TODO Subsection Input and Output : NOT DONE>

\section{Applications for SBCs}
	<TODO Section Applications for SBCs : NOT DONE>
	
The number of applications for \ac{SBC}s is enormous. The rise of the raspberry pi and many similar boards have popularized home makers tinkering with boards. New applications are being developed every day.  Below we list seven popular applications. The first being retro gaming platform. You can play Atari games or most platforms on a \ac{SBC} with an emulator installed on you \ac{SBC}. You can host a small web-server. Have you ever wanted to make a website but didn't want to pay for web hosting. An \ac{SBC} can host a site on you \ac{LAN} for free. You can also make an Amazon echo, a Christmas light show, a time clock based on facial recognition, and a stock ticker. The last application we discuss is how we can use a \ac{SBC} as a test vector generator for an \ac{FPGA}. 

\subsection{Retro Gaming}
	<TODO Subsection Retro Gaming : NOT DONE>
	
It's no surprise that processors are faster now than where they were in the 1980's. Original gaming consoles were had eight bit processors and ran in the single digit megahertz range. For a lot of people that are interested in embedded processing on a \ac{SBC} and \ac{FPGA} they grew up with these types of systems. The exciting part of growing up with these systems is getting to play them again. For people who want to understand how they work they platform is there to support it.

Clearly our \ac{SBC} platform must be of higher performance than the retro game console that we are going to emulate. On the \ac{SBC} we can limit ourselves to use less \ac{RAM}. We can burn clock cycles to emulate a lower clock speed. We are starting with an \ac{SBC} that has a \ac{CPU} that operates at hundreds of \ac{MHz} and has \ac{GBs} of \ac{RAM}. To emulate the Atari 2600 which has a \ac{CPU} that runs at 1.19 \ac{MHz} and has 128 Bytes of \ac{RAM}.

\subsection{Small Web Server}
	<TODO Subsection Small Web Server : NOT DONE>

\subsection{Amazon Echo}
	<TODO Subsection Amazon Echo : NOT DONE>

\subsection{Christmas Light Show}
	<TODO Subsection Christmas Light Show : NOT DONE>

\subsection{Facial Recognition Time Clock}
	<TODO Subsection Facial Recognition Time Clock : NOT DONE>

\subsection{Stock Ticker}
	<TODO Subsection Stock Ticker : NOT DONE>

\subsection{FPGA Test Vector Generator}
	<TODO Subsection FPGA Test Vector Generator : NOT DONE>

