\documentclass[11pt]{book}

\usepackage[margin=1in,paperwidth=6in,paperheight=9in]{geometry}

\addtolength{\oddsidemargin}{-.25in}
\addtolength{\evensidemargin}{-.5in}
\addtolength{\textwidth}{.75in}

\addtolength{\topmargin}{-.5in}
\addtolength{\textheight}{.5in}

% Import User Packages 
%\usepackage[nolist]{acronym}
\usepackage{acronym}
\usepackage{fancyvrb}
\usepackage{listings,color}
\usepackage{amsmath}
\usepackage{amsfonts}
\usepackage{makeidx}
\usepackage[scaled]{helvet}
%\usepackage{pstricks}
\usepackage{graphicx}
\usepackage{color}
% for colored solution
\lstnewenvironment{VHDLlisting}[1][]
  {\lstset{
    language=VHDL,
    basicstyle=\footnotesize\ttfamily,
    columns=flexible,
    keepspaces=true,
    keywordstyle=\color{blue},
    identifierstyle=\color{black},
    commentstyle=\color{green}
  }}{}

\makeindex
\newenvironment{dedication}
{\clearpage           % we want a new page
	\thispagestyle{empty}% no header and footer
	\vspace*{\stretch{1}}% some space at the top 
	\itshape             % the text is in italics
	\raggedleft          % flush to the right margin
}
{\par % end the paragraph
	\vspace{\stretch{3}} % space at bottom is three times that at the top
	\clearpage           % finish off the page
}

\newcommand{\expec}[1]{{\mathcal{E}}\left\{#1\right\}}
\newcommand{\expecsub}[2]{{\mathcal{E}_{#2}}\left\{#1\right\}}
\newcommand{\diag}[1]{\mbox{diag}\left(#1\right)}
\newcommand{\ord}[1]{{{\mathcal{O}}\left( #1 \right)}}
\newcommand{\set}[1]{\left\{#1\right\}}
\newcommand{\norm}[1]{\| #1 \|}
\newcommand{\abs}[1]{\left| #1 \right|}
\newcommand{\prn}[1]{\left( #1 \right)}
\newcommand{\brk}[1]{\left[ #1 \right]}
\newcommand{\tavg}[1]{\left< #1 \right>}
\newcommand{\floor}[1]{\lfloor #1 \rfloor}
\newcommand{\ceil}[1]{\lceil #1 \rceil}
\newcommand{\gausspdf}[2]{{\mathcal{N}}\left(#1,#2\right)}
\newcommand{\eq}[1]{Equation~\eqref{eq:#1}}
\newcommand{\fig}[1]{Fig.~\ref{fig:#1}}
\newcommand{\figtwo}[2]{Figs.~\ref{fig:#1} and \ref{fig:#2}}
\newcommand{\figrange}[2]{Figs.~\ref{fig:#1}-\ref{fig:#2}}
\newcommand{\tab}[1]{Table~\ref{tab:#1}}
\newcommand{\sect}[1]{Section~\ref{sec:#1}}
\newcommand{\secttwo}[2]{Sections~\ref{sec:#1} and \ref{sec:#2}}
\newcommand{\eqtwo}[2]{Equations~\eqref{eq:#1} and \eqref{eq:#2}}

%opening

\title{Field Programmable Gate Arrays and Single Board Computers: What Are They and Why to Use Them}
\author{Dr. Jason Pennington}
\date{}

\begin{document}
\frontmatter
%\maketitle

%\psset{unit=1in}
%\begin{pspicture}(9in,6in)
%\psframe[fillstyle=solid,fillcolor=Blue](0,0)(9,6)
%\end{pspicture}


\begin{titlepage}
		\centering
		\vspace{4\baselineskip}
		{\Huge 
			Field Programmable Gate Arrays and\\ Single Board Computers\par}
		\vspace{1\baselineskip}
		\par
		{\Large\textsc{What Are They and Why to Use Them}\par}
		\vspace{4\baselineskip}
		by\par
		{\large\textsc{Jason Pennington, Ph.D.}\par}
		\vfill
		FBS Publishing\par
		{\em Fueled By Spite}
\end{titlepage}


	
\begin{flushleft}
		
		\textit{Field Programmable Gate Arrays and Single Board Computers :\\What Are They and Why to Use Them}
		
		\vfill	
		\textsf{\copyright} FBS Publishing, Inc. 
		
		\vspace{1\baselineskip}		
		
		ISBN-1234567891234
		
		\vspace{1\baselineskip}
		
		\noindent All rights reserved. No part of this publication may be produced or transmitted in any form or by any means, electronic or mechanical, including photocopying recording or any information storage and retrieval system, without the prior written permission of the publisher. For permissions contact copyright@fbspublishing.com
\end{flushleft}

\begin{dedication}
	To My Family  
\end{dedication}


\tableofcontents


\mainmatter
\chapter{Overview of Computing Platforms}
\section{What can an FPGA Do}
\section{What can SBCs Do}
\section{FPGA vs Processor}
\chapter{What is a Field Programmable Gate Array}
A \ac{FPGA} is a chip that is first \emph{field programmable} which means that the chip is reconfigurable after it leaves the foundry or the manufacturer of the chip. For example the \ac{FPGA} could perform the operations of a transceiver then minutes later, after reconfiguration, the \ac{FPGA} can perform image processing algorithms then could act as a web server. Because of this versatility the \ac{FPGA} has become a very cost effective solution as opposed to \ac{ASIC}.

The second aspect of an \ac{FPGA} is that the resources that are available are in a \emph{gate array}. The architecture of the gate array changes with newer versions of \ac{FPGA}. New versions of \ac{FPGA} attempt to optimize the resources for any application the \ac{FPGA} is used. The versatility of the \ac{FPGA} makes this optimization very difficult and because of this the efficiency of the design truly lies on the developer.

The responsibility of code efficiency can be a foreign to traditional software developers especially when the goal is to get code working; efficiency is an afterthought. The idea of \emph{efficiency later} gets \ac{HDL} programmers into trouble. For example if the programmer is not aware that the \ac{RAM} available on the \ac{FPGA} is dual-port and the programmer accesses more than two memory addresses in one clock cycle the synthesizer will interpret the \ac{RAM} as distributed \ac{RAM} instead of Block-\ac{RAM}. Depending on the size of the \ac{RAM} this is a costly mistake because distributed \ac{RAM} is implemented in \ac{LUT} which could be used for other logic. Many examples such as these are found in \ac{HDL} code, and this book aims to list out good practices to avoid such pitfalls.

This Chapter covers a broad topic of what an \ac{FPGA} is and to understand this we will discuss alternatives to the \ac{FPGA}; namely, the \ac{uC} and the \ac{ASIC}. Next, a couple of examples of applications of \ac{FPGA} are discussed for the final section of this chapter in which the resources available on the \ac{FPGA} are described and some common errors seen when using the resources with particular examples from the previous section. 


\section{FPGA Alternatives and Data Rate Capabilities}
There are many platforms for embedded processing. A Raspberry Pi is a very popular platform for the electronics hobbyist and the ARM processor that is available on the Raspberry Pi is very powerful for an embedded processor. However, there are applications, such as image processing, that a little more computing power is needed. This section discusses determining the requirements for the application and determining if a particular platform can handle the data rate required by the application. 

\subsection{Introduction to Data Rate}
A data rate in general is defined as the number of bits per unit time. For a system the data rate is how much data can the system handle per unit time. The overall data rate of the system is the minimum data rate for the individual components. In designing a system with a minimum data rate requirement it is important to be mindful of the data rates of the components.

To make the concept of determining data rate a little more concrete lets look at a simple architecture of a wireless receiver. To determine the data rate of a system the knowledge of the application and more importantly the architecture of the system needs to be known. Even for a simple receiver there are multiple ways to architect the design to enable higher or lower data rates. 


\subsection{Micro-Controllers (uC)}
\ac{uC}'s are embedded processors for use in \emph{small} systems. What is meant by a small system is relative, usually embedded systems are programmed for a specific task. For example live-streaming a webcam to the \ac{LAN}. \ac{uC}'s come in varying sizes; depending on the task to be performed. You don't need the cutting-edge i7 from intel to stream video, but an 8-bit \ac{PIC} \ac{uC} is too small for the task. However, an 8-bit \ac{uC} will turn on a water pump for 10 minutes a day to water a house hold plant. Just as different vehicles suit owner's needs in different ways so do \ac{uC}'s.

\subsection{Application Specific Integrated Circuits}
\ac{ASIC} chips are designed, of course, for a specific application. The knowledge of the algorithms to be calculated and the ability in an \ac{ASIC} to make any gate on the blank silicon make the \ac{ASIC} the most power processing platform. 
 

\section{Applications for FPGAs}
%In general applications for fpga a lend themselves applications for which data rates define hired generate in this section teradata rates in fpj Linda self NFPA 58 implantation lends itself over a microcontroller nfpga can handle names of multiple many streams of parallel incoming or outgoing information microcontroller a decision between an fpga 96 is volume how many units will be sold used by company cost havana section is the application of a transceiver sing and photography all of which have higher data rate requirements

\subsection{Transceivers}
%The transceivers the best examples are Wi-Fi hotspots and cell phone contact Tiffany these applications the volume of units sold such as cell phones or Wi-Fi routers the volume have to wear an e-cig would make a cost-effective solution the academic community research into cognitive radios 5th generation Wireless technology determining whether there is a lot of interference in a particular band switching to a different band and communicating their news this is because the higher data rates however go to and we are implementing prototypes on fpgas learning disability of atoms a simple example of a cognitive radio or a simple example of sensing the environment and then transmitting on that all that band is telephone where the playStation 4 hook up to phone for the transceiver and when you first did that the phone and the base station negotiate between two channels noizy which one has less interference it's a simple example of 22 channels that trans smith's information on one channel and then on the second channel the receiver says which one was no easier which one did I get what's the transmitter or the phone no simple algorithm can do that however that out of them is overhead in the Wireless Communications since voice is a state that is relatively low gatorade place that has a relatively low generate can get away with doing this at the beginning make it seamless transition to the phone call in cell phone technology I will get more complicated it won't just be it to him a little bit exciting between for the different users of the cell phones so maybe personne person and person B have two different channels that they are experiencing because they're in two different locations cellular network they both can't use the same frequency because they're still in the range of the cell phone tower has a better frequency ranges that be making sure each user gets frequency band they are clear to transmit denne on top of that if person a is using frequency range strange but there is a wall in there traffic person be good sense white spaces of the band and use that white space to transmit their own date that is the ultimate goal for a cognitive radio however metrics need to be ensured that they don't violate FCC regulations and hinder the communication of transmitter it is the transmitter. Would not have any knowledge or need to have any knowledge of the signal transmitter app just characterizing the regular traffic of transmitter and using the white space in between Transmissions possible ace Transmissions are not regular or they're not predictable than transmitter B would take it pre-action and not transmit on the band idea of how this algorithm of work sampling the environment environment and getting a spectrum map the environment this means nC Rangers time I need to be a balance between sensing and does in general can't do both at the same time in the section with motivated the use of fpga for next-generation transceivers in particular cognitive radio where a continuous or transmitting is needed to operate in these bands

\subsection{Image Processing}
%Image processing image processing is a broad research in this section image image processing does not mean transmitting video over a transceiver if we had high-definition video and we are broadcasting that over a wireless network that would just be a high generate transceiver to be able to broadcast that much data the fall under the previous transceiver application for fpga in this section image processing talk about not only having a a camera taking data so an application of where say road and you want to put a red circle around every car that is driving down the road you can track hcar maybe you want to identify each car maybe you want to try the car and find a license plate of each car these are the applications of image processing that long so it's very well to fpga there are two out rhythms two basic items that come to mind for image processing first of which is correlation for a two dimensional correlation or we look for a known object may be filled waldo anywhere the Balto book you would know what all that looks like you would need to correlate main difference size of his face biggest faces in the book to the page of the a smaller example of that application smaller example would be you have a 128 by 128 image image when you wanted the terminal where these four pixels are in that 128 by 128 fashion To Figure you can move lacrosse tween the four pixels 4 pixels that is currently overlay Don you determine the Enterprise with the maximum inner product is where those images of where your estimate 3x tomatoes for pizza sauce the reason why this application lends itself to his pj's so well is that as you were sampling the high resolution azure sampling the high-resolution images from the camera the inner product the inner product of the four pixels application image for is independent of the inner product of the four pixels when's the state is available wait until the rest of the image is in the LPGA you can be operating on this while you're out ceiling frame from the camera a lot of times in high speed generate or how long to digital converters multiple lanes are provided at the same clock cycle so maybe instead of depending on the speed of the camera can you get 42 pixels rising edge of the clock maybe that represents one half of 1/5 of a column in the image so in this case and that example 16 calculations can start 16 calculations of calculations for 16 independent inter products can start the other application are the other algorithm for image processing has the deal with eigenvalue or some sort of naked Matrix decomposition I'd refer you to reference

\subsection{Cryptography}
%Sakura Transmissions also a great application for fpgas is it because the extreme that you would normally send in the clear available to you on the fpga it's what you're interested in any way the images that you wanted the transmitter or the text that you wanted to oyster as that voice data or images are being processed you can provide this to and intensity on the fpga to encrypt that information and you don't need to wait for all of the information to be available to you cryptographic are there such as elliptic curve cryptography or RSA pS 128 or 256 the show albums random very well to an fpga Plantation for more information on this topic please visit Wikipedia

\section{Architecture of an FPGA}
%Architecture of an fpga provides many resource blocks to the user for processing is resource blocks include Black Ram restoring information the user as the user is programming dedicated multipliers because multiplication and division in Hardware is resource-intensive if using look up tables there are dedicated multipliers available on certain models of fpga from the different vendors is also the switching matrices which are for which data can be routed throughout the year those look up tables clock trees and global inputs and outputs from the SPCA to get your data in today at PGA and off of you quickly this section will discuss all of the resources available to the user on the mpg

\subsection{Block RAM}
%Black Ram is a dedicated memory to the user must be used carefully is my block right and need to be used carefully because most of the time block Ram has either a single or dual port single or dual program because of this only one or two respectively memory blocks Indiana candy addressed in one clock cycle if your Hardware or if your vehicle is accessing multiple addresses the same clock from the same block Ram the synthesizer will interpret this as distributed instantiate that on the fpga fabric and not use block Ram it's a very inefficient use of distributor of your fpga resources you should be redesigned to pool from the Block Ram one or two depending on the architecture of the block grant you using on your fpga under 1 Clarkson


\subsection{Multipliers}
%Multiplier
%Multipliers are dedicated Hardware that can be used on any claw catch multiply two values they can also be used for division for applications such as image processing or wireless transceiver filters for manipulation alone Souls very well that using the dedicated multiplier all the major vendors for fpga provide more dedicated multipliers

\subsection{Switching Matrices}
%switching matrices are used to Route information from resource block searches look up tables and other resources switching matrices can you can think of as interceptions in a city block where the switching matrices are configured at startup to wrap hair in a certain direction writing in a certain direction switching matrices can be configured once at startup and after that they are I can figure out certain way what is possible that you can figure matrix does cut off other resources from the user the place and Route algorithm in nevados Arizona suppose Idaho tool and Altera quartus tool try to optimize this wrapped root price of Maserati my the routing optimize the oh my Godrouting due to the complexity of the lpga's resources complexity of the albums that are being developed for fpga say please and write algorithms are very it is not uncommon for design Institute hours place around


\subsection{Look Up Tables}
%Look up tables who used to do the on the fpga they can be configured in 4many different ways 4 amp five important 16 plus we're fixin to go in any set of outputs can be driven from the inputs defenders of the lpga's have different texas for the Lots what are packaged into aces or carbs slices or C RVs the slices RC Arby's table have a multiplexer and have a register inside them these registers there can be used or not use depending on the algorithm and the multiplexer is configured to Output the results from the left or help with the register having an understanding of the architecture of the fpj you're targeting whether it's 1206 input for input food fit on the internet the synthesizer and the place and route job easier understand the architecture of the board they are targeting


\subsection{Clock Trees}
%Clock trees are resources that are used in every desire every synchronous design clock trees are a dedicated source for providing the clocks to all of your entities and all of your logic set of the logic that is fpga is because it offset between areas of the fpga there could be timing violations violations occurred when the isn't that a steady-state rock Rises or Falls and that Value Inn correct value is registered if you have clock skew or clock differences between pGA you maybe bring a string values from another at the fpga an inaccurate time dedicated dedicated 4 o'clock that minimize in vhdl if you say it's rising it and then whatever the verizon Edge is that value will be allocated a clock in the clock tree it is important to not get your clock it is important to not put signals you don't want to be considered a clock price match if rising Edge

\subsection{Inputs and Outputs}
%The most designed pleasing to the outside world or referrals on the board is going to be necessary whether it's there a microphone the ethernet the push button LED any of these are inputs and outputs to the there are different size packages and different size models of fpga is providing different amounts of CO2 the SPCA low voltage differential signaling most of the LPGA high bandwidth my clock frequency data rate are hiding rate for the SPCA in determining which fpga is right for you and how many inputs and outputs you would need to your fpga is necessary
%Programmable Logic






\chapter{What is a Single Board Computer}
\chapter{Design Flow Methodology}
	<TODO Chapter Design Flow Methodology : PROOF READ>
This chapter outlines the design flow process. This process takes in as an input an idea to be implemented in an embedded system. The idea can be a large and complicated system; the design flow process will help break down the complexity into manageable pieces. These smaller more manageable pieces can then be tested then optimized for the embedded platform. 

The output of the design flow process is code that tests not only the smaller more manageable pieces but also the larger design. If the design is too large and complex to test in it's entirety a few pieces can be put together to ensure those operate properly together. 

\section{Complex Designs Need Simple Steps}
	<TODO Section Complex Designs Need Simple Steps : PROOF READ>

The term complex is relative. To a beginning student learning VHDL a sequential adder statement can be complex. The reason it is complex is because the student does not yet know the syntax for performing such a calculation. As the student learns the projects get harder, maybe that addition operation is combined with a multiplication operation and repeated many times to get a Finite Impulse Response \ac{FIR} Filter.  The FIR filter is more complex than the addition but the FIR filter can be one of the first of many blocks in a communications design. 

In any design the system needs to be broken down into subcomponents. When students first start breaking down problems they can get into trouble. The way in which they decide to break the problem down leads them into design issues when they stitch the blocks back together. In this sense a top down approach is needed to ensure compatibility in the blocks. Once this common interface is decided upon then each subcomponent is built up.
	
\section{Functional Verification}
	<TODO Section Functional Verification : PROOF READ>

Functional verification of a design assumes gates have zero propagation delay, which is of course not true on an FPGA. Some programmers have a bad practice of not functionally verifying the design before attempting to program an FPGA. They feel that functional verification is a waste of time, since even if it does work in simulation it still may not work on an FPGA. But if a design does not work in simulation it can not work on hardware and RTL is easier to debug in simulation.

To functionally verify a system, where a system is anything from a simple adder to an 802.11 compliant wireless communication receiver to an Artificially Intelligent robot we first need a model of the behavior in a high level language. It is best to debug the higher level language for performance metrics of the system. Then we take the high level language model and make some changes that reflect FPGA limitations for example floating point precision in a \ac{CPU} to fixed point reduced precision on an FPGA. Once this model is working we can start writing \ac{RTL} to do the same calculations. We can use testbenches in \ac{HDL} to stimulate the \ac{HDL} in the same way the hardware model works so that we can directly compare results to the high level language results. Once the model and the \ac{HDL} match perfectly the design is said to be functionally verified.

Here I'd like to add another step though. This involves compiling the design in a vendor's tools. This ensures that the \ac{RTL} that functionally works can be realized in hardware. It is possible that the FPGA does not have the resources needed to perform the calculations. It is also possible that the data rates for the algorithm are too high and some redesigning is needed. We can also check to ensure the design meetings timing.

		

\subsection{Overview of Design Flow}
	<TODO Subsection Overview of Design Flow : PROOF READ>

This design flow is used with one goal in mind; minimize debugging of \ac{RTL}. Of course with writing any code you want to minimize bugs in the code. But debugging \ac{RTL} consists of delaying signals by a clock cycle to make sure the data lines up with the data valid flag. The debugging is much more lower level, its actually "bare-metal" programming. When we are so far down in the details of delaying a signal by nanoseconds we should not be concerned with the theoretical or practical performance on the algorithms we are implementing. Those questions are answered by the high level model.  
	
\subsection{System Level Model}
	<TODO Subsection System Level Model : PROOF READ>

The first step in writing \ac{HDL} is not to write \ac{HDL}. First we write a higher level model of what we would like to write in \ac{RTL}. MATLAB is common but also Python or Julia is used to write a working program that generates the desired results. This practice may seem trivial when we are going to make an addition block work but this first step of modeling the system in a high level language is needed when we are designing more complex designs. It is good practice to start following this design flow so that more complex designs are handled easily.

As an example we are going to design a \ac{FIR} filter. We assume we have the filter coefficients and we are going to write our high level model in MATLAB. We can see in the code snippet that the majority of the calculation is done in one line of MATLAB code with the rest of the code setting up the parameters of the filter. We will now convert this model to something that we can write \ac{RTL} to do.

	<TODO Subsection System Level Model : INSERT CODE fir filter sys model>
	
\subsection{System Level Model with Hardware}
	<TODO Subsection System Level Model with Hardware : PROOF READ>

There are three aspects of the above system model that we will need to address when we write \ac{RTL} for this filter. First, in MATLAB the coefficients and signal to be filtered are floating point 64-bit numbers. In \ac{RTL} we do not want or need this precision. We do not want this precision because we will use up significant \ac{FPGA} resources and we do not need the precision because we can sacrifice some performance of our filter and still get the desired result. The balance of signal quality and precision can be determined in this system level hardware model to ensure when we start writing \ac{RTL} we get the desired performance.

The second difference to consider is that on an \ac{FPGA} the data is not likely to be generated and stored in a vector. Although it is possible to store the signal on the \ac{FPGA} in this fashion we would not be using the \ac{FPGA} in its highest performance mode which is when we are streaming data through the system. We always want to have data moving. So for this example we will assume that the coefficients for the \ac{FIR} filter are constant but the signal that we are going to filter are streaming in a sample at a time.

Third, we are writing the \ac{FIR} filter ourselves. So having a MATLAB function that handles the calculation for us does not help us understand how the \ac{FIR} filter works. So we need to do the calculations our selves in MATLAB so that we can compare our intermediate steps in MATLAB to our intermediate steps in \ac{RTL}. Since we know MATLAB's function works we can compare our calculation result to the MATLAB function's result to make sure our math is correct.

One last change seen in this new MATLAB script is to notice that some key signals are written to text files. The purpose of writing the signal to be filtered and the filter result to a file is so that we can test the \ac{RTL} with the exact same data. If we test the \ac{RTL} \ac{FIR} filter with the same data we can make debugging a little easier because we know what the results should be and if there is a difference we can find it immediately. We write the result to a text file because we can have the testbench check the result for us using assert statements. Automating the checking allows use to run a lot of data through the filter so make sure we do not have data overflow. Data overflow or wrapping occurs with multiplication or addition were the result can not be represented by the number of bits used. If there is a bug in our code that does not protect against overflow the only way we will find it is by launching a lot of data.  With the hardware system model in MATLAB complete we can start writing the testbench and \ac{RTL} for the \ac{FIR} filter.

	<TODO Subsection System Level Model with Hardware : INSERT CODE fir filter hw model>
	
\subsection{HDL with testbenches}
	<TODO Subsection HDL with testbenches : PROOF READ>

To write the testbench for the \ac{FIR} filter we need three \ac{RTL} files. First we need the testbench. Think of this testbench as a workbench with a breadboard, power supply, and signal generators. This is where we are going to test the \ac{FIR} filter. The second \ac{RTL} file we are going to need is a block that reads the text files that we generated in MATLAB. Since you will be testbenching every block in your design I recommend making a generic test file read block that you can drop in whenever you need it. It may come in handy to have a block that writes a file as well but we won't be using that here. The final file we need is the \ac{FIR} filter itself. Up until now I've used the generic \ac{RTL} term for \ac{FPGA} code since this can be done in Verilog or \ac{VHDL}. Now that we are ready for examples the code snippets are in \ac{VHDL} the same calculations can be done in Verilog.

<TODO Subsection HDL with testbenches : INSERT CODE tbfirfilter>

In the testbench, the first thing to note is the name. I recommend that the name of the entity and filename include the \ac{DUT} name with the prefix tb, i.e. "tbfirfilter.vhd". If you stick to this naming convention you'll have your files organized nicely when you have larger designs. Next at the top of the file you should have a comments header where you can keep a log of what you have changed recently. Larger designs require a lot of changes and you will be glad to have some comments to remind you what this block does and how to use it. If there are assumptions on how the block is used, like the reset has to be held for multiple clock cycles or the block can not handle fully-pipelined data these are help when you use the block again in another design that may not have the same data-rate requirements.

The next section of code includes some standard libraries we need. After the library usage declarations we have the entity declaration which is empty. An empty entity declaration means we do not have any inputs, outputs, or generics used in the testbench. Usually testbenches do not need any inputs or outputs since its the highest level. It is possible to use generics to make automating testing easy but we will not be using them here.

The main and last section is divided into two parts. Starting with the keyword architecture and ending with begin, we have a list of all the signals we are going to use for this testbench. If we go back to the analogy of the testbench is a workbench this section has all the wires and components needed for putting the test together. Component declarations are not always necessary when they are instantiated they can be specified by listing the library they can be found in.

After the begin statement we can start to write our testbench logic. After the begin statement the code is organized by data flow. The first instantiation is the block that reads data. An output to this block is the source of the data. The signal that we wrote to the text file in MATLAB will by supplied here. Out of the textfileread block the data goes into the fir\_filter block. Here we see the input going into the component or \ac{DUT}. We have hooked up wires to the \ac{DUT} just like you would if you put a chip on a breadboard and connected physical wires to the chip. We now have the input going into the fir filter. Next we will hook up wires to the output and we can see the data coming out of the \ac{FIR} filter.

We have the output of the \ac{FIR} filter but we want to compare it to the known truth data from MATLAB so we have another instantiation of textfileread to read out the results from the MATLAB text file. After that we have a process that gets the next value from the text file when data is valid out of the \ac{FIR} filter. Once both are fetched we compare the two values and assert they are the same. If they are not the assert statement reports there was a simulation mismatch.

Near the end of the testbench we keep our processes that are simple. Like the clock process that generates a clock at the frequency we are going to use in the \ac{FPGA}. Also we have a process that releases the \ac{FIR} filter from reset. This process can be more complicated if you intend on resetting the \ac{FIR} filter regularly and want to ensure the filter transient is handled in a certain way. As an illustration of a testbench those details were left out for brevity.

<TODO Subsection HDL with testbenches : INSERT CODE textfileread>

We now are going to look into how the textfileread entity works. The first feature to notice is that the filename to read from is a generic. The generic allows us to reuse the block to read from different files. We saw this in the testbench were we read both the input and results from MATLAB with this same block. Another needed feature that is needed is the enable input to the block, which allows us to hold off the data until we are ready to start comparing with the result of the \ac{FIR} filter. The rest of the file consists of calling functions in the TextIO package.

	
	
	
\subsection{Synthesis and verify resource usage}
	<TODO Subsection Synthesis and verify resource usage : PROOF READ>

At this point we have a \ac{RTL} block that functionally works. The last step is making sure that the design is realizable in an \ac{FPGA}. There are a few reasons why a design could simulate but not be a good solution on an \ac{FPGA}. First a very common issue is accessing more than two addressing in an intended block \ac{RAM}. Most block \ac{RAM}s are dual port. Since its dual port there are two sets of access busses which can be used on a clock cycle. If your code uses more than that the block \ac{RAM} has to be instantiated as distributed \ac{RAM} this uses up a lot of logic.

Another common issue is that there is too much logic trying to be calculated under one clock cycle. There are two potential solutions to the problem. One you could do less in a clock cycle or two make the clock period longer. Slowing down the clock may be detrimental to the over all performance of the system. These are the two options to meet timing, once the changes are made the block should be functionally verified again.  

One more note on compiling a design in the vendor tools is that depending how you set the top level in the tools it may remove all the logic. To make sure this doesn't happen you should set the inputs to the \ac{DUT} to be inputs to the FPGA (using pin planner). Also check that the outputs are used in some way like using an or-reduce to make sure the tools can't optimize away ports that are actually needed in a bigger system. Make sure to check the synthesis and place and route reports to make sure resources are being used as expected. 
	
	<TODO Subsection Synthesis and verify resource usage : INSERT CODE synthesis report or resource usage>
	
\section{Code Coverage}
	<TODO Section Code Coverage : NOT DONE>

\subsection{Universal Verification Methodology (UVM)}
	<TODO Subsection Universal Verification Methodology (UVM) : NOT DONE>

\subsection{System Verilog}
	<TODO Subsection System Verilog : NOT DONE>



\chapter{FPGA Programming Basics}
	<TODO Chapter FPGA Programming Basics} : NOT DONE>

\section{Installing Software}
	<TODO Section Installing Software} : NOT DONE>

\subsection{Xilinx}
	<TODO Subsection Xilinx} : NOT DONE>

\subsection{Altera}
	<TODO Subsection Altera} : NOT DONE>

\subsection{Microsemi}
	<TODO Subsection Microsemi} : NOT DONE>

\subsection{Lattice}
	<TODO Subsection Lattice} : NOT DONE>

\section{Number Representation}
	<TODO Section Number Representation} : NOT DONE>

\subsection{Fixed Point}
	<TODO Subsection Fixed Point} : NOT DONE>

\subsection{Floating Point}
	<TODO Subsection Floating Point} : NOT DONE>

\section{Basic Gates and Analysis}
	<TODO Section Basic Gates and Analysis} : NOT DONE>

\subsection{Combinatorial Logic Analysis}
	<TODO Subsection Combinatorial Logic Analysis} : NOT DONE>

\subsubsection{Basic Adders}
	<TODO Subsubsection  Basic Adders} : NOT DONE>

\subsubsection{Decoders and Encoders}
	<TODO Subsubsection  Decoders and Encoders} : NOT DONE>

\subsubsection{Multiplexers and De-multiplexers}
	<TODO Subsubsection  Multiplexers and De-multiplexers} : NOT DONE>

\subsubsection{Parity Generators and Checkers}
	<TODO Subsubsection  Parity Generators and Checkers} : NOT DONE>

\subsection{Sequential Logic}
	<TODO Subsection Sequential Logic} : NOT DONE>

\subsubsection{Latches and Flip-flops}
	<TODO Subsubsection  Latches and Flip-flops} : NOT DONE>

\subsubsection{Counters}
	<TODO Subsubsection  Counters} : NOT DONE>

\subsubsection{Shift Registers}
	<TODO Subsubsection  Shift Registers} : NOT DONE>

\subsubsection{Memory and Storage}
	<TODO Subsubsection  Memory and Storage} : NOT DONE>

\section{VHDL Intro}
	<TODO Section VHDL Intro} : NOT DONE>

\subsection{Signal Types}
	<TODO Subsection Signal Types} : NOT DONE>

\subsection{Sequential Statements}
	<TODO Subsection Sequential Statements} : NOT DONE>

\subsection{Subprograms}
	<TODO Subsection Subprograms} : NOT DONE>

\subsection{State Machines}
	<TODO Subsection State Machines} : NOT DONE>

\subsection{Generics}
	<TODO Subsection Generics} : NOT DONE>

\subsection{File IO}
	<TODO Subsection File IO} : NOT DONE>

\section{VHDL Advanced}
	<TODO Section VHDL Advanced} : NOT DONE>
	
\subsection{Asynchronous Resets}
	<TODO Subsection Asynchronous Resets : NOT DONE>
		
\subsection{Hierarchy} 
	<TODO Subsection Hierarchy}  : NOT DONE>

\subsection{Pipelining}
	<TODO Subsection Pipelining} : NOT DONE>

\subsection{Data Rates}
	<TODO Subsection Data Rates} : NOT DONE>

\subsection{Clock Domain Crossing}
	<TODO Subsection Clock Domain Crossing : NOT DONE>	
	
\section{TCL Scripting}
	<TODO Section TCL Scripting} : NOT DONE>

\subsection{Version Update}
	<TODO Subsection Version Update} : NOT DONE>

\subsection{Automate Simulations}
	<TODO Subsection Automate Simulations} : NOT DONE>

\subsection{Automate Build Tools}
	<TODO Subsection Automate Build Tools} : NOT DONE>

\chapter{Single Board Computer Programming Basics}
	<TODO Chapter Single Board Computer Programming Basics : NOT DONE>

Once you get your \ac{SBC} in the mail and you open it up what do you do? You read the getting started guide, you plug in the the \ac{SD} card, plug in a monitor, plug in a keyboard and mouse. The \ac{SBC} boots up to a linux desktop or command line and waits for your input. 

This Chapter is about what you can do with the \ac{SBC} after it boots up. There are many projects you can do with the \ac{SBC} and we will discuss the most popular \ac{SBC}, the \ac{RPi}. We will show examples of how to use the \ac{RPi} but most of the examples here apply to any linux distribution. 

\section{Linux Basics}
	<TODO Section Linux Basics : NOT DONE>

The first thing that you will need to get used to in Linux is the command line. At the command line the programmer types in the commands they want to run. In this interface processing power is not wasted on rendering the \ac{GUI} or handling \ac{UI} events. The overhead of a \ac{GUI} takes away from processing the tasks you are trying to perform. 

Next we will introduce aliases to make issuing common commands easier. Aliases allow you to use a shorter phase for common commands you execute, this is particularly helpful if the commands being ran have many parameters that are the same. Aliases save keystrokes per command, so long commands that are ran frequently are usually first to be aliased. 

We will then turn to the concept of \emph{Cron-Jobs}. In Linux we can setup a Cron-Job that can be ran any interval with a one minute resolution. Once scheduled, a script can be ran to scan for malware, move files, or clean up some old data. A good candidate script for a Cron-Job is one that needs regularly ran and takes a while to perform. 

Finally we discuss popular programming languages. We will discuss the basics of three languages. The first we look at C/C++. The strength of C/C++ is that it is a low level language with a high efficiency rating. Next is Python, which is gaining a lot of popularity, particularly in machine learning applications. Since Python is more abstract from bare metal we sacrifice efficiency but we gain a lot from Python. Finally, we discuss a newer language named \emph{Julia}, which shows many promising attributes.  
	
\subsection{The Terminal}
	<TODO Subsection The Terminal : NOT DONE>

\subsection{Aliases}
	<TODO Subsection Aliases : NOT DONE>
	
\subsection{Cron-Jobs}
	<TODO Subsection Cron-Jobs : NOT DONE>

\section{Programming Languages}
	<TODO Section Programming Languages : NOT DONE>

\subsection{C/C++ vs. Python vs. Julia}
	<TODO Subsection C/C++ vs. Python vs. Julia : NOT DONE>

\subsubsection{C/C++}
	<TODO Subsubsection  C/C++ : NOT DONE>

\subsubsection{Python}
	<TODO Subsubsection  Python : NOT DONE>

\subsubsection{Julia}
	<TODO Subsubsection  Julia : NOT DONE>

\section{Controlling GPIO}
	<TODO Section Controlling GPIO : NOT DONE>


\chapter{Hello World and Other Projects}
	<TODO Chapter Hello World and Other Projects : PROOF READ>

There is no better way to get started than to work on a small project. This chapter outlines small projects to work on where each of them gets a little more challenging. As you progress through this chapter you will ensure your development environment is setup properly with simulating the hello world project. You will learn how to use a clock in a design and control an \ac{LED}. Then you will learn how to transfer data into and out of the \ac{FPGA} chip. 
	
	
\section{Beginner Projects}
	<TODO Section Beginner Projects : PROOF READ>

Everyone has to start somewhere. These projects are designed to start off slowly. It may not be very exciting to add one to a counter but these exercises get you used to writing \ac{VHDL} and the development environment. Work through these exercises so that when we can get to exciting projects you won't be frustrated with working with the tools. 
	
\subsection{Hello World}
	<TODO Subsection Hello World : PROOF READ>

When learning a new language you always need to write the hello world program. There are two lessons from this project. First, setting up and interacting with the features of your simulator. Once you write the first draft of your program you go through the steps of compiling and simulating your design. You will use this process many times. Gain an understanding of keyboard short cuts and practice them. Make changes to your file and recompile and ensure the changes are reflected when you re-simulate.

The other main take-away from this example is the use of the report function. The report function allows you to print strings out to the console. The report function can be used to print out values of signals to make sure the simulation is progressing like you want in a larger more complex simulation. Or it can be used to report progress of the simulation.      

\begin{VHDLlisting}[tabsize=8]
library ieee;
  use ieee.std_logic_1164.all;
  use ieee.numeric_std.all;
  
entity hello_world is
end entity hello_world;

architecture tb of hello_world is
begin
	process
	begin
		report("Hello World!");
	end process;
end tb;
\end{VHDLlisting}
	
	
\subsection{Counter Test-Bench}
	<TODO Subsection Counter Test-Bench : PROOF READ>

In this exercise we will instantiate a \ac{DUT} in a testbench. The \ac{DUT} itself is a simple counter. We will provide a clock to the counter and a reset line that will reset the counter. We will also have an enable line that turns the counter on. A simple \ac{DUT} like this has a few inputs that will need to be driven by the testbench.      
	
\begin{VHDLlisting}
library ieee;
  use ieee.std_logic_1164.all;
  use ieee.numeric_std.all;
  
entity tb_counter is
end entity tb_counter;

architecture tb of tb_counter is
	signal w_clk         : std_logic := '0';
	signal w_rst         : std_logic := '0';
	signal w_en          : std_logic := '0';
	signal w_count_sync  : unsigned(7 downto 0):=(others => '0');
	signal w_count_async : unsigned(7 downto 0):=(others => '0');

begin
	p_gen_stim : process
	begin
		w_rst <= '1';
		wait for 37.9 ns;
		w_rst <= '0';
		
		wait for 10 ns;
		w_en <= '1';
		wait for 50 ms;	
	end process;
	
	u_counter_sync : entity work.counter_sync
	port map(i_clk    => w_clk,
			 i_rst    => w_rst,
			 i_en     => w_en,
			 o_count  => w_count_sync
	);
	
	u_counter_async : entity work.counter_async
	port map(i_clk    => w_clk,
			 i_arst   => w_rst,
			 i_en     => w_en,
			 o_count  => w_count_async
	);
	
	p_clk : process
	begin
		wait for 5 ns;
		w_clk <= not w_clk;
	end process;
end tb;
\end{VHDLlisting}

In our testbench we have a process that generates the timing for the input signals. This testbench is only as good as the assumptions made in the timing between signals. If there is a race condition or an inter-timing bug in the \ac{DUT} that the testbench does not cover then the \ac{DUT} will pass the testbench giving you confidence in the \ac{DUT}. Then you have the worst case scenario: debugging a flaky build on hardware. So we need to make sure we make our code reliable to ensure there are no race conditions. In this exercise we look at two examples of a counter. The only difference in the two examples is how the reset is handled, synchronously or asynchronously. We will explain the considerations needed when using this counter in a higher level block.      

\begin{VHDLlisting}[tabsize=8]
library ieee;
  use ieee.std_logic_1164.all;
  use ieee.numeric_std.all;
  
entity counter_sync is
port(i_clk   : in    std_logic;
	 i_rst   : in    std_logic;
	 i_en    : in    std_logic;
	 o_count :   out unsigned(7 downto 0)	
)
end entity counter_sync;

architecture rtl of counter_sync is
	-- Output Register
	signal f_count : unsigned(7 downto 0) := (others => '0');
begin

	-- Assign Outputs
	o_count <= f_count;

	p_count : process(i_clk)
	begin
		if rising_edge(i_clk) then
			if i_rst = '1' then
				f_count <= (others => '0');
			elsif i_en = '1' then
				f_count <= f_count + 1;
			end if;		
		end if;
	end process;
end rtl;

\end{VHDLlisting}
	
First the synchronous reset counter. We see in the code that the reset is inside the rising\_edge(i\_clk) if statement. That means that the reset can only happen synchronous to the i\_clk. We can see that both the reset and enable are synchronous. In general synchronous logic is preferred since the vendor tools ensure the propagation delays between the registers are short enough to avoid erroneous results.


\begin{VHDLlisting}[tabsize=8]
library ieee;
  use ieee.std_logic_1164.all;
  use ieee.numeric_std.all;
  
entity counter_async is
port(i_clk   : in    std_logic;
	 i_arst  : in    std_logic;
	 i_en    : in    std_logic;
	 o_count :   out unsigned(7 downto 0)	
)
end entity counter_async;

architecture rtl of counter_async is
	-- Output Register
	signal f_count : unsigned(7 downto 0) := (others => '0');
begin

	-- Assign Outputs
	o_count <= f_count;

	p_count : process(i_clk)
	begin
		if i_arst = '1' then
			f_count <= (others => '0');
		elsif rising_edge(i_clk) then
			if i_en = '1' then
				f_count <= f_count + 1;
			end if;		
		end if;
	end process;
end rtl;

\end{VHDLlisting}
	
There are situations when you have to use asynchronous resets. If we assume we will use this counter in a situation where we have to use an asynchronous reset we would write this block slightly differently where the reset line is used outside the clocked region of the process. Now if the i\_reset line changes or the i\_clk line changes the process executed but there doesn't have to be a rising\_edge event for the reset to apply. There is potential issue with instantiating this block with an asynchronous reset. It is possible that the reset fails timing. If the asynchronous reset is applied right before a clock edge where the counter is enabled there could be a mismatch between the expected value in the counter and the registered value. Unfortunately, the vendor tools can't warn you about the asynchronous reset failing timing since it isn't under a clocked process. The vendor tools can't know the timing of the asynchronous reset.	
	
	
	
\subsection{Blink Led with a One Second Interval}
	<TODO Subsection Blink Led with a One Second Interval : PROOF READ>

In the project you will write \ac{HDL} to blink an LED. The LED should be on for a second and off for a second. You will also need to write a testbench for the block. You will find that it will take a long time to simulate a block for multiple seconds. We will once again have a synchronous reset, enable, and a clock as inputs. The output will be wired up to an LED.

To toggle the output we need to know when a second passes. The only way to mark time passing on an \ac{FPGA} is to count clock cycles. So we can setup a counter to increment when enabled and once we get to a certain number of clock cycles we can toggle the output. 

\begin{VHDLlisting}[tabsize=8]
library ieee;
  use ieee.std_logic_1164.all;
  use ieee.numeric_std.all;
  
entity blink_led is
generic(gTOGGLECOUNT : integer)
port(i_clk   : in    std_logic;
	 i_rst   : in    std_logic;
	 i_en    : in    std_logic;
	 o_led   :   out std_logic	
)
end entity blink_led;

architecture rtl of blink_led is
	-- Output Register
	signal f_led   : std_logic := '0';

	-- Duration Counter
	signal f_count : unsigned(31 downto 0) := (others => '0');
begin

	-- Assign Outputs
	o_led <= f_led;

	p_count : process(i_clk)
	begin
		if rising_edge(i_clk) then
			if i_rst = '1' then
				f_count <= (others => '0');
			elsif i_en = '1' then
				f_count <= f_count + 1;
			end if;	

			if f_count = gTOGGLECOUNT then
				f_led <= not f_led;
				f_count <= (others => '0');
			end if;			
		end if;
	end process;
end rtl;

\end{VHDLlisting}

The certain number we need to count to is just the frequency of the clock you re using. Here is a fantastic opportunity to learn about generics. Generics can help you configure a block when instantiated. We could use a generic called gCLKFREQ but if we wanted the output to toggle at a rate other than a second this could be confusing. We will use the generic gTOGGLECOUNT. This way if we know how the counter is used and when instantiated we can set the value accordingly.
	
\begin{VHDLlisting}[tabsize=8]
library ieee;
  use ieee.std_logic_1164.all;
  use ieee.numeric_std.all;
  
entity tb_blink_led is
end entity tb_blink_led;

architecture tb of tb_blink_led is
	constant k_TOGGLECOUNT : integer := 100000000;

	signal w_clk : std_logic := '0';
	signal w_rst : std_logic := '0';
	signal w_en  : std_logic := '0';
	signal w_led : std_logic := '0';
begin

	p_gen_stim : process
	begin
		w_rst <= '1';
		wait for 34 ns;
		w_rst <= '0';
		
		wait for 15 ns;
		w_en <= '1';
		
		wait;
	end process;

	u_blink_led : entity work.blink_led
	generic map(gTOGGLECOUNT => k_TOGGLECOUNT)
	port map(i_clk   => w_clk,
		     i_rst   => w_rst,
		     i_en    => w_en ,
		     o_led   => w_led
	);

	p_clk : process
	begin
		wait for 5 ns;
		w_clk <= not w_clk;
	end process;	
end tb;
\end{VHDLlisting}

The testbench is simple with the reset, enable, and clock to generate. The difficult part of this testbench though is how long it will take to simulate. At first when you get the testbench working you don't want to have to wait 30 minutes for the simulation to finish. So at first we will set gTOGGLECOUNT to 10 and make sure the logic works. After getting the simulation two work then we can increase it and make sure the output toggles at the expected rate. 

\section{IO Interfacing}
	<TODO Section IO Interfacing : PROOF READ>
	
This section of projects gets data in to and out from the \ac{FPGA}. The list of projects here are the easiest ways of setting up two way data transfer off the chip. These protocols are popular in interacting with a computer or with other chips on the same board.

Although these projects are simple and great starting exercises you will reuse these blocks a lot when interfacing to different peripherals. We will discuss making these blocks generic so that they can be used with minimal changes on a variety of projects.

\subsection{Universal Asynchronous Receiver-Transmitter (UART)}
	<TODO Subsection Universal Asynchronous Receiver-Transmitter (UART) : PROOF READ>
	
The \ac{UART} is a very common low speed protocol. It's primary use is commanding and controlling the FPGA to do a task when a user issues the command through a general purpose computer. The \ac{UART} is also great for status or debugging what is happening on the \ac{FPGA}.

A \ac{UART} is an asynchronous protocol so there is no clock associated with the data. Since there is no clock or an enable line to know when the data is valid we first need to know the baudrate of the data. It is assumed that the datarate is known between the computer and \ac{FPGA}. Since the datarate is known once we get a bit we will know when the next bit is valid. To know when the first bit is valid the \ac{UART} protocol starts with the receive line high. Data is starting to be transmitted when the line goes low. At the falling edge of the data line we count half the bit duration. Once half the bit duration is reached we check again to make sure the data line is low. If the data line is low then we know we are receiving data. Once we have started the transfer the predetermined number of bits are sent. After the falling edge of data line there is a start bit, a stop bit, and a configurable number of data bits. The start bit is what we used to detect the start of the transfer. Then we count a bit duration for each of the data bits. Finally the stop bit can be ignored or sometimes can be configured as a parity bit.  Now we will look at some \ac{HDL} that performs the above tasks. We will use a state machine for this since it is easier to read. 

	
	
\begin{VHDLlisting}[tabsize=8]
library ieee;
use ieee.std_logic_1164.all;
use ieee.numeric_std.all;

entity uart_rx is
  generic(g_baudRate : integer := 115200;
          g_clkRate  : integer := 50000000);
  port(
      i_clk           : in    std_logic;
      i_rx            : in    std_logic;
      o_rx_data       : out   std_logic_vector(7 downto 0);
      o_rx_data_rdy   : out   std_logic
      );
end entity uart_rx;

architecture rtl of uart_rx is
  constant k_timeperbit   : real := real(1)/real(g_baudRate);
  constant k_clkperiod    : real := real(1)/real(g_clkRate);
  constant k_clksPerBit   : integer := integer(real(g_clkRate)/real(g_baudRate));
  constant k_clksPerBitd2 : integer := integer(real(k_clksPerBit)/real(2));
  constant k_TBits        : unsigned(3 downto 0) := to_unsigned(10,4);

  -- Outputs 
  signal f_rx_data     : std_logic_vector(7 downto 0) := (others => '0');
  signal f_rx_data_rdy : std_logic := '0';
  
  type sm_rxUart is (s_idle, s_cfmLow, s_rxStartBit, s_rx8bits, s_parity, s_stop, s_reset);
  signal f_cState : sm_rxUart := s_idle;
  
  signal f_clkCount : integer := 0;
  signal f_rxBits   : unsigned(3 downto 0) := (others => '0');
  
begin
  p_rxctrl : process(i_clk) 
  begin
    if rising_edge(i_clk) then
      o_rx_data_rdy <= f_rx_data_rdy;
      o_rx_data <= f_rx_data;
      case f_cState is
        when s_idle => 
          f_rx_data_rdy <= '0';
          f_rx_data <= (others => '0');
          if i_rx = '0' then
            f_cState <= s_cfmLow;
          end if;
        when s_cfmLow =>
          if f_clkCount = k_clksPerBitd2 then
            if i_rx = '0' then
              f_cState <= s_rxStartBit;
              f_clkCount <= 0;  
            else 
              f_cState <= s_idle; -- if rx line isn't still low during start bit.
              f_clkCount <= 0;
            end if;
          else
            f_clkCount <= f_clkCount + 1;
          end if;
        when s_rxStartBit =>
          if f_clkCount = k_clksPerBit then
            f_cState <= s_rx8bits;
            f_rx_data(to_integer(f_rxBits)) <= i_rx;
            f_rxBits <= f_rxBits + 1;
            f_clkCount <= 0;
          else 
            f_clkCount <= f_clkCount + 1;
          end if;        
        when s_rx8bits =>
          if f_rxBits = 8 then
            f_cState <= s_parity;
          elsif f_clkCount = k_clksPerBit then 
            f_clkCount <= 0;
            f_rx_data(to_integer(f_rxBits)) <= i_rx;
            f_rxBits <= f_rxBits + 1;
          else 
            f_clkCount <= f_clkCount + 1;
          end if;
        when s_parity =>
          if f_clkCount = k_clksPerBit then
            f_rxBits <= (others => '0');
            f_clkCount <= 0;
            if i_rx = '1' then
              f_cState <= s_stop;
            else
              f_cState <= s_idle;
            end if;
          else
            f_clkCount <= f_clkCount + 1;
          end if;
        when s_stop => 
          if f_clkCount = k_clksPerBit then
            f_clkCount <= 0;
            if i_rx = '1' then
              f_cState <= s_reset;
            else
              f_cState <= s_idle;
            end if;
          else
            f_clkCount <= f_clkCount + 1;
          end if;
        when s_reset =>
          if f_clkCount = k_clksPerBitd2 then
            f_rx_data_rdy <= '1';
            f_cState <= s_idle;
            f_rxBits <= (others => '0');
          else
            f_clkCount <= f_clkCount + 1;
          end if;
      end case; 
    end if;
  end process;
end rtl;
\end{VHDLlisting}

The five states of the state machine are idle, start, check start, acquire bits, and stop. As a side note state machines can use more resources. If you really wanted to squeeze out all the resources needed then you could rewrite the below code to not use a state machine and save some resources.The first state is s\_idle. In this state we are only concerned with whether or not the data has started. To start a UART transaction the data line is deasserted. We will use a falling edge detection circuit to determine is if the transaction has started. Once the falling edge is detected we move to s\_check.

In s\_check we count a half a bit's duration. The half a bit duration ensures we are sampling at the middle of a bit window. Since the clocks at the transmitter and receiver aren't synchronized we need to sample in the middle of the bit to gain the most reliable data. Once we have counted half a bits duration we can check that the start bit is low, if not we reset back to s\_idle, however if the start bit is low we go to s\_getbits.

In s\_getbits we will receive all the bits in a UART transaction. The number of bits in a transaction is variable but the most common is a byte or eight bits. So, after the start bit is confirmed to be low and we transition to this state we need to wait a full bit duration and sample the data line. We store the sampled data in a shift register. Once we have counted up the number of bits we continue on the the s\_stop state. In the s\_stop state we make sure the data line is high after the next but duration. Some configurations of the UART protocol have a parity bit. In this state is where you calculate the parity and check the integrity of the received data. 

\begin{VHDLlisting}[tabsize=8]
library ieee;
  use ieee.std_logic_1164.all;
  use ieee.numeric_std.all;
  use ieee.math_real.all;
  
entity uart_tx is
  generic(g_baudRate : integer := 115200;
          g_clkRate  : integer := 50000000);
  port(
      i_clk        : in    std_logic;
      i_sclr       : in    std_logic;
      i_tx_data    : in    std_logic_vector(7 downto 0);
      i_tx_data_dv : in    std_logic;
      o_tx         :   out std_logic;
      o_tx_busy    :   out std_logic
      );
end uart_tx;
architecture rtl of uart_tx is 
  constant k_timeperbit : real := real(1)/real(g_baudRate);
  constant k_clkperiod  : real := real(1)/real(g_clkRate);
  constant k_clksPerBit : integer := integer(real(g_clkRate)/real(g_baudRate)); 
  constant k_TBits      : unsigned(3 downto 0) := to_unsigned(10,4);
  
  type sm_uarttx_ctrl is (s_idle, s_txing, s_reset);
  signal f_cState : sm_uarttx_ctrl := s_idle;
  
  signal f_data2Tx : std_logic_vector(10 downto 0) := (others => '0');
  signal f_clkCount: integer := 0;
  signal f_bitCount: unsigned(3 downto 0) := (others => '0');
  
  -- Register Outs
  signal f_tx : std_logic := '1';
  signal f_tx_busy : std_logic := '0';
begin
  p_ctrl : process(i_clk) 
  begin
    if i_sclr = '1' then
        f_data2Tx <= (others => '0');
    elsif rising_edge(i_clk) then
      o_tx <= f_tx;
      o_tx_busy <= f_tx_busy;
      case f_cState is  
        when s_idle =>
          if i_tx_data_dv = '1' then
            f_data2Tx(8 downto 1) <= i_tx_data;
            f_data2Tx(10 downto 9) <= b"11";
            f_tx <= '0';
            f_tx_busy <= '1';
            f_cState <= s_txing;
            f_bitCount <= f_bitCount + 1;
          end if;
        when s_txing => 
          if f_clkCount = k_clksPerBit then
            f_clkCount <= 0;
            f_tx <= f_data2Tx(to_integer(f_bitCount));
            f_bitCount <= f_bitCount + 1;
            if f_bitCount = k_TBits then
              f_cState <= s_reset;
            end if;
          else
            f_clkCount <= f_clkCount + 1;
          end if;
        when s_reset => 
          f_tx <= '1';
          f_tx_busy <= '0';
          f_cState <= s_idle;
          f_bitCount <= (others => '0');
      end case;
    end if;
  end process;
end rtl;
\end{VHDLlisting}
	
The \ac{UART} transmitter operates in a similar fashion. The transmitter's job is is ensure the data is put on the line with the same timing the receiver expects. The data in loaded into the core in parallel but it output with a shift register. After a bit duration has passed we shift another bit out of the shift register until all the bits are sent. Then the transaction is ended.
	

\subsection{Serial Peripheral Interface (SPI)}
	<TODO Subsection Serial Peripheral Interface (SPI) : PROOF READ>

\ac{SPI} is commonly used for inter-chip communication where a chip is configured by a SPI master. The SPI master initiates all traffic between the SPI master and SPI slave. The slave chip only responses to commands from the master. The SPI bus consists of four wires; clock, chip select, input data, and output data. Since there is a clock we know this protocol is synchronous. The clock is driven by the master and the clock rate is the only way to change the data rate of the communications. In general the \ac{SPI} bus is capable of a higher data rate than the UART.

The rest of the lines generally work as follows but some chips use a slight variation to the descriptions below. General SPI code can be altered to meet specific chip standards. The datasheet for the chip you are trying to interface to should be consulted to ensure proper operation. 

The \ac(CS) line is used to enable communications with the chip. The reason for the name is that you can reuse the same clock and data lines but change only the CS lines for configuring multiple chips. In this way you save \ac{IO} pins on the \ac{FPGA}. The CS line is used to indicate that communication is starting. \ac{CS} can be active low or active high. Usually with active low signals an \emph{n} is appended to the name so if our CS line were active low it could be called CSn. 

The last lines of the SPI bus to discuss are the data lines. These lines are named \ac{MOSI} and \ac{MISO}. The names of course tell you the direction and who drives the pins. Sometimes if \ac{IO} pins are limited there is a shared data line. This is possible because it is understood that the master controls communication if the master is reading data from the slave then the master releases control of the data line in time for the slave to control the line. Care must be taken to ensure that there isn't contention on the line. If both slave and master drive the pin communication will fail. 	

Next we can look at an implementation of a \ac{SPI}-Master. Shown here

\begin{VHDLlisting}[tabsize=4]
-- spi_master.vhd

library ieee;
	use ieee.std_logic_1164.all;
	use ieee.numeric_std.all;
	
entity spi_master is
	generic(g_sclkrate  : integer;
			g_word_size : integer)
	port(i_clk      : in    std_logic;
		 i_rst      : in    std_logic;
		 
		 o_spi_clk  :   out std_logic;
		 o_spi_mosi :   out std_logic;
		 i_spi_miso : in    std_logic;
		 o_spi_ncs  :   out std_logic;
		
		 i_rd_nwr   : in    std_logic;
		 i_wrdataen : in    std_logic;
		 i_wr_data  : in    std_logic_vector(g_word_size-1 downto 0);
		
		 o_rd_data  :   out std_logic_vector(g_word_size-1 downto 0);
		 o_rd_dv    :   out std_logic;
		 o_busy     :   out std_logic	
	);
end entity spi_master;

architecture rtl of spi_master is
	signal f_clk_counter : unsigned(31 downto 0);
	signal f_o_sclk      : std_logic := '0';
	signal f_o_mosi      : std_logic := '0';
	signal f_o_ncs       : std_logic := '0';
	
	signal f_o_rd_dv     : std_logic := '0';
	signal f_rd_data     : std_logic_vector(g_word_size-1 downto 0);
	
	type sm_spi is (s_idle, s_transfer, s_end);
	signal s_spi         : sm_spi := s_idle;
	
	signal f_wr_data     : std_logic_vector(g_word_size-1 downto 0);
	signal f_rx_data     : std_logic_vector(g_word_size-1 downto 0);
	
	signal f_data_c      : unsigned(7 downto 0);
	signal f_data_rd_c   : unsigned(7 downto 0);
	
	
begin

	o_busy      <= f_busy;
	o_spi_clk   <= f_o_sclk; 
    o_spi_mosi  <= f_o_mosi;
    o_spi_ncs   <= f_o_ncs; 
	
	o_rd_data <= f_o_rd_dv;
	o_rd_dv   <= f_rd_data;

	p_gen_sclk : process(i_clk)
	begin
		if rising_edge(i_clk) then
			if i_rst = '1' then
				f_clk_counter <= (others => '0');
			else
				f_clk_counter <= f_clk_counter + 1;
				if f_clk_counter = g_sclkrate then
					f_o_sclk <= '0';
					f_clk_counter <= (others => '0');
				elsif f_clk_counter = g_sclkrate/2 then
					f_o_sclk <= '1';
				end if;
			end if;		
		end if;
	end process;

	p_launch_data : process(i_clk)
	begin
		if rising_edge(i_clk) then
			if i_rst = '1' then
				s_spi <= s_idle;
				f_busy <= '0';
				f_o_ncs <= '1';
			else
				ff_o_sclk <= f_o_sclk;
				case s_spi is
					when idle =>
						f_busy <= '0';
						f_o_ncs <= '1';
						f_o_rd_dv <= '1';
						if i_wrdataen = '1' then
							s_spi <= s_transfer;
							f_wr_data <= i_wr_data;
							f_busy <= '1';
							f_rd_nwr <= i_rd_nwr;
							f_data_c <= to_unsigned(f_wr_data'high,f_data_c'length);
							f_data_rd_c <= (others => '0');
						end if;
					when s_transfer => 
						-- Falling Edge of SCLK
						if ff_o_sclk = '1' and f_o_sclk = '0' then
							f_o_ncs <= '0';
							f_o_mosi <= f_wr_data(to_integer(f_data_c));
							f_data_c <= f_data_c - 1;
							if f_data_c = 0 then
								f_data_c <= to_unsigned(f_wr_data'high,f_data_c'length);
								s_spi <= s_end;
							end if;
						end if;					

						-- Rising Edge of SCLK
						if ff_o_sclk = '0' and f_o_sclk = '1' then
							f_rx_data(to_integer(f_data_rd_c)) <= i_spi_miso;
							f_data_rd_c <= f_data_rd_c + 1;
						end if;
					when s_end => 
						s_spi <= s_idle;
						if f_rd_nwr = '1' then
							-- Reading data from slave
							f_o_rd_dv <= '1';
							f_rd_data <= f_rx_data;
						end if;				
				end case;
			end if;
		end if;	
	end process;


end rtl;
\end{VHDLlisting}

This \ac{VHDL} uses a state machine to control when the data is sent and received. We see in the first process the \ac{SPI} clock is generated and sent out the \emph{SCLK} port. There may be certain circumstances where you don't want to clock always running. You can gate the output of the clock with the busy signal to ensure they clock is running only when it is needed. 

Looking closer at the \ac{VHDL} first we see we use two generics. The first generic is \emph{g\_sclkrate}, which is used to determine the \ac{SPI} clock speed. The \ac{SPI} clock speed is the clock speed of i\_clk divided by \emph{g\_sclkrate}. So if we have a $100$\ac{MHz} clock input and \emph{g\_sclkrate} is $4$ then \emph{o\_spi\_clk} will have a frequency of $25$\ac{MHz}. The generic helps us use the same spi\_master code for different chips that have different interfacing speeds. 

Next we have a the generic \emph{g\_word\_size}. The generic determines the number of bits being transferred in one transaction. The ports of our entity first list our clock and reset signals. Then we have the four \ac{IO} pins for the \ac{SPI} protocol. Then we have a flag that says whether we are writing data to the slave or reading data from the slave. Next we have the write data and write data valid. Followed by the read data and read data valid. And lastly we have a busy signal that lets the instantiating core know that we are currently busy spending out data. This is important because in one clock cycle the \ac{SPI}-Master is given \emph{g\_word\_size} data bits, but many clock cycles later, depending on the \ac{SPI} clock speed, the transfer is complete. 

Next we look in the architecture of the \ac{SPI}-Master and in the first process of the architecture we generate the \ac{SPI} clock, denoted \emph{f\_o\_sclk}. We use a counter that counts up to \emph{g\_sclkrate}. As the counter gets to half of \emph{g\_sclkrate} we deassert the clock then when we reach \emph{g\_sclkrate} we assert the clock. 

The next process launches data on the \ac{SPI} bus. First we wait in the state \emph{s\_idle} until write data valid is asserted. Whether we are reading data from the slave or writing data to the slave we have to start the transaction by have the master enable the \ac{CS} line. The amount of data includes the amount of data that will be read back from the slave as well. Usually the \ac{SPI} master writes an address and the rest of the data is the data at the specified address.

Once data is loaded into the \ac{SPI}-Master the state machine transitions to the \emph{s\_transfer} state. In this state we first look for the falling edge of the \ac{SPI} clock, at which, we assign data to the \ac{MOSI} line starting with the \ac{MSB}. We wait for falling edges over and over again until we have launched all the data. Then we transition to the \emph{s\_end} state. But also in the \emph{s\_transfer} state we look for the rising edge of the \ac{SPI} clock. Here we register what is coming in on the \ac{MISO} line. 

Once we are in the \emph{s\_end} state we go back to \emph{s\_idle}, but before we do if we were performing a read operation then we set the outputs providing valid data at the output ports. 

Now we should go back and discuss the reasoning for using the rising edge and falling edge of the \ac{SPI} clock. We launch the data on the falling edge of the \ac{SPI} clock because we want to have the maximum possible time for the data to be stable when the receiver of the data samples it at the rising edge. Just like when we wait for the rising edge of the clock to register the input we want the data to be stable. 

\subsection{Inter-Integrated Circuit (I2C) Bus}
	<TODO Subsection Inter-Integrated Circuit (I2C) Bus : PROOF READ>

The \ac{I2C} bus communication protocol is similar to the \ac{SPI} interface in that it is used for inter chip communication. The major advantage to \ac{I2C} is that is uses half the lines as \ac{SPI}. You may be thinking who cares about the number of \ac{IO} pins a chip takes but this can be an issue with interfacing to a few peripherals, especially when a few of them are high speed \ac{LVDS}. In this case saving a few \ac{IO} pins per chip is necessary to ensure \ac{IO} pins are available for a high speed \ac{ADC} or \ac{DDR} memory chip. 	
	
In the \ac{I2C} bus there are two wires, clock and data. And similar to \ac{SPI} each chip that you interface to will have a different clock speed and signal timing associated with the chip. For the \ac{I2C} example we show what happens when we know the chip has 32 bits per transaction. 	
	
In the \ac{I2C} protocol data is sent in eight bit chunks. So to interface to the 32 bit transaction size we break the transaction down into four - eight bit sections and ensure we have an acknowledge bit at the end of each eight bit section. 
	
We can look at some example code to see how this is done in \ac{VHDL}.	

\begin{VHDLlisting}[tabsize=4]
-- i2c_write.vhd

library ieee;
    use ieee.std_logic_1164.all;
    use ieee.numeric_std.all;


entity i2c_write is
    port(i_clk        : in    std_logic;
         i_rst        : in    std_logic;
         -- I2C Ports 
         b_sda        : inout std_logic;
         o_scl        :   out std_logic;
         -- App 
         i_data       : in    std_logic_vector(31 downto 0);
         i_dv         : in    std_logic;
         o_busy       :   out std_logic
    );
end entity i2c_write;

architecture rtl of i2c_write is 
    constant k_start_cond     : integer := 144;
	
    constant k_clks_per_scl_h : integer := 500;
    constant k_clks_per_scl   : integer := k_clks_per_scl_h*2;
	constant k_end_cond       : integer := 144 + k_clks_per_scl_h;
	
    -- Input Regs
    signal f_data    : std_logic_vector(31 downto 0) := (others => '0');
    	
    -- Output Regs
    signal f_busy    : std_logic := '0';
    signal f_sda     : std_logic := '1';
    signal f_scl     : std_logic := '1';

	constant k_n_data_frames : integer := 3;
    signal f_cc      : unsigned(11 downto 0) := (others => '0');
    signal f_os      : unsigned(7 downto 0) := (others => '0');
	signal f_ndframes: unsigned(3 downto 0) := (others => '0');
    signal f_ack_bit : std_logic := '1';
	signal f_done    : std_logic := '0';

    type sm_states is (s_idle, s_init_frame, s_reset_start_cond, s_ack_bit, s_write_data_frame, s_end);
    signal s_i2c  : sm_states := s_idle;
begin

    o_busy <= f_busy;

    b_sda <= f_sda;
    o_scl <= f_scl;

    p_i2c : process(i_clk)
    begin
        if rising_edge(i_clk) then
            if i_rst = '1' then
                s_i2c <= s_idle;
				f_sda <= '1';
                f_scl <= '1';
				f_done <= '0';
            else
                case s_i2c is
                    when s_idle => 
                        f_sda <= '1';a
                        f_scl <= '1';
						f_done <= '0';
						f_ack_bit <= '1';
						f_os <= to_unsigned(31,8);
                        if i_dv = '1' then
                            f_data <= i_data;
                            s_i2c <= s_init_frame;
                            f_sda <= '0'; -- Assumes point to point I2C
                            f_busy <= '1';
                        end if;
                    when s_init_frame => 
                        -- Start Condition deassert SDA
                        f_cc <= f_cc + 1;
                        if f_cc = k_start_cond then
                            f_cc <= (others => '0');
                            f_scl <= '0';
                            f_sda <= f_data(to_integer(f_os));
                            s_i2c <= s_write_data_frame;
                        end if;
					when s_reset_start_cond => 
						f_cc <= f_cc + 1;
						f_sda <= '1';
                        if f_cc = k_end_cond then
                            f_cc <= (others => '0');
							s_i2c <= s_init_frame;
                            f_sda <= '0';
                        end if;
                    when s_ack_bit => 
                        f_cc <= f_cc + 1;
                        if f_cc = k_clks_per_scl_h then
                            f_scl <= '1';
                            f_ack_bit <= b_sda;
                        end if; 
                        if f_cc = k_clks_per_scl then
							f_scl <= '0';
							f_cc <= (others => '0');
                            if f_ack_bit = '0' then
                                -- Success
								if f_done = '0' then
									s_i2c <= s_write_data_frame;
									f_sda <= f_data(to_integer(f_os));
								else
									f_sda <= '0';
									s_i2c <= s_end;
								end if;
                            else
                                -- Fail -- Reset
                                s_i2c <= s_idle;
                            end if;
                        end if;
                    when s_write_data_frame => 
                        f_cc <= f_cc + 1;
						f_ack_bit <= '1';
						
                        if f_cc = k_clks_per_scl_h then
                            f_scl <= '1';
                            f_os <= f_os - 1;
                        end if; 
						
                        if f_cc = k_clks_per_scl then
							f_scl <= '0';
                            f_cc <= (others => '0');
							if f_os = 23 then
								f_sda <= 'Z';
								s_i2c <= s_ack_bit;
								f_done <= '0';
                            elsif f_os = 15 then
								f_sda <= 'Z';
								s_i2c <= s_ack_bit;
								f_done <= '0';
                            elsif f_os = 7 then
								f_sda <= 'Z';
								s_i2c <= s_ack_bit;
								f_done <= '0';
                            elsif f_done = '1' then
								f_sda <= 'Z';
                                s_i2c <= s_ack_bit;
							else
								f_sda <= f_data(to_integer(f_os));
								if f_os = 0 then
									f_done <= '1';
								end if;
                            end if;
                        end if;
					when s_end =>
						f_cc <= f_cc + 1;
						if f_cc = k_clks_per_scl_h then
                            f_scl <= '1';
                        end if; 
						
                        if f_cc = k_end_cond then
                            f_cc <= (others => '0');
							f_sda <= '1';
							s_i2c <= s_idle;
                        end if;
                end case;
            end if;
        end if;
    end process;
end rtl;
\end{VHDLlisting}

For this implementation we know that we are going to send $32$ bits so we don't need a generic to specify the data width. As far as the other inputs of the block we have the system clock as well as the reset line. The two \ac{I2C} lines clock and data. And finally the data the send with a data valid flag and a busy signal. 

For the \ac{I2C} implementation we again start off the state \emph{s\_idle}. Once $32$ bits are provided to the core with the data valid flag we move to the \emph{s\_init\_frame}. In this state we set the timing between the deassertion of SDA and the falling edge of the clock. This timing is specified in the data sheet for the chip that you will be interfacing to. 

Next we start the data frame in \emph{s\_write\_data\_frame}. Where we set the current bit of \emph{f\_data} to the output on the falling edge. Then after each eight bit section we jump to \emph{s\_ack\_bit} which ensure data is being sent successfully to the \ac{I2C}-Slave.

Finally after sending all $32$ bits the done flag is asserted and we end the transaction. 
	
	
\subsection{Texas Instruments universal Parallel Port (uPP)}
	<TODO Subsection Texas Instruments universal Parallel Port (uPP) : PROOF READ>
	
A \ac{TI}-\ac{DSP} is an embedded processor that specializes in operations specific to signal processing. What that means is that the \ac{MAC} structure is used often in \ac{FIR} filter design, calculating \ac{FFT}s, and even some matrix multiplication calculations. The \ac{DSP} is equipped with with the ability to perform parallel \ac{MAC} operations. Depending on the specific processor you can expect eight parallel \ac{MAC} operations. 

The \ac{DSP} a very powerful processing platform with the added hardware to the ease of development in the C programming language. Because of these reasons is it popular to interface a \ac{DSP} to an \ac{FPGA}. With both chips on a \ac{PCB} you get the best of both worlds. 

To leverage having an \ac{FPGA} interface to a \ac{TI} - \ac{DSP} we need to communicate data between the two chips quickly. \ac{TI} has dedicated pins on their part specifically for this task. The \ac{TI} \ac{uPP} bus offers $16$ bit width data bus, with a start, clock, enable, and wait pins to control data flow. The \ac{TI} part also offers to send and receive data at \ac{DDR}. 

First to look at how the \ac{uPP} bus works we will look at the transmitter from the \ac{FPGA} to the \ac{TI}-\ac{DSP}. The \ac{VHDL} that defines the transmitter include to generics. The first generic is the \ac{DMA} size. For the \ac{FPGA} the \ac{DMA} size is the transfer length. The second generic is the \ac{uPP} clock rate ratio, which defines the clock speed for the transfers. 

Next we will define the \ac{IO} to the component. First the the clock and the reset line for the core. Next are the control lines for the \ac{uPP} bus, clock, start, enable, and wait. We also have the $16$ bit data bus. Finally we have the application signals that are used for the \ac{IO} between the other \ac{FPGA} logic and this core. 

After the signal declarations we start the architecture block with defining the outputs. We set the outputs first so that if you are looking at the block later you know what is being calculated. After we assign the full flag for the \ac{FIFO} we set the four \ac{uPP} control lines. 

We now move on to the process where we transmit data. We first reset some registers, then we define a state machine. The state machine defines two states, idle and transfer. While in idle we wait until the input \ac{FIFO} is not empty. If the input \ac{FIFO} isn't empty then we have data to transfer. Once we detect the empty flag is zero we read the data and move to the transfer state. 

Once in the transfer state we first assign the data out of the \ac{FIFO} when the \ac{FIFO} data valid is high. Next when the f\_upp\_clk\_strobe is high we are at a rising edge of the upp\_clk. We generate this strobe in a process later. For now we just assume the strobe is high for a clock cycle when the upp\_clk has a rising edge. 

If the clock strobe is high we then check the wait signal from the receiver. The wait signal is how the receiver stalls the transmitter. So if the wait signal is low we enable the \ac{uPP} transfer, read from the \ac{FIFO}, and increment the transfer count. Then if the transfer count is zero then we assert the \ac{uPP} start signal. If we are at the \ac{DMA} transfer size then we have completed the transfer then we go back to the idle state. 

However, if \ac{uPP} wait is asserted or if the clock strobe isn't high then we disable reading from the \ac{FIFO}. The only time we disable the \ac{uPP} transfer is when \ac{uPP} wait is asserted since the transfer stalls. If the clock strobe is not high that means we do not want to read from the \ac{FIFO} but we are still within a clock cycle of the \ac{uPP} clock. 

The next instantiation is the \ac{FIFO} itself with the port map. Which is then followed by a process that generates the \ac{uPP} clock, and the \ac{uPP} clock strobe. First we have a reset the clears the clock counter and holds the clock low. We leave it to the user logic to hold the transmitter in reset when not in use. 

When the core is not held in reset we first register the clock so that we can detect the rising edge. Next we count clock cycles so that we can toggle the \ac{uPP} clock based on the clock ratio specified by the instantiation. Finally we use the registering of the \ac{uPP} clock to detect the rising edge of the \ac{uPP} clock and assert and de-assert the strobe used in the first process accordingly.

\begin{VHDLlisting}[tabsize=2]
-- upp_tx.vhd

library ieee;
	use ieee.std_logic_1164.all;
	use ieee.numeric_std.all;
	
entity upp_tx is
	generic(g_dma_size      : integer := 4096;
			g_upp_clk_ratio : integer := 200)
	port(i_clk           : in    std_logic;
		 -- Used async | release sync
	     i_rst           : in    std_logic;
		 
		 -- uPP IO
		 o_upp_clk       :   out std_logic; 
		 o_upp_start     :   out std_logic;
		 o_upp_enable    :   out std_logic;
		 i_upp_wait      : in    std_logic;
		 
		 o_upp_data      :   out std_logic_vector(15 downto 0);
		 
		 -- App Interface
		 i_tx_data       : in    std_logic_vector(15 downto 0);
		 i_tx_data_dv    : in    std_logic	
		 o_tx_fifo_full  :   out std_logic	
	);
end entity upp_tx;

architecture rtl of upp_tx is
	constant k_upp_clk_ratio_half : integer := g_upp_clk_ratio/2;
	signal w_tx_fifo_full : std_logic;
	
	signal f_clk_count      : unsigned(log2(g_upp_clk_ratio)-1 downto 0);
	signal f_tx_word_count  : unsigned(log2(g_dma_size)-1 downto 0);
	signal f_upp_clk        : std_logic;
	signal ff_upp_clk       : std_logic;
	signal f_upp_clk_strobe : std_logic;
	signal f_upp_start      : std_logic;
	signal f_upp_enable     : std_logic;
	signal f_upp_data       : std_logic_vector(15 downto 0);
	                        
	signal f_fifo_rden      : std_logic;
	signal w_fifo_dout      : std_logic_vector(15 downto 0);
	signal w_fifo_rddv      : std_logic;
	signal w_fifo_empty     : std_logic;
	
	type sm_upptx is (s_idle, s_transfer, s_end);
	signal s_curr_state : sm_upptx := s_idle;

begin

	o_tx_fifo_full <= w_tx_fifo_full;
	o_upp_clk      <= f_upp_clk;
	o_upp_start    <= f_upp_start ;
	o_upp_enable   <= f_upp_enable;
	o_upp_data     <= f_upp_data;
	
	p_tx_data : process(i_clk)
	begin
		if rising_edge(i_clk) then
			if i_rst = '1' then
				s_curr_state <= s_idle;
				f_upp_start <= '0';
				f_upp_enable <= '0';
				f_tx_word_count <= (others => '0');
			else
				case s_curr_state is
					when s_idle => 
						f_upp_start <= '0';
						f_upp_enable <= '0';
						f_tx_word_count <= (others => '0');
						
						if w_fifo_empty = '0' then
							s_curr_state <= s_transfer;
							f_fifo_rden <= '1';
						end if;					
					when s_transfer => 
						if w_fifo_rddv = '1' then
							f_upp_data <= w_fifo_dout;
						end if;
						
						if f_upp_clk_strobe = '1' then
							-- Rising Edge of upp clk
							if i_upp_wait = '0' then
								f_upp_enable <= '1';
								f_fifo_rden <= '1';
								f_tx_word_count <= f_tx_word_count + 1;
							
								case f_tx_word_count is
									when 0 => 
										f_upp_start <= '1';
									when g_dma_size => 
										s_curr_state <= s_end;
									when others =>
										f_upp_start <= '0';
								end case;			
							else
								-- Transfer is stalled
								f_upp_enable <= '0';
								f_fifo_rden <= '0';
							end if;							
						else
							f_fifo_rden <= '0';
						end if;
					
					when s_end => 
				
				end case;
			end if;
		end if;
	end process;
	
	
	u_tx_fifo : entity work.tx_fifo
	port map(wrclk => i_clk, 
			 wren  => i_tx_data_dv,
			 din   => i_tx_data,
			 full  => w_tx_fifo_full,
	
			 rdclk => f_upp_clk,
			 rden  => f_fifo_rden,
			 dout  => w_fifo_dout,
			 rddv  => w_fifo_rddv,
			 empty => w_fifo_empty	
	);
	
	p_gen_upp_clk : process(i_clk)
	begin
		if rising_edge(i_clk) then
			if i_rst = '1' then
				f_clk_count <= (others => '0');
				f_upp_clk <= '0';
			else
				ff_upp_clk <= f_upp_clk;
			
				f_clk_count <= f_clk_count + 1;
				if f_clk_count = k_upp_clk_ratio_half then
					-- Toggle UPP Clk
					f_upp_clk <= not f_clk_count;
					f_clk_count <= (others => '0');
				end if;
				
				if ff_upp_clk = '0' and f_upp_clk = '1' then
					-- Generate upp clock strobe
					f_upp_clk_strobe <= '1';
				else
					f_upp_clk_strobe <= '0';
				end if;
				
			end if;
		end if;	
	end process;
\end{VHDLlisting}

Now that we have seen how the \ac{FPGA} transmits data we will transition to how the \ac{FPGA} received data from the \ac{TI}-\ac{DSP}. Once again in the entity declaration we have a clock and a reset. We next have the \ac{uPP} \ac{IO} lines. These lines are just like the transmitter's \ac{IO} lines in reverse direction. Finally we have the application logic lines. 

In the architecture declaration we have the internal signal definitions. These include the the internal registers for the data received by the core, the busy signal that is an output to the other \ac{FPGA} logic, and a \ac{DMA} count to ensure we have all the data we expect. We also have a signal that is a wire that is the full flag from the receive \ac{FIFO}.

We move past the begin keyword to the instantiated code. First we see the output declarations. As part of our coding style and guidelines we assign the outputs right after the begin to ensure other reviewers of our code can find the results for our core. Immediately after we assign some of the outputs we see an instantiation of a \ac{FIFO}. Here we need a dual clock \ac{FIFO} since the \ac{uPP} clock is different than the system clock. The dual clock \ac{FIFO} defines the reset of the outputs of the core. 

The port map of the \ac{FIFO} is separated into two section the write section and read section. The write section is controlled by a process which we will discuss later. The read section is of the \ac{FIFO} is controlled by logic external to this core. The read clock is the system clock and the read enable is an input to the core. The read data, read data valid, and empty signals are all outputs to inform the external user logic. 

We move to the receive data process. In this process we first have a synchronous reset. In the reset we clear the \ac{DMA} transfer count the receive data valid and \ac{uPP} busy signal. While not in reset we resister the \ac{uPP} start signal. Then we edge detect the \ac{uPP} start signal and if detect the rising edge of the start signal we assert the busy signal. Next we check the \ac{DMA} transfer count and if we are at the end of the transfer we de-assert the busy signal. 

The last if statement registers data of the \ac{uPP} data bus. If the transfer is enabled then we increment the \ac{DMA} count, register the data bus, and assert the write enable line for the \ac{FIFO}. If \ac{uPP} enable is low then we de-assert write enable. This process fills up the \ac{FIFO} and the external logic pulls data from the \ac{FIFO} but the \ac{FIFO} should be configured to be to be able to support the \ac{DMA} length. 

\begin{VHDLlisting}[tabsize=2]
-- upp_rx.vhd

library ieee;
	use ieee.std_logic_1164.all;
	use ieee.numeric_std.all;
	
entity upp_rx is
	generic(g_dma_size : integer := 4096)
	port(i_clk           : in    std_logic;
	     i_rst           : in    std_logic;
		 
		 -- uPP IO
		 i_upp_clk       : in    std_logic; 
		 i_upp_start     : in    std_logic;
		 i_upp_enable    : in    std_logic;
		 o_upp_wait      :   out std_logic;
		 
		 i_upp_data      : in    std_logic_vector(15 downto 0);
		 
		 -- App interface
		 o_upp_busy      :   out std_logic;
		 o_rx_data_empty :   out std_logic;
		 i_rx_data_rden  : in    std_logic;
		 o_rx_data_dv    :   out std_logic;
		 o_rx_data       :   out std_logic_vector(15 downto 0)	
	);
end entity upp_rx;


architecture rtl of upp_rx is
	
	signal f_rx_data_dv   : std_logic;
	signal f_rx_data      : std_logic_vector(15 downto 0);
	signal w_rx_fifo_full : std_logic;
	signal f_upp_busy     : std_logic;
	
	signal f_dma_count    : unsigned(log2(g_dma_size)-1 downto 0);
	
begin

	o_upp_wait <= w_rx_fifo_full;
	o_upp_busy <= f_upp_busy; 

	u_rx_fifo : entity work.rx_fifo
	port map(i_rst => i_rst,
			 wrclk => i_upp_clk, 
			 wren  => f_rx_fifo_wren,
			 din   => f_rx_fifo_data,
			 full  => w_rx_fifo_full,
	
			 rdclk => i_clk,
			 rden  => i_rx_data_rden,
			 dout  => o_rx_data,
			 rddv  => o_rx_data_dv,
			 empty => o_rx_data_empty	
	);

	p_rx_data : process(i_upp_clk, i_arst)
	begin
		if rising_edge(i_upp_clk) then
			if i_rst = '1' then
				f_dma_count <= (others => '0');
				f_rx_data_dv <= '0';
				f_upp_busy <= '0';
			else
				f_upp_start <= i_upp_start;
				if f_upp_start = '0' and i_upp_start = '1' then	
					-- Start DMA transfer
					f_upp_busy <= '1';
				end if;
				
				if f_dma_count = g_dma_size-1 then
					-- DMA Transfer Complete
					f_dma_count <= (others => '0');
					f_upp_busy <= '0';
				end if;
			
				if i_upp_enable = '1' then
					-- Load uPP data into rx fifo
					f_dma_count <= f_dma_count + 1;
					f_rx_fifo_data <= i_upp_data;
					f_rx_fifo_wren <= '1';
				else
					-- uPP Transfer Stalled.
					f_rx_fifo_wren <= '0';
				end if;
			end if;
		end if;
	end process;	
end rtl;
\end{VHDLlisting}

\section{Data Processing Projects}
	<TODO Section Data Processing Projects : PROOF READ>
	
We now transition to \ac{VHDL} examples that are processing data. The first example calculates the multiplication of two matrices. Matrix multiplication has applications in signal processing and machine learning. We next look at the knapsack problem. In this problem we are looking to maximize the reward which is applicable to \ac{FPGA} development itself. Finally we present some signal processing examples, a \ac{FIR} filter and an \ac{NCO}, both of which are used regularly in \ac{DSP} applications. 

\subsection{Strassen's Matrix Multiplication}
	<TODO Subsection Strassen's Matrix Multiplication : PROOF READ>
	
Matrix multiplication is a common operation is \ac{DSP}, machine learning, and graphics processing applications. Here we look at how to make matrix multiplication more efficient. Matrix multiplication in general is $\mathbf{O}(n^3)$. To make the algorithm more efficient in terms of \ac{FPGA} resource we will use the Strassen matrix multiplication. 

In essence the Strassen matrix multiplication reduces the number of multiplies required. Since the dedicated multiply resources are very advantageous we look to minimize the usage if we can. To multiply very large matrices we need to use the dedicated multiply blocks efficiently. The Strassen matrix multiply does this optimally for reducing input matrices to a $2 \times 2$.

We are going to calculate the matrix multiplication $\mathbf{A}\mathbf{B}=\mathbf{C}$. We assume that $\mathbf{A}$ and $\mathbf{B}$ have specific dimensions, namely, the number of rows and columns equate to a power of 2. If the matrices you want to multiply are not powers of $2$ you can zero pad each dimension. Each dimension is zero padded to $2^k$ where $k$ is an integer. If both $\mathbf{A}$ and $\mathbf{B}$ have dimensions that are $2^k$ then the matrix multiplication can be recursively reduced to a number of $2 \times 2$ matrices. 

To calculate a $2 \times 2$ matrix multiply we turn to the Strassen algorithm. The input to the Strassen algorithm are the two matrices $\mathbf{A}$ and $\mathbf{B}$. Each matrix is indexed into by a row and column;

\begin{equation}
\mathbf{A}=
\begin{bmatrix}
  a_{1,1} & a_{1,2} \\
  a_{2,1} & a_{2,2} \\
 \end{bmatrix},~~~~~~~~~~~~~
\mathbf{B}=
\begin{bmatrix}
  b_{1,1} & b_{1,2} \\
  b_{2,1} & b_{2,2} \\
 \end{bmatrix}.
\end{equation}

Then the Strassen algorithm defines some intermediate values as:

\begin{eqnarray}
\label{eq:capp}
P &=& (a_{1,1} + a_{2,2})(b_{1,1} + b_{2,2})\\
Q &=& (a_{2,1} + a_{2,2})b_{1,1}          \\
R &=& a_{1,1}(b_{1,2} - b_{2,2})\\
S &=& a_{2,2}(b_{2,1} - b_{1,1})\\
T &=& (a_{1,1} + a_{1,2})b_{2,2}  \\        
U &=& (a_{2,1} - a_{1,1})(b_{1,1} + b_{1,2})\\
V &=& (a_{1,2} - a_{2,2})(b_{2,1} + b_{2,2})
\label{eq:capv}
\end{eqnarray}

And finally with the intermediate values defined the elements of the result matrix, $\mathbf{C}$, are calculated,

\begin{eqnarray}
c_{1,1} &=& P + S - T + V\\
c_{1,2} &=& R + T\\
c_{2,1} &=& Q + S\\
c_{2,2} &=& P + R - Q + U
\label{eq:couts}
\end{eqnarray}

Now that the algorithm is defined we will spend the reset of this section discussing how we can implement the Strassen Matrix Multiplication algorithm in \ac{VHDL}. Before we start implementing the algorithm we first must think about what the expected data rate is going to be. Since the Strassen algorithm helps with reducing the number of multiplies required from eight to seven we won't see a huge savings in \ac{FPGA} resources with small matrix multiplies. Our target implementation is going to be handling very large matrices that will be parted down to many $2 \times 2$ matrix multiplication. For this implementation we expect to instantiate this block many times so that we can handle all the data as it is streaming in. 

So for this implementation we want to be able to handle the two input matrices in parallel and data can be valid every clock cycle. So after the latency of the core is over we will have resulting $\mathbf{C}$ matrices every clock cycle. This implementation will be able to handle the largest data rates possible. 

To get started with this implementation we start with an entity declaration,

\begin{VHDLlisting}[tabsize=2]
-- matrix_multiply_strassen.vhd

library ieee;
	use ieee.std_logic_1164.all;
	use ieee.numeric_std.all;
	
entity matrix_multiply_strassen is
	generic(g_bitwidth : integer)
	port(i_clk    : in    std_logic;
		 i_rst    : in    std_logic;
		 
		 i_dv     : in    std_logic;
		 i_a_11   : in    signed(g_bitwidth-1 downto 0);
		 i_a_12   : in    signed(g_bitwidth-1 downto 0);
		 i_a_21   : in    signed(g_bitwidth-1 downto 0);
		 i_a_22   : in    signed(g_bitwidth-1 downto 0);
		
		 i_b_11   : in    signed(g_bitwidth-1 downto 0);
		 i_b_12   : in    signed(g_bitwidth-1 downto 0);
		 i_b_21   : in    signed(g_bitwidth-1 downto 0);
		 i_b_22   : in    signed(g_bitwidth-1 downto 0);
		 
		 o_dv     :   out std_logic;
		 o_c_11   :   out signed(2*g_bitwidth-1 downto 0);
		 o_c_12   :   out signed(2*g_bitwidth-1 downto 0);
		 o_c_21   :   out signed(2*g_bitwidth-1 downto 0);
		 o_c_22   :   out signed(2*g_bitwidth-1 downto 0)	
	);
end entity matrix_multiply_strassen;
\end{VHDLlisting}

The entity declaration for \emph{matrix\_multiply\_strassen}has one generic which sets the bit width for the elements of the matrices. The port definitions start with the clock and reset, then the two matrix inputs, with four elements each along with a single data valid line for both matrices. This interface puts the responsibility of ensuring the data is ready and valid in a single clock cycle on the instantiating core. The last set of signals are the output, the four elements of the matrix $\mathbf{C}$ and a data valid line. 

Now we will move on the architecture block of the core. First we define a shift register for the data valid line. The shift register keeps track of the pipeline stage. If the data was valid in a clock cycle then the data calculation and the valid is registered. In this way the valid flag follows along with the data. 

\begin{VHDLlisting}[tabsize=2]
signal f_i_dv : std_logic_vector(5 downto 0);

-- Stage 1 : Register Inputs
signal f_a_11 : signed(g_bitwidth-1 downto 0);
signal f_a_12 : signed(g_bitwidth-1 downto 0);
signal f_a_21 : signed(g_bitwidth-1 downto 0);
signal f_a_22 : signed(g_bitwidth-1 downto 0);
signal f_b_11 : signed(g_bitwidth-1 downto 0);
signal f_b_12 : signed(g_bitwidth-1 downto 0);
signal f_b_21 : signed(g_bitwidth-1 downto 0);
signal f_b_22 : signed(g_bitwidth-1 downto 0);
\end{VHDLlisting}

We now look at the signals that are going to be used in each stage of calculation. In the signal declaration section we organize and comment the groups of signals so that as our \ac{VHDL} files get larger we can orgainize the file in a logical order which helps keeps the logic more manageable. 

The first stage of calculation is registering the inputs under the data valid line. It is a good habit to register the inputs and outputs of any block that you write. This way when you instantiate the core you won't have to worry about if another core registers it's outputs. If you are using other people's code that doesn't register outputs and if you don't register inputs then its likely that those paths will be the first to fail timing. 

If the code you are interfacing to does register the outputs and you register the inputs the synthesizer is reliable enough to remove one register. This practice is very helpful when you are trying to achieve the highest data rate possible. Once we have made this core fully pipelined then the only other way to increase the data rate is to increase the clock speed. To run this core at the highest possible clock speed we will be glad to have the extra registers in the code. 

The process that registers the inputs is shown here. 

\begin{VHDLlisting}[tabsize=2]
p_s1 : process(i_clk)
	begin
		if rising_edge(i_clk) then
			if i_rst = '1' then
				f_i_dv <= (others => '0');
			else
				f_i_dv <= f_i_dv(f_i_dv'high-1 downto 0) & i_dv;
				
				if i_dv = '1' then
					f_a_11  <= i_a_11;
				    f_a_12  <= i_a_12;
				    f_a_21  <= i_a_21;
				    f_a_22  <= i_a_22;
				    f_b_11  <= i_b_11;
				    f_b_12  <= i_b_12;
				    f_b_21  <= i_b_21;
				    f_b_22  <= i_b_22;
				end if;
			end if;
		end if;
	end process;
\end{VHDLlisting}

Under the rising edge clause of this process we first check the reset line. If the reset line is asserted we only need to clear the data valid shift register. Clearing these registers ensure no data is output from the core. Next we shift in the data valid input into our shift register. This signal is what will be used in the subsequent processes to know when data is valid. Finally we register all the inputs under the single data valid. 

\begin{VHDLlisting}[tabsize=2]
-- Stage 2 : Term Calculations
signal ff_p_t1 : signed(g_bitwidth-1 downto 0);
signal ff_p_t2 : signed(g_bitwidth-1 downto 0);
signal ff_q_t1 : signed(g_bitwidth-1 downto 0);
signal ff_q_t2 : signed(g_bitwidth-1 downto 0);
signal ff_r_t1 : signed(g_bitwidth-1 downto 0);
signal ff_r_t2 : signed(g_bitwidth-1 downto 0);
signal ff_s_t1 : signed(g_bitwidth-1 downto 0);
signal ff_s_t2 : signed(g_bitwidth-1 downto 0);
signal ff_t_t1 : signed(g_bitwidth-1 downto 0);
signal ff_t_t2 : signed(g_bitwidth-1 downto 0);
signal ff_u_t1 : signed(g_bitwidth-1 downto 0);
signal ff_u_t2 : signed(g_bitwidth-1 downto 0);
signal ff_v_t1 : signed(g_bitwidth-1 downto 0);
signal ff_v_t2 : signed(g_bitwidth-1 downto 0);
\end{VHDLlisting} 

In stage two we declare signals that are used in the addition or subtraction calculations in \eq{capp}-\eq{capv}. In the signal names we use the notation $t1$ or $t2$ to indicate the two operands of the multiplication in each equation. 

\begin{VHDLlisting}[tabsize=2]
p_s2 : process(i_clk)
	begin
		if rising_edge(i_clk) then
			if f_i_dv(0) = '1' then
				ff_p_t1 <= f_a_11 + f_a_22;
				ff_p_t2 <= f_b_11 + f_b_22;
				ff_q_t1 <= f_a_21 + f_a_22;
				ff_q_t2 <= f_b_11;
				ff_r_t1 <= f_a_11;
				ff_r_t2 <= f_b_12 - f_b_22;
				ff_s_t1 <= f_a_22;
				ff_s_t2 <= f_b_21 - f_b_11;
				ff_t_t1 <= f_a_11 + f_a_12;
				ff_t_t2 <= f_b_22;
				ff_u_t1 <= f_a_21 - f_a_11;
				ff_u_t2 <= f_b_11 + f_b_12;
				ff_v_t1 <= f_a_12 - f_a_22;
				ff_v_t2	<= f_b_21 + f_b_22;
			end if;
		end if;
	end process;
\end{VHDLlisting}

For each term we follow \eq{capp}-\eq{capv} and assign each term. To ensure all the data with the associated input matrices stay aligned we need to register intermediate values even if no calculation is done. For example, the second term of $Q$ is just $b_{1,1}$ so we register that value because we will need it in the next clock cycle and another input matrix may be in the pipeline next clock cycle overwriting our data. So we need to register it in this process.

\begin{VHDLlisting}[tabsize=2]
-- Stage 3 : P,Q,R,S,T,U,V Calculations
signal fff_p   : signed(2*g_bitwidth-1 downto 0);
signal fff_q   : signed(2*g_bitwidth-1 downto 0);
signal fff_r   : signed(2*g_bitwidth-1 downto 0);
signal fff_s   : signed(2*g_bitwidth-1 downto 0);
signal fff_t   : signed(2*g_bitwidth-1 downto 0);
signal fff_u   : signed(2*g_bitwidth-1 downto 0);
signal fff_v   : signed(2*g_bitwidth-1 downto 0);
\end{VHDLlisting}

The third stage calculates the intermediate variables in the Strassen algorithm. The size of the result of a multiply operation is bigger. Just like when we multiply $9 \times 9 = 81$ we need more bits to handle the range of the possible results. 

\begin{VHDLlisting}[tabsize=2]
p_s3 : process(i_clk)
begin
	if rising_edge(i_clk) then
		if f_i_dv(1) = '1' then
			fff_p <= ff_p_t1 * ff_p_t2;
			fff_q <= ff_q_t1 * ff_q_t2;
			fff_r <= ff_r_t1 * ff_r_t2;
			fff_s <= ff_s_t1 * ff_s_t2;
			fff_t <= ff_t_t1 * ff_t_t2;
			fff_u <= ff_u_t1 * ff_u_t2;
			fff_v <= ff_v_t1 * ff_v_t2;
		end if;
	end if;
end process;
\end{VHDLlisting}

The process that calculates the multiplication is shown here. Under the third pipeline stage data valid we calculate all the intermediate values $P$ - $V$. 

\begin{VHDLlisting}[tabsize=2]
-- Stage 4 : C term calculations
signal f4_c11_t1 : signed(2*g_bitwidth-1 downto 0);
signal f4_c11_t2 : signed(2*g_bitwidth-1 downto 0);
signal f4_c12_t1 : signed(2*g_bitwidth-1 downto 0);
signal f4_c21_t1 : signed(2*g_bitwidth-1 downto 0);
signal f4_c22_t1 : signed(2*g_bitwidth-1 downto 0);
signal f4_c22_t2 : signed(2*g_bitwidth-1 downto 0);
               
-- Stage 5 : Output calculations
signal f5_c11    : signed(2*g_bitwidth-1 downto 0);
signal f5_c12    : signed(2*g_bitwidth-1 downto 0);
signal f5_c21    : signed(2*g_bitwidth-1 downto 0);
signal f5_c22    : signed(2*g_bitwidth-1 downto 0);
\end{VHDLlisting}

The reset of the signal declarations have the same bit width as the multiplication result. We declare the two terms for the final output in stage four and the four element results are calculated in stage five. 

\begin{VHDLlisting}[tabsize]
p_s4 : process(i_clk)
begin
	if rising_edge(i_clk) then
		if f_i_dv(2) = '1' then
			f4_c11_t1 <= fff_p + fff_s;
		    f4_c11_t2 <= fff_t + fff_v;
		    f4_c12_t1 <= fff_r + fff_t;
		    f4_c21_t1 <= fff_q + fff_s;
		    f4_c22_t1 <= fff_p + fff_r;
		    f4_c22_t2 <= fff_q + fff_u;
		
		end if;
	end if;
end process;
\end{VHDLlisting}

In the calculation of stage four we calculate the terms in \eq{couts}. The additions are calculated under the data valid line for stage four and once complete they are ready for the final stage of calculation. 

\begin{VHDLlisting}[tabsize=2]
p_s5 : process(i_clk)
begin
	if rising_edge(i_clk) then
		if f_i_dv(3) = '1' then
			f5_c11 <= f4_c11_t1 - f4_c11_t2;
		    f5_c12 <= f4_c12_t1;
		    f5_c21 <= f4_c21_t1;
		    f5_c22 <= f4_c22_t1 - f4_c22_t2;
		end if;
	end if;
end process;
\end{VHDLlisting} 

The final stage, stage five, is calculated once again from \eq{couts} where for terms $c_{1,1}$ and $c_{2,2}$ a subtraction is needed but the other two elements are already final but we need to register them again to ensure data is aligned. 

\begin{VHDLlisting}[tabsize=2]
-- Assign Outputs
o_dv   <= f_i_dv(5);
o_c_11 <= f5_c11;
o_c_12 <= f5_c12;
o_c_21 <= f5_c21;
o_c_22 <= f5_c22;
\end{VHDLlisting}

After the final stage is calculated and registered we can use these registers as the output registers so now all we need to do is assign them as the output. At this point we need the fifth data valid in the shift register to denote that the output is valid. 

This shift register denotes the latency of the core. If we put in data at clock cycle zero. It will take six clock cycles to get valid data out. Note that index five in the data valid array is actually the sixth element since the array is zero indexed. 

\subsection{Digital Signal Processing}
	<TODO Subsection Digital Signal Processing : PROOF READ>
	
A common application for \ac{FPGA}s is \ac{DSP}. In this section we discuss the implementation of common \ac{DSP} functions. Here will discuss the \ac{FIR} filter and an \ac{NCO}. Which are both vital in a \ac{DSP} application. 

\subsubsection{Finite Impulse Response Filter}
	<TODO Subsubsection  Finite Impulse Response Filter : PROOF READ>

There are many needs for \ac{FIR} filters in \ac{DSP}. Here we will present one type of \ac{FIR} filter that is quite useful. The type of filter we will be implementing in this section is the \emph{half-band \ac{FIR} \ac{LPF}}. Breaking down the name we first know what a \ac{FIR} filter is and does which depending on the impulse response attenuates some frequencies while keeping other frequencies of interest. 

The \emph{half-band} aspect of this filter refers to the ability of the filter to reduce the bandwidth of the filtered signal by half. The reduction of bandwidth of the filtered signal reduces the data rate of the system after the filter. You may be asking why would you have a higher data-rate coming into the \ac{FPGA} from the \ac{ADC} than you would need. The data-rate coming into the \ac{FPGA} is dependent on the \ac{IF} used in the \ac{RF} front-end and the \ac{ADC} sample rate. 

The \ac{IF} used in the front-end is usually standard and your selection of the \ac{IF} is fixed. What you do have control over is the \ac{ADC} sampling rate. The main input parameter used in deciding your sampling rate is the bandwidth of the signal you are interested in, since your sampling rate needs to be at least twice the bandwidth of the signal you are interested in. 

Once the \ac{ADC} sampling rate is chosen we can start interfacing the \ac{ADC} to the \ac{FPGA}. The data lines are routed to the \ac{FPGA} along with a clock if the \ac{ADC} is has a source-synchronous interface. In any case the data is registered on the \ac{FPGA} and now we have another design parameter to decide. What processing speed are we going to use. In general we can slow the clock down and process multiple samples in parallel or speed the clock up and have more clocks per sample. In either case we need to consider the algorithm in which we are implementing. 

With all that said we will assume moving forward all of those parameters have been worked out. What happens quite frequently is that the bandwidth of the received signal is twice as large as we are interested in so we are going to decimate by two. To decimate by two we need to filter out the aliases to avoid sampling interference. This filter is called an anti-alias filter. So the filter we are going to design here is going to be out anti-alias filter. The data is then sent out of the core where the data valid line is dropped every other sample to perform the decimation by two. 

\begin{VHDLlisting}[tabsize=2]
-- halfbandLPF.vhd

library ieee;
	use ieee.std_logic_1164.all;
	use ieee.numeric_std.all;
	
entity halfbandLPF is 
	generic(g_bitwidth : integer := 16;
	        g_ncoeffs  : integer := 35);
	port(
			i_clk      : in    std_logic;
			i_reset    : in    std_logic;
			i_data     : in    std_logic_vector(g_bitwidth-1 downto 0);
			i_dv       : in    std_logic;
			o_data     :   out std_logic_vector(2*g_bitwidth-1 downto 0);
			o_dv       :   out std_logic	
	);
end entity halfbandLPF;

architecture rtl of halfbandLPF is 
	type T_Sbw is array (natural range<>) of signed(g_bitwidth-1 downto 0);
	type T_S2bw is array (natural range<>) of signed(2*g_bitwidth-1 downto 0);
	signal f_x_mix    : T_Sbw(0 to g_ncoeffs-1) := (others => (others => '0'));
	signal f_dv_shift : std_logic_vector(0 to g_ncoeffs-1) := (others => '0');
	
	-- Length of this vector is Ceil((g_ncoeffs-1)/4)+1
	-- These coefficients need to be loaded externally
	signal f_h_uni      : T_Sbw(0 to 9) := (x"0026",x"FFA6",x"00BD",x"FEA0",x"0261",x"FC05",x"06BC",x"F350",x"2870",x"4000");

	signal ff_mix_add   : T_Sbw(0 to 9) := (others => (others => '0'));
	signal fff_add_mult : T_S2bw(0 to 9) := (others => (others => '0'));
	
	signal f4_sum_s1    : T_S2bw(0 to 4) := (others => (others => '0'));
	signal f5_sum_s2    : T_S2bw(0 to 2) := (others => (others => '0'));
	signal f6_sum_s3    : T_S2bw(0 to 1) := (others => (others => '0'));
	signal f7_sum_s4    : signed(2*g_bitwidth-1 downto 0) := (others => '0');
begin
	o_data <= std_logic_vector(f7_sum_s4);
	o_dv <= f_dv_shift(6);

	p_calc : process(i_clk) is
	begin
		if rising_edge(i_clk) then
			if i_reset = '1' then
				f_dv_shift <= (others => '0');
			elsif i_dv = '1' then
				-- Shift Register Data
				f_x_mix(34) <= signed(i_data);
				f_x_mix(0 to g_ncoeffs-2) <= f_x_mix(1 to g_ncoeffs-1);
				
				-- Shift Register Data Valid
				f_dv_shift(0) <= i_dv;
				f_dv_shift(1 to g_ncoeffs-1) <= f_dv_shift(0 to g_ncoeffs-2);
				
				-- Take advantage of symmetric filter. 
				ff_mix_add(0) <= f_x_mix(0) + f_x_mix(34);
				ff_mix_add(1) <= f_x_mix(2) + f_x_mix(32);
				ff_mix_add(2) <= f_x_mix(4) + f_x_mix(30);
				ff_mix_add(3) <= f_x_mix(6) + f_x_mix(28);
				ff_mix_add(4) <= f_x_mix(8) + f_x_mix(26);
				ff_mix_add(5) <= f_x_mix(10) + f_x_mix(24);
				ff_mix_add(6) <= f_x_mix(12) + f_x_mix(22);
				ff_mix_add(7) <= f_x_mix(14) + f_x_mix(20);
				ff_mix_add(8) <= f_x_mix(16) + f_x_mix(18);
				ff_mix_add(9) <= f_x_mix(17);
				
				-- Multiply Impulse response. 
				for i in 0 to 9 loop
					fff_add_mult(i) <= ff_mix_add(i) * f_h_uni(i);
				end loop;				
				
				-- Start of the adder tree.
				for i in 0 to 4 loop
					f4_sum_s1(i) <= fff_add_mult(2*i) + fff_add_mult(2*i+1);
				end loop;
				
				f5_sum_s2(2) <= f4_sum_s1(4);
				for i in 0 to 1 loop
					f5_sum_s2(i) <= f4_sum_s1(2*i) + f4_sum_s1(2*i+1);
				end loop;
				
				f6_sum_s3(0) <= f5_sum_s2(0) + f5_sum_s2(1);
				f6_sum_s3(1) <= f5_sum_s2(2);
								
				-- Result of adder tree.
				f7_sum_s4 <= f6_sum_s3(0) + f6_sum_s3(1);
			end if;			
		end if;
	end process;
end rtl;
\end{VHDLlisting}
	
In the entity declaration we have two generics defined. The first the is bit-width of the input samples. The second is the number of coefficients or taps in the \ac{FIR} filter. The port declarations have the clock, reset, input, and output data lines. We notice that the output is twice the bit width as the input since the \ac{FIR} taps are the same bit width as the input data and we multiply the data by the taps getting twice the bit-width in growth. 

In the architecture block before the \emph{begin} keyword we define two types. Each of which is an array one for the input data width and the other twice the input data width. The first signal that is declared is a shift register for the input data and a data valid shift register is defined after. 

Next we have signal that is initialized to a set of 10 coefficients. We first develop the \ac{FIR} filter with the taps already in place so that we know there are not any issues we loading the coefficients externally. Loading the coefficients externally would make this \ac{FIR} filter implementation the most flexible and very easy to reuse the the data-rates were to change in future projects. Lastly defined are signals for five stages of an adder tree.

After the \emph{begin} keyword we have the outputs being assigned. Next we move to a calculation process that has a synchronous reset. Then we have the shift register for the data. Then another shift register for the data valid. This implementation assumes that the input is directly from an \ac{ADC} so once the data valid is asserted the input is always valid. 

Next we take advantage of the symmetric filter taps used in this \ac{FIR} filter to reduce the number of multiples. To do this we need to add the data at various indices together. Then in the next stage the impulse response is multiplied by the summation in the previous step. After which the ten values are summed in an adder tree where the number of adders is $log_2$ of the number of elements left to add. 
	
\subsubsection{Numerically Controlled Oscillator}
	<TODO Subsubsection  Numerically Controlled Oscillator : PROOF READ>
	
An \ac{NCO} is used to generate \emph{sine} and \emph{cosine} waves on an \ac{FPGA}. The generation of sinusoidal waves is beneficial for digitally mixing signals. In this section we discuss how to make an \ac{NCO} in two parts. First a phase accumulator is used to generate an address which is used to index in to a \ac{BRAM} which is assigned as the output. 

\begin{VHDLlisting}[tabsize=2]
-- nco.vhd

library ieee;
	ieee.std_logic_1164.all;
	ieee.numeric_std.all;
	
entity nco is
generic(g_phinc_width : integer;
	    g_bitwidth    : integer;);
port(i_clk        : in    std_logic;
	 i_rst        : in    std_logic;
	 i_enable     : in    std_logic;
	 i_phz_init   : in    unsigned(g_phinc_width-1 downto 0);
	 i_phz_dv     : in    std_logic;
	 i_phinc_init : in    unsigned(g_phinc_width-1 downto 0);
	 i_phinc_dv   : in    std_logic;
	 o_nco_data   :   out unsigned(g_bitwidth-1 downto 0);
	 o_nco_dv     :   out std_logic
);
end 

architecture rtl of nco is
	type t_us_g is array (natural range <>) of unsigned(g_bitwidth-1 downto 0);
	signal f_bram       : t_us_g(0 to 2**g_phinc_width-1);
	signal f_phz_accum  : unsigned(g_phinc_width-1 downto 0);
	signal f_phinc      : unsigned(g_phinc_width-1 downto 0);
	signal f_nco_data   : unsigned(g_bitwidth-1 downto 0);
	signal f_nco_dv     : std_logic := '0';
	signal ff_nco_dv    : std_logic := '0';
begin

	o_nco_data <= f_nco_data;
	o_nco_dv   <= ff_nco_dv;
	
	p_init : process(i_clk)
	begin
		if rising_edge(i_clk) then
			if i_rst = '1' then
				f_phz_accum <= (others => '0');
				f_nco_dv <= '0';
			else
				if i_phz_dv = '1' then
					f_phz_accum <= i_phz_init;
				end if;
				
				if i_phinc_dv = '1' then
					f_phinc <= i_phinc_init;
				end if;
				
				f_nco_dv <= i_enable;
				ff_nco_dv <= f_nco_dv;
				f_nco_data <= f_bram(to_integer(f_phz_accum));
				if i_enable = '1' then
					f_phz_accum <= f_phz_accum + f_phinc;
				end if;				
			end if;
		end if;	
	end process;
end rtl;
\end{VHDLlisting}

The entity declaration starts with the generic declarations. We have a generic that determines the phase increment width, denoted \emph{phinc\_width}. We also define a generic that determines the bit width of the \ac{NCO} output. The difference here is that our frequency resolution in our \ac{NCO} is independent of our dynamic range of our sinusoidal wave. 

We move on to the port definitions of the core. We, of course, have the clock and reset. Next we have an enable line that activates the core in producing outputs. Then two data lines with corresponding valid flags for phase initial value and phase increment value, each of which are discussed later on. Finally the output of the \ac{NCO} with a valid line. 

In the architecture block we first define signals for internal use of the core. The core uses a \ac{BRAM} to store the \ac{LUT} for the sinusoid, for which we need a type definition. Next we have the phase accumulator. Then we have the registered phase increment. Finally registers for the outputs are defined.

After the \emph{begin} keyword, we first assign the outputs that are calculated later in the core. Next, we have the only process in the core. The process is a synchronous process that resets the phase accumulation register and the output data valid register. If not in reset then the output of the \ac{NCO} is calculated. 

While not in reset we perform three major tasks. First of which is registering the initial phase input. The initial phase allows the entity that instantiates the \ac{NCO} to determine the starting phase of the sinusoid. The second major task is the registration of the \emph{phase increment} once registered, the phase increment is used in the third major task.

The third major task has two parts. While the enable is high the phase accumulator is incremented by the phase increment. The phase increment is used as an address into a \ac{BRAM}. The \ac{BRAM} is initialized with a sinusoid that is sampled by the phase accumulator. As the phase accumulator rolls over in unsigned value the sinusoid \ac{BRAM} cycles over the next period of the wave being generated. The sinusoid is an output of the core. 

\section{Security in Hardware}
	<TODO Section Security in Hardware : PROOF READ>

There is a fundamental difference in programming a processor and designing hardware, as we have seen in this chapter. Consider the differences, when programming a processor there is an atomic instruction that is executed with references to register addresses. However, in an \ac{FPGA}, data is registered with no way of accessing the data by address and no way of changing the \ac{FPGA} configuration without erasing the data. 

Due to the trusted nature of hardware, which an \ac{FPGA} is configurable hardware, \ac{FPGA}s are finding a place as a front-line malicious packet detection. If a data packet comes over the network that is detected as malicious then appropriate action can be taken. 

\chapter{Communication System}

\section{Wired Communications}
\subsection{PCI}
\subsection{USB}

\section{Wireless Communications}
\subsection{Binary Phase Shift Keying (BPSK)}
\subsection{Quadrature Amplitude Modulation (QAM)}
\subsection{Gaussian Minimum Shift Keying (GMSK)}

\subsection{Multiple-Carrier Modulation Schemes}
\subsubsection{Orthogonal Frequency Division Multiplexing (OFDM)}
\subsubsection{Single-Carrier Frequency Division Multiple-Access (SC-FDMA)}

\subsection{Forward Error Correction}
\subsubsection{Basic FEC}
\subsubsection{Reed-Solomon}
\subsubsection{Convolutional Codes}

\chapter{Moving Object Tracking in Real-Time Video}
	<TODO Chapter Moving Object Tracking in Real-Time Video : PROOF READ>

In this Chapter we discuss the basic image processing algorithms used in real-time video processing applications. The algorithms covered here focus on altering the captured video to highlight or enhance important aspects of the video to alert a user. An example application could include security cameras where motion detection alerts the security guard, or tracking suspicious vehicles outside a guarded area. A more fun application is tracking motion on a person while video chatting so the subject is always in frame. 

	
\section{Image Processing Algorithms}
	<TODO Section Image Processing Algorithms : PROOF READ>

Basic image processing algorithms are used regularly in all image processing applications. After a general knowledge of the basic algorithms is understood they can be pieced together to get a working system that is tailored to the application in mind. We start with matrix multiplication, which provides linear image affects like rotation and skew. We then move on to the \ac{FFT} which can be used for image registration. We then add to the functionality to the \ac{FFT} with wavelets. We finish up with convolution.

	
\subsection{Matrix Multiplication}
	<TODO Subsection Matrix Multiplication : PROOF READ>

Matrix multiplication is very useful in image processing since scaling, skew, and highlighting can be accomplished efficiently with spatial transformations. When applying a spatial transformation to an image we first break down the image into many smaller images. 

When we consider a point in the image we narrow our focus to the surrounding pixels. For example if we consider pixel $(x\times y)$ we limit our calculations to the range in $x$ to be $x-1$ to $x+1$ and the $y$ range is $y-1$ to $y+1$. In this method we are able to reduce our calculations to a $3 \times 3$ matrix multiply. 

The processes of applying a spatial transformation to an image would then consist of calculating a $3 \times 3$ matrix multiply at each pixel. Clearly if the image is large there are a lot of calculations required. However, we can perform the calculations in parallel since all the matrix multiplies are independent.

To calculate the transformation for the entire image consider the following Python code,

\begin{lstlisting}[language=Python]
for x in range(1,N-1):
  for y in range(1,N-1):
	timage = T*image(x-1:x+1,y-1:y+1)
	trans_image(x-1:x+1,y-1:y+1) += timage.
\end{lstlisting}

Where $T$ is the $3\times 3$ matrix that represents the transformation, \emph{image} is the input image to be transformed, and \emph{trans\_image} is the resulting transformed image. In this code there are a couple of things to notice. As defined we consider the pixel of interest and the eight surrounding pixels. All the edges are special cases. Depending on the transform we will zero-pad a row and column on the left and right and top and bottom of the image. Or we may duplicate that rows and columns to ensure the transformed image is the same dimensions. 
 
The second thing to notice about this is that in the transformed image a pixel will be the summation of nine matrix multiply results. So depending on the transformation we could have up to nine pixels contributing to a single transformed pixel. 

Next we need to define $T$. We discussed $T$ represents the transformation but what does it look like. First we will consider the identity transform. If we apply this transform we get the exact same image back. The identity transform looks like this,

\begin{equation}
\begin{bmatrix}
0 & 0 & 0\\
0 & 1 & 0\\
0 & 0 & 0\\
\end{bmatrix}.
\end{equation}
	
In the case of the identity transformation each pixel is multiplied by one and the surrounding pixels are multiplied by zero.

Next we consider a spatial lowpass filter. For the lowpass filter $T$ is,

\begin{equation}
\frac{1}{9}\begin{bmatrix}
1 & 1 & 1\\
1 & 1 & 1\\
1 & 1 & 1\\
\end{bmatrix}.
\end{equation}

For the lowpass filter we weight each pixel by $\frac{1}{9}$ then we sum the nine weighted pixels. This results in each pixel being the average of the surrounding pixels. The lowpass filter is also known as a smoothing filter. If we have sharp edges on our image we can smooth them out with a lowpass filter. 

The last spatial transform we will present is the highpass filter. The highpass filter can be used to detect edges since edges will have larger differences in color intensity. The larger differences are not attenuated by the highpass filter however the rest of the slowly varying color changes will be filtered to be the same.

The highpass filter transform matrix is,

\begin{equation}
\frac{1}{9}\begin{bmatrix}
0 & -1 & 0\\
-1 & 4 & -1\\
0 & -1 & 0\\
\end{bmatrix}.
\end{equation}

\cite{image processing book}	
	
\subsection{Fast Fourier Transform}
	<TODO Subsection Fast Fourier Transform : PROOF READ>
	
In the previous section we saw two spatial filter examples. In this section we transition to the frequency domain of an image. For image processing the two dimensional \ac{FFT} is needed. To transform the image into the two dimensional frequency domain we use,

\begin{equation}
F\left(k,l\right)=\sum_{i=0}^{N-1}\sum_{j=0}^{N-1}f\left(i,j\right)e^{-i2\pi\left(\frac{ki}{N}+\frac{lj}{N}\right)}
\end{equation}

Once in the frequency domain we can also perform a lowpass filter. Usually the advantage of performing a lowpass filter in the frequency domain is complexity since the \ac{FFT} is more efficient that matrix multiplication. But also we may need to do other processing in the frequency domain anyway. 

To implement an ideal lowpass filter in the frequency domain we first need the transfer function of the filter. The transfer function $H(k,l)$ is,

\begin{equation}
H(k,l) = 
\begin{cases}
1 & \text{if $D(k,l) \leq D_0$}\\
0 & \text{if $D(k,l) > D_0$}
\end{cases}
\end{equation}
\noindent
where $D(k,l)$ is the distance pixel $k,l$ is from the origin or center of the image. In the two dimensional \ac{FFT} result the low frequency region is the center and expanding out increases frequency. If we define a circle with radius $D_0$ and centered at the origin then every frequency inside the circle is not attenuated in the ideal filter and every frequency outside the circle is removed. 

We have now seen two ways to implement a lowpass filter. The first method was in the spatial domain and the other the in frequency domain. The advantage to the frequency domain implementation is that once the \ac{FFT} is complete a simple element-wise multiplication is needed. Then an \ac{IFFT} back to the spatial domain is needed. 

If our only goal is to smooth the image then the spatial domain implementation is preferred. However, rarely do we just want to smooth an image with no further calculation. In typical applications the smoothing filter is applied as an image conditioning step then further analysis is ran. The further analysis is often easier in the frequency domain.

An example of this is the correlation and convolution operations. There are slight differences between these two operations one of the differences being that convolution includes a complex conjugate but since we are operating on real images that difference is removed. The other difference is that convolution mirrors one of the images about the origin. So correlation determines how similar two images are and convolution determine how similar a mirrored image is with another image. 

If we have two images $f(x,y)$ and $h(x,y)$ we can use correlation to determine how similar $f$ and $h$ are. In the spatial domain we would need to perform the following calculation,

\begin{equation}
C_{f,h}(x,y) = \frac{1}{MN}\sum_{m=0}^{M}\sum_{n=0}^{N}f^*(m,n)h(x+m,y+n).
\end{equation}
\noindent

However since we are in the frequency domain we are able to calculate $C_{f,h}(x,y)$ with less computational complexity. To do this we first need to remember that convolution in the spatial domain is multiplication in the frequency domain. Next, since the correlation that we are interested in is very similar to convolution it can be shown that correlation in the frequency domain is equivalent to $F^*(k,l)H(k,l)$. In the frequency domain we multiply the images and take the \ac{IFFT} to get back to the spatial domain. 

\subsection{Wavelets}
	<TODO Subsection Wavelets : PROOF READ>

Wavelets build on the Fourier analysis by introducing a \emph{mother function}. The mother function is denoted by $\Psi$. The mother function can be selected based on the application but here we will look at the simplest mother function the \emph{haar} function.

The continuous time general form of the wavelet is,
\begin{equation}
X(a,b) = \frac{1}{\sqrt{a}}\int_{-\infty}^{\infty}\Psi\left(\frac{t-b}{a}\right) x(t)dt.
\end{equation}
\noindent
For our image processing application we are interested in the discrete wavelet transform, which is defined as,

\begin{equation}
Y[n,m] = \frac{1}{\sqrt{c_0^n}}\sum_{k=0}^{K-1}y[k]\Psi\left[\left(\frac{k}{c_0^n}-m\right)T\right].
\end{equation}
\noindent
where $c_0^n$ is the discretized scaling factor and $\Phi$ is sampled in the frequency domain. The definition in the frequency domain allow the ability to benefit from the wavelet analysis with minimal complexity. 

For signal and image processing the wavelet is a great tool for de-noising signals and de-blurring images. The wavelet transform is used for to calculation a deconvolution. Wavelet based deconvolution algorithms can be iterative which leads to data rate considerations. For more information on wavelet deconvolution see \cite{Emerging applications of wavelets: A review}.

\section{Kalman Filters}
	<TODO Section Kalman Filters: PROOF READ>

Kalman filter is estimates and tracks a \emph{true state}. The true state is corrupted by process and observation noise. The noise in the system corrupts an observation that we will use a Kalman filter to reduce the noise corruption. 

The perfect image that we hope to estimate is denoted $\mathbf{x}_k$, with $k$ denoting the frame number that we wish to estimate. As we iterate through images, say in a video, we update our estimate of the true state, denoted $\hat{\mathbf{x}}_k$.

The relation between the true state and the observation is modeled as,
\begin{equation}
\mathbf{z}_k = \mathbf{H}_k\mathbf{x}_{k} + \mathbf{v}_k.
\end{equation}
\noindent 
Where $\mathbf{z}_k$ is the observation, $\mathbf{x}_k$ is the true state and $\mathbf{H}_k$ transforms $\mathbf{x}_k$ into the observation state and $\mathbf{v}_k$ is observation noise. 

Next we need to understand how the underlying model for the the true state is dependent on the previous frames. The true state at $k$ depends on the state at $k-1$ and a transition model. The true state is updated as,	
\begin{equation}
\mathbf{x}_k = \mathbf{F}_k\mathbf{x}_{k-1} + \mathbf{B}_k\mathbf{u}_{k} + \mathbf{w}_k.
\end{equation}
\noindent
Where $\mathbf{F}_k$ is the state transition model. This models how previous images in the video are transformed to the new image. For example if the background is not changing the transition model is the identity for those regions. If there are rotations or skew introduced from the previous image then $\mathbf{F}_k$ also models these changes.

% k is the time index
% F is the state transition model
% H is the observation model
% Q is the covariance of the process noise
% R is the covariance of the observation noise
% B is the control-input model
% u control vector
% w process noise ~ N(0,Qk)
% observation zk is

Next $\mathbf{B}_k$ and $\mathbf{u}_{k}$ are the control model and control vector respectively. These terms model the input to the system that is not dependent on the previous frame. $\mathbf{B}_k\mathbf{u}_{k}$ would model how and where in the frame a new object is placed for example. Finally $\mathbf{w}_k$ is the process noise in the system.

The Kalman filter consists of two phases. The first phase is the \emph{predict} phase and the second is the \emph{update} phase. We will start with the predict phase. In the predict step we have not fully considered all the information at time $k$ these estimates are for $k-1$ time increments. To denote this explicitly we will use the notation that denotes $\hat{\mathbf{x}}_{n|m}$ as the estimate of $x$ at time $n$ with observations at time $m\leq n$.

Our prediction phase with our new notation is then,
\begin{equation}
\hat{\mathbf{x}}_{k|k-1} = \mathbf{F}_k\hat{\mathbf{x}}_{k-1|k-1} + \mathbf{B}_k\mathbf{u}_k,
\end{equation}

\begin{equation}
\mathbf{P}_{k|k-1} = \mathbf{F}_k\mathbf{P}_{k-1|k-1}\mathbf{F}_k^T + \mathbf{Q}_k.
\end{equation}
\noindent
Where $\mathbf{P}_{k|k-1}$ is the error in the covariance matrix for the true state. 

The update phase is starts off with a measure of the residual error,

\begin{equation}
\tilde{\mathbf{y}}_k = \mathbf{z}_k - \mathbf{H}_{k}\tilde{\mathbf{x}}_{k|k-1}.
\end{equation}
\noindent
The residual error is the difference in our observation and the true state transformed to the observation state. 

Next we can calculate the residual covariance,
\begin{equation}
\mathbf{S}_k = \mathbf{R}_k + \mathbf{H}_{k}\mathbf{P}_{k|k-1}\mathbf{H}_{k}^T,
\end{equation}
\noindent
where $\mathbf{R}_k$ is the covariance of the observation noise. Which we use in the calculation of the optimal Kalman filter gain,
\begin{equation}
\mathbf{K}_k = \mathbf{P}_{k|k-1}\mathbf{H}_{k}^T\mathbf{S}_k^{-1}.
\end{equation}
\noindent

We then can update the state estimate as,
\begin{equation}
\tilde{\mathbf{x}}_{k|k} = \tilde{\mathbf{x}}_{k|k-1} +  \mathbf{K}_k\tilde{\mathbf{y}}_{k}.
\end{equation}
\noindent
We now have enough information to update our estimate of the covariance to the state,
\begin{equation}
\mathbf{P}_{k|k} = \left(\mathbf{I} - \mathbf{K}_k\mathbf{H}_{k}\right) \mathbf{P}_{k|k-1}\left(\mathbf{I} - \mathbf{K}_k\mathbf{H}_{k}\right)^T + \mathbf{K}_k\mathbf{R}_k\mathbf{K}_k^T.
\end{equation}
\noindent
And finally we have the post-fit residual error,
\begin{equation}
\tilde{\mathbf{y}}_{k|k} = \tilde{\mathbf{z}}_{k} -  \mathbf{H}_k\tilde{\mathbf{x}}_{k|k},
\end{equation}
\noindent
which is used in the next iteration when $k$ is incremented to $k+1$. 

Clearly if the state $\mathbf{x}$ is large the calculations in the Kalman filter can be computationally intractable. Some dimensionality reduction techniques can be applied to reduce the problem down to a manageable size. 










\chapter{Machine Learning Algorithms}
	<TODO Chapter Machine Learning Algorithms : PROOF READ>
	
This chapter is an overview of common machine learning algorithms. First the algorithm itself is presented but perhaps more importantly the scenarios in which to use the algorithm is illustrated with an example. In this way you will add tools to your machine learning tool box but also know when to use them. 

\section{K-Nearest Neighbors Algorithm}
	<TODO Section K-Nearest Neighbors Algorithms : PROOF READ>
	
The \ac{KNN} algorithm is one of the simplest algorithms that is used in machine learning. The \ac{KNN} is used to group similar things together. The most popular application is the recommended products feature on some websites. Where you have selected to view an item, the website will take that item you viewed as an input to the \ac{KNN} algorithm, find the five or so items that are similar and present them to the user.

The \ac{KNN} algorithm uses a metric to determine how \emph{similar} another item is to it. The usefulness of the product recommendation engine is solely based on the distance metric used for the products. This can be a very complex decision since the relationship between items can be different for different people. 

The use of data mining or data gathering from users is a key to operating a good product recommendation algorithm. If you collect data from the users that if they select a certain item then they end up buying another this link can be established in your \ac{KNN} algorithm for a few different reasons. Capturing all the reasons will improve the results and usability.

To understand how the \ac{KNN} algorithm works we will first define our database of products. The database is denoted, $\mathbf{D}$ where each row describes a product. For each product there are a number of columns that list the attributes of the item. The number of attributes is up to the developer that can help in differentiating the products. For a particular product we can query the database to get all the attributes associated with the product, $\mathbf{D}_p = \mathbf{D}[p,:]$ where $1 \leq v \leq N_P$ and $N_P$ is the total number of products available.  

Next a user can request information about a particular product which is in say row $u$. After we know the user is interested in $\mathbf{D}_u$ then we can start our search for similar products. To search for similar products we first need to calculate the \emph{distance} between $\mathbf{D}_u$ and $\mathbf{D}_v$ where $1 \leq v \leq N_P$. 

We calculate the distance between the $\mathbf{D}_u$ and $\mathbf{D}_v$ which we denote, $d_{u,v} = \norm{\mathbf{D}_u-\mathbf{D}_v}$. Next we find the lowest $k$ distance measures $d_{u,v}$ and return the products to the user. 

A quick note on the algorithm's complexity. If we have a large number of products, $N_P$, we will need to search over the entire catalog of products. Of course we want to return the recommended products as quickly as possible to the user to provide a great user experience. We are able to pre-calculate the distances between the products and store the results. We can reduce the amount of calculation required while we are on the user's clock. 
	
\section{Linear Regression}
	<TODO Section Linear Regression : PROOF READ>
	
\index{Linear Regression}, in it's simplest form, models the relationship between an independent variable and a dependent variable. Once the relationship is characterized then any future scenario can be analyzed. The dependent variable can be predicted based on the independent variable for the new scenario. 

You may have already performed some Linear Regression and not realized it. If in high school or undergraduate studies you may have had a paper to write or a book to read. You can time yourself to determine how long it takes you to write or read one page. Then since you know how many total pages you have to read or write you can determine how long it will take you total to finish writing or reading your paper or book respectively. 

In this scenario the independent variable is the number of pages you need to write or read. The amount of time you spend writing or reading a page is the \emph{slope} or rate at which you accomplish the work. Then the simple multiplication of the number of pages times the rate results in the total time needed to finish the assignment. 

To formally map this to a mathematical model we first consider the linear equation we learned in school,

\begin{equation}
y = mx+b.
\label{eq:svlinmodel}
\end{equation}
\noindent
The independent variable is $x$ the total number of pages, $m$ is the rate or the amount of time per page, $y$ is the dependent variable or the total time to accomplish the task. We haven't discussed $b$ which could model a constant setup time to start writing or reading. In this scenario $b=0$ is probably a good assumption. 

Lets look at the book reading example a little closer. Lets say we are reading a fictional novel and we determine $m$. We can figure out how long it will take the read the book. Then we decide to read another fictional novel where the pages are about the same size. There is not really a reason to determine what $m$ is again we can just use the previous value of $m$. 

However, can we use the same $m$ for reading and absorbing the material from a dense mathematics book. No, in this case or $m$ value would be smaller. We would need to take more time on each page since not only are the pages longer but also we probably can't just read a page once and move on.

So a word of caution here is that if we \emph{train} under some circumstance to obtain a value for $m$. In this example, the \emph{training} was timing ourselves reading a page, or better yet reading 10 pages and averaging the results. If we \emph{train} under a particular circumstance $m$ should only be used for similar scenarios. 

The training under similar scenarios sounds obvious but there may be some applications where maybe there are underlying assumptions that we don't know we are making. In the next example we will see how this could be true in the housing market. 

Before we dive into the next example we should define the single variant model, like we saw above and the multi variant model, which we will see in the next example. In the signal variant case we only considered an independent variable of one dimension, total pages. Next we will look at a model that considers multiple variables that all need to be considered in the dependent variable. 

Our next example discusses a model for valuing a house. There are of course many factors that go into the value of a house. The size of the house in square feet is a metric that would influence the house value. How about number of bedrooms, bathrooms, size of kitchen, size of yard, school district quality, and age of the house. 

We just listed seven metrics that could be used to value a house. I'm sure you can double that list if you wanted but the point here is that we need to consider as many variables as we can to accurately predict the house's value. We know need to define an equation to model our multi variant system. We can change \eq{svlinmodel} to,

\begin{equation}
y = \mathbf{m}^T\mathbf{x} + b.
\end{equation}
\noindent
Here we have a vector $\mathbf{x}$ of numbers that each represent an aspect of the house. In our case $\mathbf{x}$ is a list of seven numbers. The first number, $\mathbf{x}_1$ is the square-footage of a particular house all the way up to the last number, $\mathbf{x}_7$ that represents the age of the house in months. 

Next we can define the vector $\mathbf{m}$ which is another list of numbers of the same length of $\mathbf{x}$ but $\mathbf{m}$ is a weighting or dollar amount associated with each metric of the house. The units assigned to $\mathbf{m}$ will influence how we use our model.

There are a few ways in which we could use this model. First we could come up with all the possible metrics of a house and organize them in $\mathbf{x}$ and $\mathbf{m}$ then do some market research in different neighborhoods in America and look at the going rates and try to calculate the $\mathbf{m}$ values for different places. Clearly New York City will be different than rural Iowa. If you are doing research on trends in housing maybe this is useful. 

We could also use this model in a different way and this way is think would be more applicable to a broader audience. In this model we populate $\mathbf{x}$ and $\mathbf{m}$ we metrics that are valuable to you. If you have to have a house that has more than five bedrooms you put the number of bedrooms on the list. If you want to have a lot of land then that is another metric to put into $\mathbf{x}$. After compiling the list you can then look at some houses that may meet some of your wishes the idea would be to use metrics that include both \emph{have to haves} and also \emph{like to haves} so that a house either doesn't meet the basic needs, exceeds basic needs. Then you can choose the house that exceeds the basic needs for the right price. 

The second use for this model then doesn't have to have an $\mathbf{m}$ vector that has dollar amounts tied to bathrooms, which is difficult to calculate. Instead we can have relative weights or percentages be used. If a smaller house is desired, like downsizing in retirement then the weight assigned to square-footage is higher when square-footage is low. In this case at the end of the calculation you can compare the weighted result for all the houses to determine which one is most appropriate for you and your family. 

In general \index{Linear Regression} models a relationship between independent variables and dependent variables. The relationship is characterized by the single scalar $m$, vector $\mathbf{m}$, or $\mathbf{m}$ can be defined as a matrix which relates the independent variables to multiple dependent variable with a different linear combination or weighting. No matter the number of variables the calculation of $\mathbf{m}$ should only be applied in the context for which it was calculated. For the housing example above $\mathbf{m}$ may need to be recalculated if the housing market changes or your budget changes. 


\section{Linear Discriminant Analysis}
	<TODO Section Linear Discriminant Analysis : PROOF READ>

\index{Linear Discriminant Analysis} is used to reduce the number of dimensions of a problem to enable timely execution of the calculations. Similar to \ac{PCA} the analysis is performed to first reduce the dimensions needs then to perform the task first set out to do. 

In this section we will look at the algorithm to do face recognition. To do facial recognition we first must train the algorithm to know who we can recognize. Assume we have an image of a person where $\mathbf{P}$ is $(M\times N)$ pixels. Furthermore we have a $\mathbf{P}$ for each person say a total of $N_T$ people we would like to be able to recognize. 

To make this example more concrete say we are implementing a security system where a camera is used at the front door. If any of the $N_T$ employees we have walk up to the door we can recognize them and allow them to come in, otherwise we can take further action as necessary. When an employee is hired we need to take a picture of them for a badge \ac{ID} which is common now. This same picture can be our $\mathbf{P}$ used for the new employee. 

Next we need to construct a vectorized representation of all the $N_T$ images we have. The new matrix, $\mathbf{T}$ is $(MN\times N_T)$ where the $\operatorname{vec}$ takes the columns of the matrix and appends them each under the first column. So the result of $\operatorname{vec}{\mathbf{P}}$ is a column vector of the $MN$ pixels. We do this for each image and construct $\mathbf{T}$,
	
\begin{equation}
\mathbf{T} = \left[\operatorname{vec}(\mathbf{P}_1),\dots,\operatorname{vec}(\mathbf{P}_{N_T})\right].
\end{equation}	
	
After the construction of $\mathbf{T}$ we can compute the \ac{SVD} of $\mathbf{T}$. The \ac{SVD} is a computationally intensive algorithm but we will only need to do this in the training phase of the algorithm. We decompose $\mathbf{T}$ with \ac{SVD} as,

\begin{equation}
\mathbf{T}=\mathbf{U}\mathbf{\Sigma}\mathbf{V}^T.
\end{equation}

Now we can use the \ac{SVD} to get the eigenvalues and eigenvectors of the matrix $\mathbf{T}\mathbf{T}^T$. The matrix $\mathbf{U}$ already contains the eigenvectors. To get the eigenvalues we need to multiply the \ac{SVD} result for $\mathbf{T}$ with $\mathbf{T}^T$ like so,

\begin{eqnarray}
\mathbf{T}\mathbf{T}^T&=&\mathbf{U}\mathbf{\Sigma}\mathbf{V}^T\mathbf{V}\mathbf{\Sigma}^T\mathbf{U}^T\\
&=&\mathbf{U}\mathbf{\Sigma}\mathbf{\Sigma}^T\mathbf{U}^T\\
&=&\mathbf{U}\mathbf{\Lambda}\mathbf{U}^T.
\end{eqnarray}	

Now that we have the eigenvalues we can reduce dimensions, via the \ac{PCA} method but removing the dimensions with the smaller eigenvalues. We will now define $\mathbf{U}_{PCA}$ that has $k$ eigenvectors that correspond to the $k$ largest eigenvalues. 

For the \ac{LDA} we need to define two matrices. The first we already calculated which is the \emph{within} image scatter. We will denote the \emph{within} covariance as $\mathbf{S}_w$ which is going to be our $\mathbf{T}\mathbf{T}^T$. The second matrix we need to define is the \emph{between} image scatter, $\mathbf{S}_b$. To calculate $\mathbf{S}_b$ we first need the average across images, $\mathbf{\mu}_P = \expec{T}$. The result is a vector of length $NM$. This is opposed to the mean of an image, which would result in a vector of length $N_T$.

To calculate $\mathbf{S}_b$ we use:

\begin{equation}
\mathbf{S}_b = \mathbf{\mu}_P\mathbf{\mu}^T_P.
\end{equation}

\ac{LDA} aims to minimize $\mathbf{S}_w$ and maximize $\mathbf{S}_b$. To solve this problem we solve the general eigenvalue decomposition or \ac{SVD} of $\mathbf{S}_b\mathbf{S}_w^{-1}$. Which again results in similar calculations for the \ac{PCA} method,

\begin{equation}
\mathbf{S}_b\mathbf{S}_w^{-1} = \mathbf{U}\mathbf{\Sigma}\mathbf{V}^T.
\end{equation}

We now need the eigenvalues for $\mathbf{S}_b\mathbf{S}_w^{-1}$ which are obtained by,

\begin{eqnarray}
\mathbf{S}_b\mathbf{S}_w^{-1}(\mathbf{S}_b\mathbf{S}_w^{-1})^T &=& \mathbf{U}_{LDA}\mathbf{\Sigma}\mathbf{V}^T\mathbf{V}\mathbf{\Sigma}^T\mathbf{U}_{LDA}^T\\
&=& \mathbf{U}_{LDA}\mathbf{\Sigma}\mathbf{\Sigma}^T\mathbf{U}_{LDA}^T\\
&=& \mathbf{U}_{LDA}\mathbf{\Lambda}_{LDA}\mathbf{U}_{LDA}^T.\\
\end{eqnarray}
\noindent
Where once again $\mathbf{U}_{LDA}$ are the eigenvectors with $\mathbf{\Lambda}_{LDA}$ is a diagonal matrix with the eigenvalues. We can perform a dimension reduction based on eigenvalue by keeping the $k$ largest eigenvalues and associated eigenvectors. 

Now that \ac{PCA} and \ac{LDA} are defined we can move forward with the facial recognition algorithm. We first need to project our training images into the eigenspace that our \ac{PCA} or \ac{LDA} produced. 

The term \emph{eigenface} identifies the image in the \ac{PCA} eigenspace. The term \emph{fisherface} is used for an image in the \ac{LDA} space. The term \emph{fisher} is used since $\mathbf{S}_b\mathbf{S}_w^{-1}$ is referred to as the Fisher matrix. From here we define our two \emph{faces} as,

\begin{equation}
\mathbf{F}_E = \mathbf{T}\mathbf{U}_{PCA}~~~~~~~~\mathbf{F}_F = \mathbf{T}\mathbf{U}_{LDA}
\end{equation}	

\cite{eigenfisherface}

% CITATIONClose[8] M. Turk; A. Pentland (1991). "Eigenfaces for recognition" (PDF). Journal of Cognitive Neuroscience. 3 (1): 71–86. doi:10.1162/jocn.1991.3.1.71. PMID 23964806

We are now ready to analyze a new image. Going back to our security system example. We are ready now to start the camera and as people walk up to our door we take an image, after we take the image we vectorize the image, $\mathbf{n}=\operatorname{vec}{\mathbf{P}}$. Next we need to project the new image into the eigenspace of our choosing resulting in $\mathbf{W}$,
\begin{equation}
\mathbf{W}_{PCA} = \mathbf{U}_{PCA}\mathbf{n} ~~~~~~~~~~~~~ \mathbf{W}_{LDA} = \mathbf{U}_{LDA}\mathbf{n}
\end{equation}	

Next we compare $\mathbf{W}$ to the respective $\mathbf{F}$ we subtract and find the magnitude of the result. The calculation results in a vector of distances of $N_T$ length,

\begin{equation}
\mathbf{d} = \norm{\mathbf{F}-\mathbf{W}}^2.
\end{equation}	

The smallest distance represents the most similar image. Some testing is needed depending on the camera resolution and other factors like lighting and the ability to get the person to be looking directly at the camera. In practical systems some experimentation is used to determine ranges for which the distance metric can fall into. If the minimum distance is low enough then the face is recognized. If the minimum distance is low but not low enough to be a match then the face isn't in the library of trained images. Or if the lowest distance is very high the image isn't a face at all. Depending on the system some of the information may not be useful. 
	
\section{Neural Networks}
	<TODO Section Neural Networks : NOT DONE>

\section{Support Vector Machines}
	<TODO Section Support Vector Machines : NOT DONE>

\section{K-Means Clustering}
	<TODO Section K-Means Clustering : NOT DONE>




\backmatter


\clearpage
\chapter{Acronym}
%\begin{acronym}
%    \acro{AOA}{Angle of Arrival}
%    \acro{CRLB}{Cramer-Rao Lower Bound}
%    \acro{FIM}{Fisher Information Matrix}
%    \acro{GPS}{Global Positioning System}
%    \acro{MCRLB}{Modified CRLB}
%    \acro{ML}{Maximum Likelihood}
%    \acro{MLE}{Maximum Likelihood Estimate}
%    \acro{MSE}{Mean Squared Error}
%    \acro{NP}{Neyman-Pearson}
%    \acro{PDF}{Probability Density Function}
%    \acro{RMSE}{Root Mean Squared Error}
%    \acro{PSD}{Power Spectral Density}
%    \acro{ROC}{Receiver Operating Characteristic}
%    \acro{RSS}{Received Signal Strength}
%    \acro{SNR}{Signal to Noise Ratio}
%    \acro{TDOA}{Time Difference of Arrival}
%    \acro{TOA}{Time of Arrival}
%    \acro{WSN}{Wireless Sensor Network}
%\end{acronym}

\begin{acronym}[WPAFB]
  \acro{4G-LTE}{Fourth Generation - Long Term Evolution}	
  \acro{ADC}{Analog to Digital Converter}
  %\acro{AFRL}{Air Force Research Laboratory}
  %\acro{AFIT}{Air Force Institute of Technology}
  \acro{AIS}{Automatic Identification System}
  \acro{ALU}{Arithmetic Logic Unit}
  \acro{ARM}{Advanced RISC Machine}
  \acro{ASCII}{American Standard Code for Information Interchange}
  \acro{ASIC}{Application Specific Integrated Circuit}
  \acro{ATMEGA}{Atmel mega Family}
  \acro{AWGN}{Additive White Gaussian Noise}
  \acro{AXI}{Advanced Microcontroller Bus}
  \acro{AXI4}{Advanced Microcontroller Bus Version 4}
  \acro{BER}{Bit Error Rate}
  \acro{BPSK}{Binary Phase Shift Keying}
  \acro{BRAM}{Block Random Access Memory}
  \acro{CD}{Compact Disc}
  \acro{CE}{Channel Estimate}
  \acro{CFO}{Carrier Frequency Offset}
  \acro{CLB}{Configurable Logic Block}
  \acro{CLI}{Command Line Interface}
  \acro{CLT}{Central Limit Theorem}
  \acro{CORDIC}{COordinate Rotation DIgital Computer}
  \acro{CP}{Cyclic Prefix}
  \acro{CPLD}{Complex Programmable Logic Device}
  \acro{CPU}{Central Processing Unit}
  \acro{CS}{Chip Select}
  \acro{CSI}{Channel State Information}
  \acro{CSS}{Cascading Style Sheets}
  \acro{DC}{Direct Current}
  \acro{DDR}{Double Data Rate}
  \acro{DFT}{Discrete Fourier Transform}
  \acro{DMA}{Direct Memory Access}
  \acro{DSP}{Digital Signal Processing}
  \acro{DUT}{Device Under Test}
  \acro{DVD}{Digital Versatile Disc}
  \acro{EEPROM}{Electrically Erasable Programmable Read-Only Memory}
  \acro{EOF}{End of File}
  \acro{EOL}{End of Line}
  %\acro{Ex}{Eavesdropping Receiver}
  \acro{FCC}{Federal Communications Commission}
  %\acro{FDCE}{Frequency Domain Channel Estimation}
  %\acro{FDE}{Frequency Domain Equalization}
  \acro{FEC}{Forward Error Correction}
  \acro{FFT}{Fast Fourier Transform}
  \acro{FIFO}{First-In First-Out}
  \acro{FIR}{Finite Impulse Reponse}
  \acro{FPGA}{Field Programmable Gate Array}
  \acro{FSK}{Frequency Shift Keying}
  \acro{GAL}{Generic Array Logic}
  \acro{GB}{Giga-Bytes}
  \acro{GHz}{Giga-Hertz}
  \acro{GCC}{GNU Compiler Collection}
  \acro{GMSK}{Gaussian Minimum Shift Keying}
  \acro{GPIO}{General Purpose Input Output}
  \acro{GPU}{Graphical Processing Unit}
  \acro{GSM}{Global System for Mobile Communications}
  \acro{GUI}{Graphical User Interface}
  \acro{HD}{High Definition}
  \acro{HDD}{Hard Disk Drive}
  \acro{HDL}{Hardware Description Language}
  \acro{HTML}{Hyper-Text Markup Language}
  \acro{Hz}{Hertz}
  \acro{I}{In-Phase}
  \acro{IBI}{Inter-Block Interference}
  \acro{I2C}{Inter-Integrated Circuit}
  \acro{IC}{Integrated Circuit}
  \acro{ID}{Identification}
  \acro{ICI}{Inter-Carrier Interference}
  \acro{IEEE}{Institute of Electrical and Electronics Engineers}
  \acro{IF}{Intermediate Frequency}
  \acro{IFFT}{Inverse Fast Fourier Transform}
  \acro{i.i.d.}{Independent and Identically Distributed}
  \acro{IP}{Intellectual Property}
  \acro{IO}{Input Output}
  \acro{IoT}{Internet of Things}
  \acro{ISI}{Inter-Symbol Interference}
  \acro{ISM}{Industrial, Scientific and Medical}
  \acro{ISP}{Internet Service Provider}
  \acro{IQM}{IQ Mismatch}
  \acro{KB}{KiloBytes}
  \acro{KNN}{k-Nearest Neighbors}
  \acro{LAN}{Local Area Network}
  \acro{LDA}{Linear Discriminant Analysis}
  \acro{LED}{Light Emitting Diode}
  \acro{LFSR}{Linear Feedback Shift Register}
  \acro{LOS}{Line-of-Sight}
  \acro{LP}{Low Power}
  \acro{LPD}{Low Probability of Detection}
  \acro{LPF}{Low Pass Filter}
  \acro{LS}{Least Squares}
  \acro{LSB}{Least Significant Bit}
  \acro{LTE}{Long-Term Evolution}
  \acro{LUT}{Look-Up Table}
  \acro{LVDS}{Low Voltage Differential Signaling}
  \acro{MAC}{Multiply Accumulate}
  \acro{Mb}{Megabit}
  \acro{MB}{Mega-Bytes}
  \acro{MHz}{Mega-Hertz}
  \acro{MIMO}{Multiple-Input Multiple-Output}
  \acro{MIPS}{Microprocessor without Interlocked Pipelined Stages}
  \acro{MISO}{Master-In Slave-Out}
  \acro{ML}{Machine Learning}
  \acro{MOSI}{Master-Out Slave-In}
  \acro{ms}{millisecond}
  \acro{MSB}{Most Significant Bit}
  \acro{MSK}{Minimum Shift Keying}
  \acro{NAS}{Network Attached Storage}
  \acro{NCO}{Numerically Controller Oscillator}
  \acro{NRE}{Non-Recurring Engineering}
  \acro{OFDM}{Orthogonal Frequency Division Multiplexing}
  \acro{OFDMA}{Orthogonal Frequency Division Multiple Access}
  \acro{OS}{Operating System}
  \acro{PAL}{Programmable Array Logic}
  \acro{PAPR}{Peak-to-Average Power Ratio}
  \acro{PCA}{Principle Component Analysis}
  \acro{PCB}{Printed Circuit Board}
  \acro{PCIe}{Peripheral Component Interconnect Express}
  \acro{PDF}{Probability Density Function}
  \acro{PIA}{Programmable Interface Array}
  \acro{PIC}{Peripheral Interface Controller}
  \acro{PMF}{Probability Mass Function}
  \acro{PMI}{Precoding Matrix Index}
  \acro{PN}{Pseudo-Random Noise}
  \acro{PPC}{Performance Optimization With Enhanced RISC – Performance Computing}
  \acro{PSD}{Power Spectral Density}
  \acro{Q}{Quadrature}
  \acro{QAM}{Quadrature-Amplitude Modulation}
  \acro{RAM}{Random Access Memory}
  %\acro{RB}{Receiver Block}
  \acro{RGB}{Red Green Blue}
  \acro{RF}{Radio Frequency}
  \acro{RISCV}{Reduced Instruction Set Computing Version 5}
  \acro{RMSE}{Root-Mean Squared Error}
  \acro{ROA}{Region of Activity}
  \acro{RPi}{Raspberry Pi}
  \acro{RTL}{Register-Transfer Level}
  %\acro{Rx}{intended receiver}
  \acro{SATA}{Serial Advanced Technology Attachment}
  \acro{SBC}{Single Board Computer}
  \acro{SC-FDMA}{Single Carrier - Frequency Division Multiple Access}
  \acro{SD}{Secure Digital}
  \acro{SERDES}{Serializer and Deserializer}
  \acro{SDR}{Software Defined Radio}
  \acro{SINR}{Signal to Interference plus Noise Ratio}
  \acro{SIMO}{Single-Input Multiple-Output}
  \acro{SISO}{Single-Input Single-Output}
  \acro{SNR}{Signal to Noise Ratio}
  \acro{SoC}{System on Chip}
  \acro{SOP}{Sum of Products}
  \acro{SPI}{Serial Peripheral Interface}
  \acro{SPLD}{Simple Programmable Logic Device}
  \acro{SRAM}{Synchronous Random Access Memory}
  \acro{SSD}{Solid State Drive}
  \acro{STBC}{Space-Time Block Codes}
  \acro{SVD}{Singular Value Decomposition}
  \acro{SVM}{Support Vector Machine}
  \acro{TB}{Tera-Bytes}
  \acro{TCL}{Tool Command Language}
  %\acro{TDCE}{Time Domain Channel Estimation}
  \acro{TI}{Texas Instruments}
  \acro{TI-DSP}{Texas Instruments-Digital Signal Processor}
  %\acro{TOC}{Tactical Operation Centers}
  %\acro{TOR}{Time Of Reference}
  %\acro{TRB}{Transmitter-Receiver Block}
  %\acro{UAVs}{Unmanned Aerial Vehicles}
  \acro{UART}{Universal Asynchronous Receiver Transmitter}
  \acro{uC}{micro-Controller}
  \acro{uP}{micro-Processor}
  \acro{UI}{User Interface}
  \acro{uPP}{Universal Parallel Port}
  \acro{us}{microsecond}
  \acro{US}{United States}
  \acro{USB}{Universal Serial Bus}
  \acro{USRP}{Universal Software Radio Peripheral}
  \acro{UVM}{Universal Verification Methodology}
  \acro{V}{Voltage}
  \acro{VHDL}{VHSIC Hardware Description Language}
  \acro{VLSI}{Very Large Scale Integration}
  %\acro{WARP}{Wireless open-Access Research Platform}
  \acro{WLAN}{Wireless Local Area Network}
  %\acro{WPAFB}{Wright Patterson Air Force Base}
\end{acronym}

  % put this at end of paper

\printindex
\bibliographystyle{IEEEtran}
\bibliography{thesis}

\end{document}
