\documentclass[]{book}

% Import User Packages 
\usepackage[nolist]{acronym}

%opening
\title{Field Programmable Gate Arrays: What Are They and Why to Use Them}
\author{Dr. Jason Pennington}

\begin{document}

\maketitle

\tableofcontents

%\chapter{Overview of Computing Platforms}
	<TODO Chapter Overview of Computing Platforms : PROOF READ>
Today's electronics hobbyist has many platforms available to learn anything from programming basics to complicated topics like web hosting and robotics. This chapter discussed all the possibilities of available types of computing platforms. We discuss Single board computer's, Field Programmable Gate Arrays, and Application Specific Integrated Circuits. All of which have varying levels for skill required to get started. This book will get you started in designing systems for all the platforms.  

If you are a beginner in all this then you should start with single board computers. They are a smaller and cheaper version of your laptop. They offer an operating system and are user friendly. They are not as computationally equipped but do offer some ability to do complex processing. 

A Graphical Processing Unit (GPU) offers more processing power depending on the application. GPUs offer performance gains when the algorithm has a lot of multiply-accumulate operations. However the working memory can be limiting; the overhead of moving data to and from the host CPU degrades performance.

For high-data rate applications where large data sets are ingested by algorithms FPGAs can be used to select significant pieces of the data for analysis. This data selection concept can be a little abstract but for particular applications it can be applied quite effectively. For instance in image processing a portion of the image can be eliminated or just edges are considered. 

Finally, the last step and the ultimate in computational performance is the Application Specific Integrated Circuit. These chips are touched on briefly here for the sake of completeness however the hobbyist would only explore this option if the hobby became an amazing success and would pay for a chip to be manufactured. 
	
	
\section{What Can an FPGA Do}
	<TODO Section What Can an FPGA Do : PROOF READ>

The development of todays FPGAs started in the late 1988s, where Simple Programmable Logic Devices took the place of hand placing logic gates. These SPLDs gave way to Complex Programmable Logic Devices. This section provides a brief history which will help in the understanding of how and why FPGAs are used.  	
	

\subsection{History of FPGAs}
	<TODO Subsection History of FPGAs : NOT DONE>

\subsubsection{SPLDs and CPLDs}
	<TODO Subsubsection  SPLDs and CPLDs : NOT DONE>

\subsubsection{Macrocells}
	<TODO Subsubsection  Macrocells : NOT DONE>

\subsection{Why Use an FPGA}
	<TODO Subsection Why Use an FPGA : PROOF READ>
	
FPGAs are really the blank slate of embedded processing. The features in FPGAs today lead to a large range of applications, including Artificial Intelligence and machine learning. FPGAs have many characteristics that make them a great versatile computational platform. 

Most people of familiar with processors, CPUs, or microprocessors. The architecture of these platforms vary great but at the root of each processor has one or more ALU. An FPGA can have an ALU implemented in it. Research in processor design use FPGAs to test new architecture performance and prove out designs. This research is enabled by the ability to reprogram the FPGA to make design changes and test whether the changes are an improvement in practice. 

Another ability that makes FPGAs versatile is the ability to add additional logic in parallel. When an algorithm is being implemented the programmer is looking to optimize the code to get the result as fast as possible. With an FPGA the programmer can add another operation in parallel as opposed to waiting for an Arithmetic Logic Unit to be ready to do the calculation. 

Another benefit of an FPGA is the amount of configurable input and output pins. If your application needs to interface to many peripherals FPGAs have hundreds even thousands of pins available for configuration. Some pins are high performance pins that can be used to interface to DDR, SATA, and PCIe controllers. 

Finally, an FPGA is a great testing platform for ASIC development. The ultimate in performance is the ASIC however you get one chance for a die. The NRE for a second attempt can make it not cost effective.

\subsubsection{Introduction to Data Rate}
	<TODO Subsubsection  Introduction to Data Rate : PROOF READ>

A data rate in general is defined as the number of bits per unit time. For a system the data rate is how much data can the system handle per unit time. The overall data rate of the system is the minimum data rate for the individual components. In designing a system with a minimum data rate requirement it is important to be mindful of the data rates of the components.

To make the concept of determining data rate a little more concrete lets look at a simple architecture of a wireless receiver. To determine the data rate of a system the knowledge of the application and more importantly the architecture of the system needs to be known. Even for a simple receiver there are multiple ways to architect the design to enable higher or lower data rates. 

\subsubsection{Data Rate Study - Image and Video Processing}
	<TODO Subsubsection  Data Rate Study - Image and Video Processing : NOT DONE>

\section{What Can a SBC Do}
	<TODO Section What Can a SBC Do : NOT DONE>

\subsubsection{The Performance of Decisions}
	<TODO Subsubsection  The Performance of Decisions : NOT DONE>

\subsubsection{Data Rate Study - Web Hosting}
	<TODO Subsubsection  Data Rate Study - Web Hosting : NOT DONE>

\section{FPGA vs. Processor}
	<TODO Section FPGA vs. Processor : NOT DONE>

\subsubsection{Balancing Data Rate and Decision Making}
	<TODO Subsubsection  Balancing Data Rate and Decision Making : NOT DONE>

\chapter{Overview of Computing Platforms}
\section{What can an FPGA Do}
\section{What can SBCs Do}
\section{What can SBCs Do}

%\chapter{What is a Field Programmable Gate Array}
A \ac{FPGA} is a chip that is first \emph{field programmable} which means that the chip is reconfigurable after it leaves the foundry or the manufacturer of the chip. For example the \ac{FPGA} could perform the operations of a transceiver then minutes later, after reconfiguration, the \ac{FPGA} can perform image processing algorithms then could act as a web server. Because of this versatility the \ac{FPGA} has become a very cost effective solution as opposed to \ac{ASIC}.

The second aspect of an \ac{FPGA} is that the resources that are available are in a \emph{gate array}. The architecture of the gate array changes with newer versions of \ac{FPGA}. New versions of \ac{FPGA} attempt to optimize the resources for any application the \ac{FPGA} is used. The versatility of the \ac{FPGA} makes this optimization very difficult and because of this the efficiency of the design truly lies on the developer.

The responsibility of code efficiency can be a foreign to traditional software developers especially when the goal is to get code working; efficiency is an afterthought. The idea of \emph{efficiency later} gets \ac{HDL} programmers into trouble. For example if the programmer is not aware that the \ac{RAM} available on the \ac{FPGA} is dual-port and the programmer accesses more than two memory addresses in one clock cycle the synthesizer will interpret the \ac{RAM} as distributed \ac{RAM} instead of Block-\ac{RAM}. Depending on the size of the \ac{RAM} this is a costly mistake because distributed \ac{RAM} is implemented in \ac{LUT} which could be used for other logic. Many examples such as these are found in \ac{HDL} code, and this book aims to list out good practices to avoid such pitfalls.

This Chapter covers a broad topic of what an \ac{FPGA} is and to understand this we will discuss alternatives to the \ac{FPGA}; namely, the \ac{uC} and the \ac{ASIC}. Next, a couple of examples of applications of \ac{FPGA} are discussed for the final section of this chapter in which the resources available on the \ac{FPGA} are described and some common errors seen when using the resources with particular examples from the previous section. 


\section{FPGA Alternatives and Data Rate Capabilities}
There are many platforms for embedded processing. A Raspberry Pi is a very popular platform for the electronics hobbyist and the ARM processor that is available on the Raspberry Pi is very powerful for an embedded processor. However, there are applications, such as image processing, that a little more computing power is needed. This section discusses determining the requirements for the application and determining if a particular platform can handle the data rate required by the application. 

\subsection{Introduction to Data Rate}
A data rate in general is defined as the number of bits per unit time. For a system the data rate is how much data can the system handle per unit time. The overall data rate of the system is the minimum data rate for the individual components. In designing a system with a minimum data rate requirement it is important to be mindful of the data rates of the components.

To make the concept of determining data rate a little more concrete lets look at a simple architecture of a wireless receiver. To determine the data rate of a system the knowledge of the application and more importantly the architecture of the system needs to be known. Even for a simple receiver there are multiple ways to architect the design to enable higher or lower data rates. 


\subsection{Micro-Controllers (uC)}
\ac{uC}'s are embedded processors for use in \emph{small} systems. What is meant by a small system is relative, usually embedded systems are programmed for a specific task. For example live-streaming a webcam to the \ac{LAN}. \ac{uC}'s come in varying sizes; depending on the task to be performed. You don't need the cutting-edge i7 from intel to stream video, but an 8-bit \ac{PIC} \ac{uC} is too small for the task. However, an 8-bit \ac{uC} will turn on a water pump for 10 minutes a day to water a house hold plant. Just as different vehicles suit owner's needs in different ways so do \ac{uC}'s.

\subsection{Application Specific Integrated Circuits}
\ac{ASIC} chips are designed, of course, for a specific application. The knowledge of the algorithms to be calculated and the ability in an \ac{ASIC} to make any gate on the blank silicon make the \ac{ASIC} the most power processing platform. 
 

\section{Applications for FPGAs}
%In general applications for fpga a lend themselves applications for which data rates define hired generate in this section teradata rates in fpj Linda self NFPA 58 implantation lends itself over a microcontroller nfpga can handle names of multiple many streams of parallel incoming or outgoing information microcontroller a decision between an fpga 96 is volume how many units will be sold used by company cost havana section is the application of a transceiver sing and photography all of which have higher data rate requirements

\subsection{Transceivers}
%The transceivers the best examples are Wi-Fi hotspots and cell phone contact Tiffany these applications the volume of units sold such as cell phones or Wi-Fi routers the volume have to wear an e-cig would make a cost-effective solution the academic community research into cognitive radios 5th generation Wireless technology determining whether there is a lot of interference in a particular band switching to a different band and communicating their news this is because the higher data rates however go to and we are implementing prototypes on fpgas learning disability of atoms a simple example of a cognitive radio or a simple example of sensing the environment and then transmitting on that all that band is telephone where the playStation 4 hook up to phone for the transceiver and when you first did that the phone and the base station negotiate between two channels noizy which one has less interference it's a simple example of 22 channels that trans smith's information on one channel and then on the second channel the receiver says which one was no easier which one did I get what's the transmitter or the phone no simple algorithm can do that however that out of them is overhead in the Wireless Communications since voice is a state that is relatively low gatorade place that has a relatively low generate can get away with doing this at the beginning make it seamless transition to the phone call in cell phone technology I will get more complicated it won't just be it to him a little bit exciting between for the different users of the cell phones so maybe personne person and person B have two different channels that they are experiencing because they're in two different locations cellular network they both can't use the same frequency because they're still in the range of the cell phone tower has a better frequency ranges that be making sure each user gets frequency band they are clear to transmit denne on top of that if person a is using frequency range strange but there is a wall in there traffic person be good sense white spaces of the band and use that white space to transmit their own date that is the ultimate goal for a cognitive radio however metrics need to be ensured that they don't violate FCC regulations and hinder the communication of transmitter it is the transmitter. Would not have any knowledge or need to have any knowledge of the signal transmitter app just characterizing the regular traffic of transmitter and using the white space in between Transmissions possible ace Transmissions are not regular or they're not predictable than transmitter B would take it pre-action and not transmit on the band idea of how this algorithm of work sampling the environment environment and getting a spectrum map the environment this means nC Rangers time I need to be a balance between sensing and does in general can't do both at the same time in the section with motivated the use of fpga for next-generation transceivers in particular cognitive radio where a continuous or transmitting is needed to operate in these bands

\subsection{Image Processing}
%Image processing image processing is a broad research in this section image image processing does not mean transmitting video over a transceiver if we had high-definition video and we are broadcasting that over a wireless network that would just be a high generate transceiver to be able to broadcast that much data the fall under the previous transceiver application for fpga in this section image processing talk about not only having a a camera taking data so an application of where say road and you want to put a red circle around every car that is driving down the road you can track hcar maybe you want to identify each car maybe you want to try the car and find a license plate of each car these are the applications of image processing that long so it's very well to fpga there are two out rhythms two basic items that come to mind for image processing first of which is correlation for a two dimensional correlation or we look for a known object may be filled waldo anywhere the Balto book you would know what all that looks like you would need to correlate main difference size of his face biggest faces in the book to the page of the a smaller example of that application smaller example would be you have a 128 by 128 image image when you wanted the terminal where these four pixels are in that 128 by 128 fashion To Figure you can move lacrosse tween the four pixels 4 pixels that is currently overlay Don you determine the Enterprise with the maximum inner product is where those images of where your estimate 3x tomatoes for pizza sauce the reason why this application lends itself to his pj's so well is that as you were sampling the high resolution azure sampling the high-resolution images from the camera the inner product the inner product of the four pixels application image for is independent of the inner product of the four pixels when's the state is available wait until the rest of the image is in the LPGA you can be operating on this while you're out ceiling frame from the camera a lot of times in high speed generate or how long to digital converters multiple lanes are provided at the same clock cycle so maybe instead of depending on the speed of the camera can you get 42 pixels rising edge of the clock maybe that represents one half of 1/5 of a column in the image so in this case and that example 16 calculations can start 16 calculations of calculations for 16 independent inter products can start the other application are the other algorithm for image processing has the deal with eigenvalue or some sort of naked Matrix decomposition I'd refer you to reference

\subsection{Cryptography}
%Sakura Transmissions also a great application for fpgas is it because the extreme that you would normally send in the clear available to you on the fpga it's what you're interested in any way the images that you wanted the transmitter or the text that you wanted to oyster as that voice data or images are being processed you can provide this to and intensity on the fpga to encrypt that information and you don't need to wait for all of the information to be available to you cryptographic are there such as elliptic curve cryptography or RSA pS 128 or 256 the show albums random very well to an fpga Plantation for more information on this topic please visit Wikipedia

\section{Architecture of an FPGA}
%Architecture of an fpga provides many resource blocks to the user for processing is resource blocks include Black Ram restoring information the user as the user is programming dedicated multipliers because multiplication and division in Hardware is resource-intensive if using look up tables there are dedicated multipliers available on certain models of fpga from the different vendors is also the switching matrices which are for which data can be routed throughout the year those look up tables clock trees and global inputs and outputs from the SPCA to get your data in today at PGA and off of you quickly this section will discuss all of the resources available to the user on the mpg

\subsection{Block RAM}
%Black Ram is a dedicated memory to the user must be used carefully is my block right and need to be used carefully because most of the time block Ram has either a single or dual port single or dual program because of this only one or two respectively memory blocks Indiana candy addressed in one clock cycle if your Hardware or if your vehicle is accessing multiple addresses the same clock from the same block Ram the synthesizer will interpret this as distributed instantiate that on the fpga fabric and not use block Ram it's a very inefficient use of distributor of your fpga resources you should be redesigned to pool from the Block Ram one or two depending on the architecture of the block grant you using on your fpga under 1 Clarkson


\subsection{Multipliers}
%Multiplier
%Multipliers are dedicated Hardware that can be used on any claw catch multiply two values they can also be used for division for applications such as image processing or wireless transceiver filters for manipulation alone Souls very well that using the dedicated multiplier all the major vendors for fpga provide more dedicated multipliers

\subsection{Switching Matrices}
%switching matrices are used to Route information from resource block searches look up tables and other resources switching matrices can you can think of as interceptions in a city block where the switching matrices are configured at startup to wrap hair in a certain direction writing in a certain direction switching matrices can be configured once at startup and after that they are I can figure out certain way what is possible that you can figure matrix does cut off other resources from the user the place and Route algorithm in nevados Arizona suppose Idaho tool and Altera quartus tool try to optimize this wrapped root price of Maserati my the routing optimize the oh my Godrouting due to the complexity of the lpga's resources complexity of the albums that are being developed for fpga say please and write algorithms are very it is not uncommon for design Institute hours place around


\subsection{Look Up Tables}
%Look up tables who used to do the on the fpga they can be configured in 4many different ways 4 amp five important 16 plus we're fixin to go in any set of outputs can be driven from the inputs defenders of the lpga's have different texas for the Lots what are packaged into aces or carbs slices or C RVs the slices RC Arby's table have a multiplexer and have a register inside them these registers there can be used or not use depending on the algorithm and the multiplexer is configured to Output the results from the left or help with the register having an understanding of the architecture of the fpj you're targeting whether it's 1206 input for input food fit on the internet the synthesizer and the place and route job easier understand the architecture of the board they are targeting


\subsection{Clock Trees}
%Clock trees are resources that are used in every desire every synchronous design clock trees are a dedicated source for providing the clocks to all of your entities and all of your logic set of the logic that is fpga is because it offset between areas of the fpga there could be timing violations violations occurred when the isn't that a steady-state rock Rises or Falls and that Value Inn correct value is registered if you have clock skew or clock differences between pGA you maybe bring a string values from another at the fpga an inaccurate time dedicated dedicated 4 o'clock that minimize in vhdl if you say it's rising it and then whatever the verizon Edge is that value will be allocated a clock in the clock tree it is important to not get your clock it is important to not put signals you don't want to be considered a clock price match if rising Edge

\subsection{Inputs and Outputs}
%The most designed pleasing to the outside world or referrals on the board is going to be necessary whether it's there a microphone the ethernet the push button LED any of these are inputs and outputs to the there are different size packages and different size models of fpga is providing different amounts of CO2 the SPCA low voltage differential signaling most of the LPGA high bandwidth my clock frequency data rate are hiding rate for the SPCA in determining which fpga is right for you and how many inputs and outputs you would need to your fpga is necessary
%Programmable Logic






\chapter{FPGAs in Depth}

%\chapter{What is a Single Board Computer}

\section{Parts of a Single Board Computer (SBC)}
\subsection{Microprocessor(s)}
\subsection{Memory}
\subsection{Input and Output}

\section{Applications for SBCs}
\subsection{Retro Gaming}
\subsection{Small Web Server}
\subsection{Amazon Echo}
\subsection{Christmas Light Show}
\subsection{Facial Recognition Time Clock}
\subsection{Stock Ticker}
\subsection{FPGA Test Vector Generator}



\chapter{SBCs in Depth}

%\chapter{Design Flow Methodology}

\section{Complex Designs Need Simple Steps}

\section{Functional Verification}
\subsection{Overview of Design Flow}
\subsection{System Level Model}
\subsection{System Level Model with Hardware}
\subsection{HDL with testbenches}
\subsection{Synthesis and verify resource usage}

\section{Code Coverage}
\subsection{Universal Verification Methodology (UVM)}
\subsection{System Verilog}
\chapter{Design Flow Methodology}

%\chapter{FPGA Programming Basics}
	<TODO Chapter FPGA Programming Basics : NOT DONE>

\section{Installing Software}
	<TODO Section Installing Software : NOT DONE>

\subsection{Xilinx}
	<TODO Subsection Xilinx : NOT DONE>

\subsection{Altera}
	<TODO Subsection Altera : NOT DONE>

\subsection{Microsemi}
	<TODO Subsection Microsemi : NOT DONE>

\subsection{Lattice}
	<TODO Subsection Lattice : NOT DONE>

\section{Number Representation}
	<TODO Section Number Representation : NOT DONE>

\subsection{Fixed Point}
	<TODO Subsection Fixed Point : NOT DONE>

\subsection{Floating Point}
	<TODO Subsection Floating Point : NOT DONE>

\section{Basic Gates and Analysis}
	<TODO Section Basic Gates and Analysis : NOT DONE>

\subsection{Combinatorial Logic Analysis}
	<TODO Subsection Combinatorial Logic Analysis : NOT DONE>

\subsubsection{Basic Adders}
	<TODO Subsubsection  Basic Adders : NOT DONE>

\subsubsection{Decoders and Encoders}
	<TODO Subsubsection  Decoders and Encoders : NOT DONE>

\subsubsection{Multiplexers and De-multiplexers}
	<TODO Subsubsection  Multiplexers and De-multiplexers : NOT DONE>

\subsubsection{Parity Generators and Checkers}
	<TODO Subsubsection  Parity Generators and Checkers : NOT DONE>

\subsection{Sequential Logic}
	<TODO Subsection Sequential Logic : NOT DONE>

\subsubsection{Latches and Flip-flops}
	<TODO Subsubsection  Latches and Flip-flops : NOT DONE>

\subsubsection{Counters}
	<TODO Subsubsection  Counters : NOT DONE>

\subsubsection{Shift Registers}
	<TODO Subsubsection  Shift Registers : NOT DONE>

\subsubsection{Memory and Storage}
	<TODO Subsubsection  Memory and Storage : NOT DONE>

\section{VHDL Intro}
	<TODO Section VHDL Intro : NOT DONE>

\subsection{Signal Types}
	<TODO Subsection Signal Types : NOT DONE>

\subsection{Sequential Statements}
	<TODO Subsection Sequential Statements : NOT DONE>

\subsection{Subprograms}
	<TODO Subsection Subprograms : NOT DONE>

\subsection{State Machines}
	<TODO Subsection State Machines : NOT DONE>

\subsection{Generics}
	<TODO Subsection Generics : NOT DONE>

\subsection{File IO}
	<TODO Subsection File IO : NOT DONE>

\section{VHDL Advanced}
	<TODO Section VHDL Advanced : NOT DONE>
	
\subsection{Asynchronous Resets}
	<TODO Subsection Asynchronous Resets : NOT DONE>
		
\subsection{Hierarchy} 
	<TODO Subsection Hierarchy} : NOT DONE>

\subsection{Pipelining}
	<TODO Subsection Pipelining : NOT DONE>

\subsection{Data Rates}
	<TODO Subsection Data Rates : NOT DONE>

\subsection{Clock Domain Crossing}
	<TODO Subsection Clock Domain Crossing : NOT DONE>
	
\section{TCL Scripting}
	<TODO Section TCL Scripting : NOT DONE>

\subsection{Version Update}
	<TODO Subsection Version Update : NOT DONE>

\subsection{Automate Simulations}
	<TODO Subsection Automate Simulations : NOT DONE>

\subsection{Automate Build Tools}
	<TODO Subsection Automate Build Tools : NOT DONE>



\section{What is a Micro-Controller}
\section{Why Use a Micro-Controller}
\subsection{Quick Development Time}
\subsection{Low Data Rate Applications}
\subsubsection{Web Server}
\subsubsection{Sensor Controller}
\section{Case Study: Image Processing}
\subsection{System Data Rate Study}
\subsection{Correlation Scan Across Picture}

\section{Introduction to Hardware Scripting}
\section{Intro to TCL}
\subsection{Changing the Build Version}
\subsection{Automating Simulations and Testbenches}
\subsection{Automating Builds}
\section{Intro to Python}
\section{C language basics}

%\chapter{Single Board Computer Programming Basics}

\section{Linux Basics}

\section{Controlling GPIO}






%\chapter{Controlling Peripherals}
%\section{pull up resistor}
%\section{debouncing}
%\section{Storage (in flash EEPROM)}
%\section{LCD Displays}

%\chapter{Hello World and Other Projects}
	<TODO Chapter Hello World and Other Projects : PROOF READ>

There is no better way to get started than to work on a small project. This chapter outlines small projects to work on where each of them gets a little more challenging. As you progress through this chapter you will ensure your development environment is setup properly with simulating the hello world project. You will learn how to use a clock in a design and control an \ac{LED}. Then you will learn how to transfer data into and out of the \ac{FPGA} chip. 
	
	
\section{Beginner Projects}
	<TODO Section Beginner Projects : PROOF READ>

Everyone has to start somewhere. These projects are designed to start off slowly. It may not be very exciting to add one to a counter but these exercises get you used to writing \ac{VHDL} and the development environment. Work through these exercises so that when we can get to exciting projects you won't be frustrated with working with the tools. 
	
\subsection{Hello World}
	<TODO Subsection Hello World : PROOF READ>

When learning a new language you always need to write the hello world program. There are two lessons from this project. First, setting up and interacting with the features of your simulator. Once you write the first draft of your program you go through the steps of compiling and simulating your design. You will use this process many times. Gain an understanding of keyboard short cuts and practice them. Make changes to your file and recompile and ensure the changes are reflected when you re-simulate.

The other main take-away from this example is the use of the report function. The report function allows you to print strings out to the console. The report function can be used to print out values of signals to make sure the simulation is progressing like you want in a larger more complex simulation. Or it can be used to report progress of the simulation.      

\begin{VHDLlisting}[tabsize=8]
library ieee;
  use ieee.std_logic_1164.all;
  use ieee.numeric_std.all;
  
entity hello_world is
end entity hello_world;

architecture tb of hello_world is
begin
	process
	begin
		report("Hello World!");
	end process;
end tb;
\end{VHDLlisting}
	
	
\subsection{Counter Test-Bench}
	<TODO Subsection Counter Test-Bench : PROOF READ>

In this exercise we will instantiate a \ac{DUT} in a testbench. The \ac{DUT} itself is a simple counter. We will provide a clock to the counter and a reset line that will reset the counter. We will also have an enable line that turns the counter on. A simple \ac{DUT} like this has a few inputs that will need to be driven by the testbench.      
	
\begin{VHDLlisting}
library ieee;
  use ieee.std_logic_1164.all;
  use ieee.numeric_std.all;
  
entity tb_counter is
end entity tb_counter;

architecture tb of tb_counter is
	signal w_clk         : std_logic := '0';
	signal w_rst         : std_logic := '0';
	signal w_en          : std_logic := '0';
	signal w_count_sync  : unsigned(7 downto 0):=(others => '0');
	signal w_count_async : unsigned(7 downto 0):=(others => '0');

begin
	p_gen_stim : process
	begin
		w_rst <= '1';
		wait for 37.9 ns;
		w_rst <= '0';
		
		wait for 10 ns;
		w_en <= '1';
		wait for 50 ms;	
	end process;
	
	u_counter_sync : entity work.counter_sync
	port map(i_clk    => w_clk,
			 i_rst    => w_rst,
			 i_en     => w_en,
			 o_count  => w_count_sync
	);
	
	u_counter_async : entity work.counter_async
	port map(i_clk    => w_clk,
			 i_arst   => w_rst,
			 i_en     => w_en,
			 o_count  => w_count_async
	);
	
	p_clk : process
	begin
		wait for 5 ns;
		w_clk <= not w_clk;
	end process;
end tb;
\end{VHDLlisting}

In our testbench we have a process that generates the timing for the input signals. This testbench is only as good as the assumptions made in the timing between signals. If there is a race condition or an inter-timing bug in the \ac{DUT} that the testbench does not cover then the \ac{DUT} will pass the testbench giving you confidence in the \ac{DUT}. Then you have the worst case scenario: debugging a flaky build on hardware. So we need to make sure we make our code reliable to ensure there are no race conditions. In this exercise we look at two examples of a counter. The only difference in the two examples is how the reset is handled, synchronously or asynchronously. We will explain the considerations needed when using this counter in a higher level block.      

\begin{VHDLlisting}[tabsize=8]
library ieee;
  use ieee.std_logic_1164.all;
  use ieee.numeric_std.all;
  
entity counter_sync is
port(i_clk   : in    std_logic;
	 i_rst   : in    std_logic;
	 i_en    : in    std_logic;
	 o_count :   out unsigned(7 downto 0)	
)
end entity counter_sync;

architecture rtl of counter_sync is
	-- Output Register
	signal f_count : unsigned(7 downto 0) := (others => '0');
begin

	-- Assign Outputs
	o_count <= f_count;

	p_count : process(i_clk)
	begin
		if rising_edge(i_clk) then
			if i_rst = '1' then
				f_count <= (others => '0');
			elsif i_en = '1' then
				f_count <= f_count + 1;
			end if;		
		end if;
	end process;
end rtl;

\end{VHDLlisting}
	
First the synchronous reset counter. We see in the code that the reset is inside the rising\_edge(i\_clk) if statement. That means that the reset can only happen synchronous to the i\_clk. We can see that both the reset and enable are synchronous. In general synchronous logic is preferred since the vendor tools ensure the propagation delays between the registers are short enough to avoid erroneous results.


\begin{VHDLlisting}[tabsize=8]
library ieee;
  use ieee.std_logic_1164.all;
  use ieee.numeric_std.all;
  
entity counter_async is
port(i_clk   : in    std_logic;
	 i_arst  : in    std_logic;
	 i_en    : in    std_logic;
	 o_count :   out unsigned(7 downto 0)	
)
end entity counter_async;

architecture rtl of counter_async is
	-- Output Register
	signal f_count : unsigned(7 downto 0) := (others => '0');
begin

	-- Assign Outputs
	o_count <= f_count;

	p_count : process(i_clk)
	begin
		if i_arst = '1' then
			f_count <= (others => '0');
		elsif rising_edge(i_clk) then
			if i_en = '1' then
				f_count <= f_count + 1;
			end if;		
		end if;
	end process;
end rtl;

\end{VHDLlisting}
	
There are situations when you have to use asynchronous resets. If we assume we will use this counter in a situation where we have to use an asynchronous reset we would write this block slightly differently where the reset line is used outside the clocked region of the process. Now if the i\_reset line changes or the i\_clk line changes the process executed but there doesn't have to be a rising\_edge event for the reset to apply. There is potential issue with instantiating this block with an asynchronous reset. It is possible that the reset fails timing. If the asynchronous reset is applied right before a clock edge where the counter is enabled there could be a mismatch between the expected value in the counter and the registered value. Unfortunately, the vendor tools can't warn you about the asynchronous reset failing timing since it isn't under a clocked process. The vendor tools can't know the timing of the asynchronous reset.	
	
	
	
\subsection{Blink Led with a One Second Interval}
	<TODO Subsection Blink Led with a One Second Interval : PROOF READ>

In the project you will write \ac{HDL} to blink an LED. The LED should be on for a second and off for a second. You will also need to write a testbench for the block. You will find that it will take a long time to simulate a block for multiple seconds. We will once again have a synchronous reset, enable, and a clock as inputs. The output will be wired up to an LED.

To toggle the output we need to know when a second passes. The only way to mark time passing on an \ac{FPGA} is to count clock cycles. So we can setup a counter to increment when enabled and once we get to a certain number of clock cycles we can toggle the output. 

\begin{VHDLlisting}[tabsize=8]
library ieee;
  use ieee.std_logic_1164.all;
  use ieee.numeric_std.all;
  
entity blink_led is
generic(gTOGGLECOUNT : integer)
port(i_clk   : in    std_logic;
	 i_rst   : in    std_logic;
	 i_en    : in    std_logic;
	 o_led   :   out std_logic	
)
end entity blink_led;

architecture rtl of blink_led is
	-- Output Register
	signal f_led   : std_logic := '0';

	-- Duration Counter
	signal f_count : unsigned(31 downto 0) := (others => '0');
begin

	-- Assign Outputs
	o_led <= f_led;

	p_count : process(i_clk)
	begin
		if rising_edge(i_clk) then
			if i_rst = '1' then
				f_count <= (others => '0');
			elsif i_en = '1' then
				f_count <= f_count + 1;
			end if;	

			if f_count = gTOGGLECOUNT then
				f_led <= not f_led;
				f_count <= (others => '0');
			end if;			
		end if;
	end process;
end rtl;

\end{VHDLlisting}

The certain number we need to count to is just the frequency of the clock you re using. Here is a fantastic opportunity to learn about generics. Generics can help you configure a block when instantiated. We could use a generic called gCLKFREQ but if we wanted the output to toggle at a rate other than a second this could be confusing. We will use the generic gTOGGLECOUNT. This way if we know how the counter is used and when instantiated we can set the value accordingly.
	
\begin{VHDLlisting}[tabsize=8]
library ieee;
  use ieee.std_logic_1164.all;
  use ieee.numeric_std.all;
  
entity tb_blink_led is
end entity tb_blink_led;

architecture tb of tb_blink_led is
	constant k_TOGGLECOUNT : integer := 100000000;

	signal w_clk : std_logic := '0';
	signal w_rst : std_logic := '0';
	signal w_en  : std_logic := '0';
	signal w_led : std_logic := '0';
begin

	p_gen_stim : process
	begin
		w_rst <= '1';
		wait for 34 ns;
		w_rst <= '0';
		
		wait for 15 ns;
		w_en <= '1';
		
		wait;
	end process;

	u_blink_led : entity work.blink_led
	generic map(gTOGGLECOUNT => k_TOGGLECOUNT)
	port map(i_clk   => w_clk,
		     i_rst   => w_rst,
		     i_en    => w_en ,
		     o_led   => w_led
	);

	p_clk : process
	begin
		wait for 5 ns;
		w_clk <= not w_clk;
	end process;	
end tb;
\end{VHDLlisting}

The testbench is simple with the reset, enable, and clock to generate. The difficult part of this testbench though is how long it will take to simulate. At first when you get the testbench working you don't want to have to wait 30 minutes for the simulation to finish. So at first we will set gTOGGLECOUNT to 10 and make sure the logic works. After getting the simulation two work then we can increase it and make sure the output toggles at the expected rate. 

\section{IO Interfacing}
	<TODO Section IO Interfacing : PROOF READ>
	
This section of projects gets data in to and out from the \ac{FPGA}. The list of projects here are the easiest ways of setting up two way data transfer off the chip. These protocols are popular in interacting with a computer or with other chips on the same board.

Although these projects are simple and great starting exercises you will reuse these blocks a lot when interfacing to different peripherals. We will discuss making these blocks generic so that they can be used with minimal changes on a variety of projects.

\subsection{Universal Asynchronous Receiver-Transmitter (UART)}
	<TODO Subsection Universal Asynchronous Receiver-Transmitter (UART) : PROOF READ>
	
The \ac{UART} is a very common low speed protocol. It's primary use is commanding and controlling the FPGA to do a task when a user issues the command through a general purpose computer. The \ac{UART} is also great for status or debugging what is happening on the \ac{FPGA}.

A \ac{UART} is an asynchronous protocol so there is no clock associated with the data. Since there is no clock or an enable line to know when the data is valid we first need to know the baudrate of the data. It is assumed that the datarate is known between the computer and \ac{FPGA}. Since the datarate is known once we get a bit we will know when the next bit is valid. To know when the first bit is valid the \ac{UART} protocol starts with the receive line high. Data is starting to be transmitted when the line goes low. At the falling edge of the data line we count half the bit duration. Once half the bit duration is reached we check again to make sure the data line is low. If the data line is low then we know we are receiving data. Once we have started the transfer the predetermined number of bits are sent. After the falling edge of data line there is a start bit, a stop bit, and a configurable number of data bits. The start bit is what we used to detect the start of the transfer. Then we count a bit duration for each of the data bits. Finally the stop bit can be ignored or sometimes can be configured as a parity bit.  Now we will look at some \ac{HDL} that performs the above tasks. We will use a state machine for this since it is easier to read. 

	
	
\begin{VHDLlisting}[tabsize=8]
library ieee;
use ieee.std_logic_1164.all;
use ieee.numeric_std.all;

entity uart_rx is
  generic(g_baudRate : integer := 115200;
          g_clkRate  : integer := 50000000);
  port(
      i_clk           : in    std_logic;
      i_rx            : in    std_logic;
      o_rx_data       : out   std_logic_vector(7 downto 0);
      o_rx_data_rdy   : out   std_logic
      );
end entity uart_rx;

architecture rtl of uart_rx is
  constant k_timeperbit   : real := real(1)/real(g_baudRate);
  constant k_clkperiod    : real := real(1)/real(g_clkRate);
  constant k_clksPerBit   : integer := integer(real(g_clkRate)/real(g_baudRate));
  constant k_clksPerBitd2 : integer := integer(real(k_clksPerBit)/real(2));
  constant k_TBits        : unsigned(3 downto 0) := to_unsigned(10,4);

  -- Outputs 
  signal f_rx_data     : std_logic_vector(7 downto 0) := (others => '0');
  signal f_rx_data_rdy : std_logic := '0';
  
  type sm_rxUart is (s_idle, s_cfmLow, s_rxStartBit, s_rx8bits, s_parity, s_stop, s_reset);
  signal f_cState : sm_rxUart := s_idle;
  
  signal f_clkCount : integer := 0;
  signal f_rxBits   : unsigned(3 downto 0) := (others => '0');
  
begin
  p_rxctrl : process(i_clk) 
  begin
    if rising_edge(i_clk) then
      o_rx_data_rdy <= f_rx_data_rdy;
      o_rx_data <= f_rx_data;
      case f_cState is
        when s_idle => 
          f_rx_data_rdy <= '0';
          f_rx_data <= (others => '0');
          if i_rx = '0' then
            f_cState <= s_cfmLow;
          end if;
        when s_cfmLow =>
          if f_clkCount = k_clksPerBitd2 then
            if i_rx = '0' then
              f_cState <= s_rxStartBit;
              f_clkCount <= 0;  
            else 
              f_cState <= s_idle; -- if rx line isn't still low during start bit.
              f_clkCount <= 0;
            end if;
          else
            f_clkCount <= f_clkCount + 1;
          end if;
        when s_rxStartBit =>
          if f_clkCount = k_clksPerBit then
            f_cState <= s_rx8bits;
            f_rx_data(to_integer(f_rxBits)) <= i_rx;
            f_rxBits <= f_rxBits + 1;
            f_clkCount <= 0;
          else 
            f_clkCount <= f_clkCount + 1;
          end if;        
        when s_rx8bits =>
          if f_rxBits = 8 then
            f_cState <= s_parity;
          elsif f_clkCount = k_clksPerBit then 
            f_clkCount <= 0;
            f_rx_data(to_integer(f_rxBits)) <= i_rx;
            f_rxBits <= f_rxBits + 1;
          else 
            f_clkCount <= f_clkCount + 1;
          end if;
        when s_parity =>
          if f_clkCount = k_clksPerBit then
            f_rxBits <= (others => '0');
            f_clkCount <= 0;
            if i_rx = '1' then
              f_cState <= s_stop;
            else
              f_cState <= s_idle;
            end if;
          else
            f_clkCount <= f_clkCount + 1;
          end if;
        when s_stop => 
          if f_clkCount = k_clksPerBit then
            f_clkCount <= 0;
            if i_rx = '1' then
              f_cState <= s_reset;
            else
              f_cState <= s_idle;
            end if;
          else
            f_clkCount <= f_clkCount + 1;
          end if;
        when s_reset =>
          if f_clkCount = k_clksPerBitd2 then
            f_rx_data_rdy <= '1';
            f_cState <= s_idle;
            f_rxBits <= (others => '0');
          else
            f_clkCount <= f_clkCount + 1;
          end if;
      end case; 
    end if;
  end process;
end rtl;
\end{VHDLlisting}

The five states of the state machine are idle, start, check start, acquire bits, and stop. As a side note state machines can use more resources. If you really wanted to squeeze out all the resources needed then you could rewrite the below code to not use a state machine and save some resources.The first state is s\_idle. In this state we are only concerned with whether or not the data has started. To start a UART transaction the data line is deasserted. We will use a falling edge detection circuit to determine is if the transaction has started. Once the falling edge is detected we move to s\_check.

In s\_check we count a half a bit's duration. The half a bit duration ensures we are sampling at the middle of a bit window. Since the clocks at the transmitter and receiver aren't synchronized we need to sample in the middle of the bit to gain the most reliable data. Once we have counted half a bits duration we can check that the start bit is low, if not we reset back to s\_idle, however if the start bit is low we go to s\_getbits.

In s\_getbits we will receive all the bits in a UART transaction. The number of bits in a transaction is variable but the most common is a byte or eight bits. So, after the start bit is confirmed to be low and we transition to this state we need to wait a full bit duration and sample the data line. We store the sampled data in a shift register. Once we have counted up the number of bits we continue on the the s\_stop state. In the s\_stop state we make sure the data line is high after the next but duration. Some configurations of the UART protocol have a parity bit. In this state is where you calculate the parity and check the integrity of the received data. 

\begin{VHDLlisting}[tabsize=8]
library ieee;
  use ieee.std_logic_1164.all;
  use ieee.numeric_std.all;
  use ieee.math_real.all;
  
entity uart_tx is
  generic(g_baudRate : integer := 115200;
          g_clkRate  : integer := 50000000);
  port(
      i_clk        : in    std_logic;
      i_sclr       : in    std_logic;
      i_tx_data    : in    std_logic_vector(7 downto 0);
      i_tx_data_dv : in    std_logic;
      o_tx         :   out std_logic;
      o_tx_busy    :   out std_logic
      );
end uart_tx;
architecture rtl of uart_tx is 
  constant k_timeperbit : real := real(1)/real(g_baudRate);
  constant k_clkperiod  : real := real(1)/real(g_clkRate);
  constant k_clksPerBit : integer := integer(real(g_clkRate)/real(g_baudRate)); 
  constant k_TBits      : unsigned(3 downto 0) := to_unsigned(10,4);
  
  type sm_uarttx_ctrl is (s_idle, s_txing, s_reset);
  signal f_cState : sm_uarttx_ctrl := s_idle;
  
  signal f_data2Tx : std_logic_vector(10 downto 0) := (others => '0');
  signal f_clkCount: integer := 0;
  signal f_bitCount: unsigned(3 downto 0) := (others => '0');
  
  -- Register Outs
  signal f_tx : std_logic := '1';
  signal f_tx_busy : std_logic := '0';
begin
  p_ctrl : process(i_clk) 
  begin
    if i_sclr = '1' then
        f_data2Tx <= (others => '0');
    elsif rising_edge(i_clk) then
      o_tx <= f_tx;
      o_tx_busy <= f_tx_busy;
      case f_cState is  
        when s_idle =>
          if i_tx_data_dv = '1' then
            f_data2Tx(8 downto 1) <= i_tx_data;
            f_data2Tx(10 downto 9) <= b"11";
            f_tx <= '0';
            f_tx_busy <= '1';
            f_cState <= s_txing;
            f_bitCount <= f_bitCount + 1;
          end if;
        when s_txing => 
          if f_clkCount = k_clksPerBit then
            f_clkCount <= 0;
            f_tx <= f_data2Tx(to_integer(f_bitCount));
            f_bitCount <= f_bitCount + 1;
            if f_bitCount = k_TBits then
              f_cState <= s_reset;
            end if;
          else
            f_clkCount <= f_clkCount + 1;
          end if;
        when s_reset => 
          f_tx <= '1';
          f_tx_busy <= '0';
          f_cState <= s_idle;
          f_bitCount <= (others => '0');
      end case;
    end if;
  end process;
end rtl;
\end{VHDLlisting}
	
The \ac{UART} transmitter operates in a similar fashion. The transmitter's job is is ensure the data is put on the line with the same timing the receiver expects. The data in loaded into the core in parallel but it output with a shift register. After a bit duration has passed we shift another bit out of the shift register until all the bits are sent. Then the transaction is ended.
	

\subsection{Serial Peripheral Interface (SPI)}
	<TODO Subsection Serial Peripheral Interface (SPI) : PROOF READ>

\ac{SPI} is commonly used for inter-chip communication where a chip is configured by a SPI master. The SPI master initiates all traffic between the SPI master and SPI slave. The slave chip only responses to commands from the master. The SPI bus consists of four wires; clock, chip select, input data, and output data. Since there is a clock we know this protocol is synchronous. The clock is driven by the master and the clock rate is the only way to change the data rate of the communications. In general the \ac{SPI} bus is capable of a higher data rate than the UART.

The rest of the lines generally work as follows but some chips use a slight variation to the descriptions below. General SPI code can be altered to meet specific chip standards. The datasheet for the chip you are trying to interface to should be consulted to ensure proper operation. 

The \ac(CS) line is used to enable communications with the chip. The reason for the name is that you can reuse the same clock and data lines but change only the CS lines for configuring multiple chips. In this way you save \ac{IO} pins on the \ac{FPGA}. The CS line is used to indicate that communication is starting. \ac{CS} can be active low or active high. Usually with active low signals an \emph{n} is appended to the name so if our CS line were active low it could be called CSn. 

The last lines of the SPI bus to discuss are the data lines. These lines are named \ac{MOSI} and \ac{MISO}. The names of course tell you the direction and who drives the pins. Sometimes if \ac{IO} pins are limited there is a shared data line. This is possible because it is understood that the master controls communication if the master is reading data from the slave then the master releases control of the data line in time for the slave to control the line. Care must be taken to ensure that there isn't contention on the line. If both slave and master drive the pin communication will fail. 	

Next we can look at an implementation of a \ac{SPI}-Master. Shown here

\begin{VHDLlisting}[tabsize=4]
-- spi_master.vhd

library ieee;
	use ieee.std_logic_1164.all;
	use ieee.numeric_std.all;
	
entity spi_master is
	generic(g_sclkrate  : integer;
			g_word_size : integer)
	port(i_clk      : in    std_logic;
		 i_rst      : in    std_logic;
		 
		 o_spi_clk  :   out std_logic;
		 o_spi_mosi :   out std_logic;
		 i_spi_miso : in    std_logic;
		 o_spi_ncs  :   out std_logic;
		
		 i_rd_nwr   : in    std_logic;
		 i_wrdataen : in    std_logic;
		 i_wr_data  : in    std_logic_vector(g_word_size-1 downto 0);
		
		 o_rd_data  :   out std_logic_vector(g_word_size-1 downto 0);
		 o_rd_dv    :   out std_logic;
		 o_busy     :   out std_logic	
	);
end entity spi_master;

architecture rtl of spi_master is
	signal f_clk_counter : unsigned(31 downto 0);
	signal f_o_sclk      : std_logic := '0';
	signal f_o_mosi      : std_logic := '0';
	signal f_o_ncs       : std_logic := '0';
	
	signal f_o_rd_dv     : std_logic := '0';
	signal f_rd_data     : std_logic_vector(g_word_size-1 downto 0);
	
	type sm_spi is (s_idle, s_transfer, s_end);
	signal s_spi         : sm_spi := s_idle;
	
	signal f_wr_data     : std_logic_vector(g_word_size-1 downto 0);
	signal f_rx_data     : std_logic_vector(g_word_size-1 downto 0);
	
	signal f_data_c      : unsigned(7 downto 0);
	signal f_data_rd_c   : unsigned(7 downto 0);
	
	
begin

	o_busy      <= f_busy;
	o_spi_clk   <= f_o_sclk; 
    o_spi_mosi  <= f_o_mosi;
    o_spi_ncs   <= f_o_ncs; 
	
	o_rd_data <= f_o_rd_dv;
	o_rd_dv   <= f_rd_data;

	p_gen_sclk : process(i_clk)
	begin
		if rising_edge(i_clk) then
			if i_rst = '1' then
				f_clk_counter <= (others => '0');
			else
				f_clk_counter <= f_clk_counter + 1;
				if f_clk_counter = g_sclkrate then
					f_o_sclk <= '0';
					f_clk_counter <= (others => '0');
				elsif f_clk_counter = g_sclkrate/2 then
					f_o_sclk <= '1';
				end if;
			end if;		
		end if;
	end process;

	p_launch_data : process(i_clk)
	begin
		if rising_edge(i_clk) then
			if i_rst = '1' then
				s_spi <= s_idle;
				f_busy <= '0';
				f_o_ncs <= '1';
			else
				ff_o_sclk <= f_o_sclk;
				case s_spi is
					when idle =>
						f_busy <= '0';
						f_o_ncs <= '1';
						f_o_rd_dv <= '1';
						if i_wrdataen = '1' then
							s_spi <= s_transfer;
							f_wr_data <= i_wr_data;
							f_busy <= '1';
							f_rd_nwr <= i_rd_nwr;
							f_data_c <= to_unsigned(f_wr_data'high,f_data_c'length);
							f_data_rd_c <= (others => '0');
						end if;
					when s_transfer => 
						-- Falling Edge of SCLK
						if ff_o_sclk = '1' and f_o_sclk = '0' then
							f_o_ncs <= '0';
							f_o_mosi <= f_wr_data(to_integer(f_data_c));
							f_data_c <= f_data_c - 1;
							if f_data_c = 0 then
								f_data_c <= to_unsigned(f_wr_data'high,f_data_c'length);
								s_spi <= s_end;
							end if;
						end if;					

						-- Rising Edge of SCLK
						if ff_o_sclk = '0' and f_o_sclk = '1' then
							f_rx_data(to_integer(f_data_rd_c)) <= i_spi_miso;
							f_data_rd_c <= f_data_rd_c + 1;
						end if;
					when s_end => 
						s_spi <= s_idle;
						if f_rd_nwr = '1' then
							-- Reading data from slave
							f_o_rd_dv <= '1';
							f_rd_data <= f_rx_data;
						end if;				
				end case;
			end if;
		end if;	
	end process;


end rtl;
\end{VHDLlisting}

This \ac{VHDL} uses a state machine to control when the data is sent and received. We see in the first process the \ac{SPI} clock is generated and sent out the \emph{SCLK} port. There may be certain circumstances where you don't want to clock always running. You can gate the output of the clock with the busy signal to ensure they clock is running only when it is needed. 

Looking closer at the \ac{VHDL} first we see we use two generics. The first generic is \emph{g\_sclkrate}, which is used to determine the \ac{SPI} clock speed. The \ac{SPI} clock speed is the clock speed of i\_clk divided by \emph{g\_sclkrate}. So if we have a $100$\ac{MHz} clock input and \emph{g\_sclkrate} is $4$ then \emph{o\_spi\_clk} will have a frequency of $25$\ac{MHz}. The generic helps us use the same spi\_master code for different chips that have different interfacing speeds. 

Next we have a the generic \emph{g\_word\_size}. The generic determines the number of bits being transferred in one transaction. The ports of our entity first list our clock and reset signals. Then we have the four \ac{IO} pins for the \ac{SPI} protocol. Then we have a flag that says whether we are writing data to the slave or reading data from the slave. Next we have the write data and write data valid. Followed by the read data and read data valid. And lastly we have a busy signal that lets the instantiating core know that we are currently busy spending out data. This is important because in one clock cycle the \ac{SPI}-Master is given \emph{g\_word\_size} data bits, but many clock cycles later, depending on the \ac{SPI} clock speed, the transfer is complete. 

Next we look in the architecture of the \ac{SPI}-Master and in the first process of the architecture we generate the \ac{SPI} clock, denoted \emph{f\_o\_sclk}. We use a counter that counts up to \emph{g\_sclkrate}. As the counter gets to half of \emph{g\_sclkrate} we deassert the clock then when we reach \emph{g\_sclkrate} we assert the clock. 

The next process launches data on the \ac{SPI} bus. First we wait in the state \emph{s\_idle} until write data valid is asserted. Whether we are reading data from the slave or writing data to the slave we have to start the transaction by have the master enable the \ac{CS} line. The amount of data includes the amount of data that will be read back from the slave as well. Usually the \ac{SPI} master writes an address and the rest of the data is the data at the specified address.

Once data is loaded into the \ac{SPI}-Master the state machine transitions to the \emph{s\_transfer} state. In this state we first look for the falling edge of the \ac{SPI} clock, at which, we assign data to the \ac{MOSI} line starting with the \ac{MSB}. We wait for falling edges over and over again until we have launched all the data. Then we transition to the \emph{s\_end} state. But also in the \emph{s\_transfer} state we look for the rising edge of the \ac{SPI} clock. Here we register what is coming in on the \ac{MISO} line. 

Once we are in the \emph{s\_end} state we go back to \emph{s\_idle}, but before we do if we were performing a read operation then we set the outputs providing valid data at the output ports. 

Now we should go back and discuss the reasoning for using the rising edge and falling edge of the \ac{SPI} clock. We launch the data on the falling edge of the \ac{SPI} clock because we want to have the maximum possible time for the data to be stable when the receiver of the data samples it at the rising edge. Just like when we wait for the rising edge of the clock to register the input we want the data to be stable. 

\subsection{Inter-Integrated Circuit (I2C) Bus}
	<TODO Subsection Inter-Integrated Circuit (I2C) Bus : PROOF READ>

The \ac{I2C} bus communication protocol is similar to the \ac{SPI} interface in that it is used for inter chip communication. The major advantage to \ac{I2C} is that is uses half the lines as \ac{SPI}. You may be thinking who cares about the number of \ac{IO} pins a chip takes but this can be an issue with interfacing to a few peripherals, especially when a few of them are high speed \ac{LVDS}. In this case saving a few \ac{IO} pins per chip is necessary to ensure \ac{IO} pins are available for a high speed \ac{ADC} or \ac{DDR} memory chip. 	
	
In the \ac{I2C} bus there are two wires, clock and data. And similar to \ac{SPI} each chip that you interface to will have a different clock speed and signal timing associated with the chip. For the \ac{I2C} example we show what happens when we know the chip has 32 bits per transaction. 	
	
In the \ac{I2C} protocol data is sent in eight bit chunks. So to interface to the 32 bit transaction size we break the transaction down into four - eight bit sections and ensure we have an acknowledge bit at the end of each eight bit section. 
	
We can look at some example code to see how this is done in \ac{VHDL}.	

\begin{VHDLlisting}[tabsize=4]
-- i2c_write.vhd

library ieee;
    use ieee.std_logic_1164.all;
    use ieee.numeric_std.all;


entity i2c_write is
    port(i_clk        : in    std_logic;
         i_rst        : in    std_logic;
         -- I2C Ports 
         b_sda        : inout std_logic;
         o_scl        :   out std_logic;
         -- App 
         i_data       : in    std_logic_vector(31 downto 0);
         i_dv         : in    std_logic;
         o_busy       :   out std_logic
    );
end entity i2c_write;

architecture rtl of i2c_write is 
    constant k_start_cond     : integer := 144;
	
    constant k_clks_per_scl_h : integer := 500;
    constant k_clks_per_scl   : integer := k_clks_per_scl_h*2;
	constant k_end_cond       : integer := 144 + k_clks_per_scl_h;
	
    -- Input Regs
    signal f_data    : std_logic_vector(31 downto 0) := (others => '0');
    	
    -- Output Regs
    signal f_busy    : std_logic := '0';
    signal f_sda     : std_logic := '1';
    signal f_scl     : std_logic := '1';

	constant k_n_data_frames : integer := 3;
    signal f_cc      : unsigned(11 downto 0) := (others => '0');
    signal f_os      : unsigned(7 downto 0) := (others => '0');
	signal f_ndframes: unsigned(3 downto 0) := (others => '0');
    signal f_ack_bit : std_logic := '1';
	signal f_done    : std_logic := '0';

    type sm_states is (s_idle, s_init_frame, s_reset_start_cond, s_ack_bit, s_write_data_frame, s_end);
    signal s_i2c  : sm_states := s_idle;
begin

    o_busy <= f_busy;

    b_sda <= f_sda;
    o_scl <= f_scl;

    p_i2c : process(i_clk)
    begin
        if rising_edge(i_clk) then
            if i_rst = '1' then
                s_i2c <= s_idle;
				f_sda <= '1';
                f_scl <= '1';
				f_done <= '0';
            else
                case s_i2c is
                    when s_idle => 
                        f_sda <= '1';a
                        f_scl <= '1';
						f_done <= '0';
						f_ack_bit <= '1';
						f_os <= to_unsigned(31,8);
                        if i_dv = '1' then
                            f_data <= i_data;
                            s_i2c <= s_init_frame;
                            f_sda <= '0'; -- Assumes point to point I2C
                            f_busy <= '1';
                        end if;
                    when s_init_frame => 
                        -- Start Condition deassert SDA
                        f_cc <= f_cc + 1;
                        if f_cc = k_start_cond then
                            f_cc <= (others => '0');
                            f_scl <= '0';
                            f_sda <= f_data(to_integer(f_os));
                            s_i2c <= s_write_data_frame;
                        end if;
					when s_reset_start_cond => 
						f_cc <= f_cc + 1;
						f_sda <= '1';
                        if f_cc = k_end_cond then
                            f_cc <= (others => '0');
							s_i2c <= s_init_frame;
                            f_sda <= '0';
                        end if;
                    when s_ack_bit => 
                        f_cc <= f_cc + 1;
                        if f_cc = k_clks_per_scl_h then
                            f_scl <= '1';
                            f_ack_bit <= b_sda;
                        end if; 
                        if f_cc = k_clks_per_scl then
							f_scl <= '0';
							f_cc <= (others => '0');
                            if f_ack_bit = '0' then
                                -- Success
								if f_done = '0' then
									s_i2c <= s_write_data_frame;
									f_sda <= f_data(to_integer(f_os));
								else
									f_sda <= '0';
									s_i2c <= s_end;
								end if;
                            else
                                -- Fail -- Reset
                                s_i2c <= s_idle;
                            end if;
                        end if;
                    when s_write_data_frame => 
                        f_cc <= f_cc + 1;
						f_ack_bit <= '1';
						
                        if f_cc = k_clks_per_scl_h then
                            f_scl <= '1';
                            f_os <= f_os - 1;
                        end if; 
						
                        if f_cc = k_clks_per_scl then
							f_scl <= '0';
                            f_cc <= (others => '0');
							if f_os = 23 then
								f_sda <= 'Z';
								s_i2c <= s_ack_bit;
								f_done <= '0';
                            elsif f_os = 15 then
								f_sda <= 'Z';
								s_i2c <= s_ack_bit;
								f_done <= '0';
                            elsif f_os = 7 then
								f_sda <= 'Z';
								s_i2c <= s_ack_bit;
								f_done <= '0';
                            elsif f_done = '1' then
								f_sda <= 'Z';
                                s_i2c <= s_ack_bit;
							else
								f_sda <= f_data(to_integer(f_os));
								if f_os = 0 then
									f_done <= '1';
								end if;
                            end if;
                        end if;
					when s_end =>
						f_cc <= f_cc + 1;
						if f_cc = k_clks_per_scl_h then
                            f_scl <= '1';
                        end if; 
						
                        if f_cc = k_end_cond then
                            f_cc <= (others => '0');
							f_sda <= '1';
							s_i2c <= s_idle;
                        end if;
                end case;
            end if;
        end if;
    end process;
end rtl;
\end{VHDLlisting}

For this implementation we know that we are going to send $32$ bits so we don't need a generic to specify the data width. As far as the other inputs of the block we have the system clock as well as the reset line. The two \ac{I2C} lines clock and data. And finally the data the send with a data valid flag and a busy signal. 

For the \ac{I2C} implementation we again start off the state \emph{s\_idle}. Once $32$ bits are provided to the core with the data valid flag we move to the \emph{s\_init\_frame}. In this state we set the timing between the deassertion of SDA and the falling edge of the clock. This timing is specified in the data sheet for the chip that you will be interfacing to. 

Next we start the data frame in \emph{s\_write\_data\_frame}. Where we set the current bit of \emph{f\_data} to the output on the falling edge. Then after each eight bit section we jump to \emph{s\_ack\_bit} which ensure data is being sent successfully to the \ac{I2C}-Slave.

Finally after sending all $32$ bits the done flag is asserted and we end the transaction. 
	
	
\subsection{Texas Instruments universal Parallel Port (uPP)}
	<TODO Subsection Texas Instruments universal Parallel Port (uPP) : PROOF READ>
	
A \ac{TI}-\ac{DSP} is an embedded processor that specializes in operations specific to signal processing. What that means is that the \ac{MAC} structure is used often in \ac{FIR} filter design, calculating \ac{FFT}s, and even some matrix multiplication calculations. The \ac{DSP} is equipped with with the ability to perform parallel \ac{MAC} operations. Depending on the specific processor you can expect eight parallel \ac{MAC} operations. 

The \ac{DSP} a very powerful processing platform with the added hardware to the ease of development in the C programming language. Because of these reasons is it popular to interface a \ac{DSP} to an \ac{FPGA}. With both chips on a \ac{PCB} you get the best of both worlds. 

To leverage having an \ac{FPGA} interface to a \ac{TI} - \ac{DSP} we need to communicate data between the two chips quickly. \ac{TI} has dedicated pins on their part specifically for this task. The \ac{TI} \ac{uPP} bus offers $16$ bit width data bus, with a start, clock, enable, and wait pins to control data flow. The \ac{TI} part also offers to send and receive data at \ac{DDR}. 

First to look at how the \ac{uPP} bus works we will look at the transmitter from the \ac{FPGA} to the \ac{TI}-\ac{DSP}. The \ac{VHDL} that defines the transmitter include to generics. The first generic is the \ac{DMA} size. For the \ac{FPGA} the \ac{DMA} size is the transfer length. The second generic is the \ac{uPP} clock rate ratio, which defines the clock speed for the transfers. 

Next we will define the \ac{IO} to the component. First the the clock and the reset line for the core. Next are the control lines for the \ac{uPP} bus, clock, start, enable, and wait. We also have the $16$ bit data bus. Finally we have the application signals that are used for the \ac{IO} between the other \ac{FPGA} logic and this core. 

After the signal declarations we start the architecture block with defining the outputs. We set the outputs first so that if you are looking at the block later you know what is being calculated. After we assign the full flag for the \ac{FIFO} we set the four \ac{uPP} control lines. 

We now move on to the process where we transmit data. We first reset some registers, then we define a state machine. The state machine defines two states, idle and transfer. While in idle we wait until the input \ac{FIFO} is not empty. If the input \ac{FIFO} isn't empty then we have data to transfer. Once we detect the empty flag is zero we read the data and move to the transfer state. 

Once in the transfer state we first assign the data out of the \ac{FIFO} when the \ac{FIFO} data valid is high. Next when the f\_upp\_clk\_strobe is high we are at a rising edge of the upp\_clk. We generate this strobe in a process later. For now we just assume the strobe is high for a clock cycle when the upp\_clk has a rising edge. 

If the clock strobe is high we then check the wait signal from the receiver. The wait signal is how the receiver stalls the transmitter. So if the wait signal is low we enable the \ac{uPP} transfer, read from the \ac{FIFO}, and increment the transfer count. Then if the transfer count is zero then we assert the \ac{uPP} start signal. If we are at the \ac{DMA} transfer size then we have completed the transfer then we go back to the idle state. 

However, if \ac{uPP} wait is asserted or if the clock strobe isn't high then we disable reading from the \ac{FIFO}. The only time we disable the \ac{uPP} transfer is when \ac{uPP} wait is asserted since the transfer stalls. If the clock strobe is not high that means we do not want to read from the \ac{FIFO} but we are still within a clock cycle of the \ac{uPP} clock. 

The next instantiation is the \ac{FIFO} itself with the port map. Which is then followed by a process that generates the \ac{uPP} clock, and the \ac{uPP} clock strobe. First we have a reset the clears the clock counter and holds the clock low. We leave it to the user logic to hold the transmitter in reset when not in use. 

When the core is not held in reset we first register the clock so that we can detect the rising edge. Next we count clock cycles so that we can toggle the \ac{uPP} clock based on the clock ratio specified by the instantiation. Finally we use the registering of the \ac{uPP} clock to detect the rising edge of the \ac{uPP} clock and assert and de-assert the strobe used in the first process accordingly.

\begin{VHDLlisting}[tabsize=2]
-- upp_tx.vhd

library ieee;
	use ieee.std_logic_1164.all;
	use ieee.numeric_std.all;
	
entity upp_tx is
	generic(g_dma_size      : integer := 4096;
			g_upp_clk_ratio : integer := 200)
	port(i_clk           : in    std_logic;
		 -- Used async | release sync
	     i_rst           : in    std_logic;
		 
		 -- uPP IO
		 o_upp_clk       :   out std_logic; 
		 o_upp_start     :   out std_logic;
		 o_upp_enable    :   out std_logic;
		 i_upp_wait      : in    std_logic;
		 
		 o_upp_data      :   out std_logic_vector(15 downto 0);
		 
		 -- App Interface
		 i_tx_data       : in    std_logic_vector(15 downto 0);
		 i_tx_data_dv    : in    std_logic	
		 o_tx_fifo_full  :   out std_logic	
	);
end entity upp_tx;

architecture rtl of upp_tx is
	constant k_upp_clk_ratio_half : integer := g_upp_clk_ratio/2;
	signal w_tx_fifo_full : std_logic;
	
	signal f_clk_count      : unsigned(log2(g_upp_clk_ratio)-1 downto 0);
	signal f_tx_word_count  : unsigned(log2(g_dma_size)-1 downto 0);
	signal f_upp_clk        : std_logic;
	signal ff_upp_clk       : std_logic;
	signal f_upp_clk_strobe : std_logic;
	signal f_upp_start      : std_logic;
	signal f_upp_enable     : std_logic;
	signal f_upp_data       : std_logic_vector(15 downto 0);
	                        
	signal f_fifo_rden      : std_logic;
	signal w_fifo_dout      : std_logic_vector(15 downto 0);
	signal w_fifo_rddv      : std_logic;
	signal w_fifo_empty     : std_logic;
	
	type sm_upptx is (s_idle, s_transfer, s_end);
	signal s_curr_state : sm_upptx := s_idle;

begin

	o_tx_fifo_full <= w_tx_fifo_full;
	o_upp_clk      <= f_upp_clk;
	o_upp_start    <= f_upp_start ;
	o_upp_enable   <= f_upp_enable;
	o_upp_data     <= f_upp_data;
	
	p_tx_data : process(i_clk)
	begin
		if rising_edge(i_clk) then
			if i_rst = '1' then
				s_curr_state <= s_idle;
				f_upp_start <= '0';
				f_upp_enable <= '0';
				f_tx_word_count <= (others => '0');
			else
				case s_curr_state is
					when s_idle => 
						f_upp_start <= '0';
						f_upp_enable <= '0';
						f_tx_word_count <= (others => '0');
						
						if w_fifo_empty = '0' then
							s_curr_state <= s_transfer;
							f_fifo_rden <= '1';
						end if;					
					when s_transfer => 
						if w_fifo_rddv = '1' then
							f_upp_data <= w_fifo_dout;
						end if;
						
						if f_upp_clk_strobe = '1' then
							-- Rising Edge of upp clk
							if i_upp_wait = '0' then
								f_upp_enable <= '1';
								f_fifo_rden <= '1';
								f_tx_word_count <= f_tx_word_count + 1;
							
								case f_tx_word_count is
									when 0 => 
										f_upp_start <= '1';
									when g_dma_size => 
										s_curr_state <= s_end;
									when others =>
										f_upp_start <= '0';
								end case;			
							else
								-- Transfer is stalled
								f_upp_enable <= '0';
								f_fifo_rden <= '0';
							end if;							
						else
							f_fifo_rden <= '0';
						end if;
					
					when s_end => 
				
				end case;
			end if;
		end if;
	end process;
	
	
	u_tx_fifo : entity work.tx_fifo
	port map(wrclk => i_clk, 
			 wren  => i_tx_data_dv,
			 din   => i_tx_data,
			 full  => w_tx_fifo_full,
	
			 rdclk => f_upp_clk,
			 rden  => f_fifo_rden,
			 dout  => w_fifo_dout,
			 rddv  => w_fifo_rddv,
			 empty => w_fifo_empty	
	);
	
	p_gen_upp_clk : process(i_clk)
	begin
		if rising_edge(i_clk) then
			if i_rst = '1' then
				f_clk_count <= (others => '0');
				f_upp_clk <= '0';
			else
				ff_upp_clk <= f_upp_clk;
			
				f_clk_count <= f_clk_count + 1;
				if f_clk_count = k_upp_clk_ratio_half then
					-- Toggle UPP Clk
					f_upp_clk <= not f_clk_count;
					f_clk_count <= (others => '0');
				end if;
				
				if ff_upp_clk = '0' and f_upp_clk = '1' then
					-- Generate upp clock strobe
					f_upp_clk_strobe <= '1';
				else
					f_upp_clk_strobe <= '0';
				end if;
				
			end if;
		end if;	
	end process;
\end{VHDLlisting}

Now that we have seen how the \ac{FPGA} transmits data we will transition to how the \ac{FPGA} received data from the \ac{TI}-\ac{DSP}. Once again in the entity declaration we have a clock and a reset. We next have the \ac{uPP} \ac{IO} lines. These lines are just like the transmitter's \ac{IO} lines in reverse direction. Finally we have the application logic lines. 

In the architecture declaration we have the internal signal definitions. These include the the internal registers for the data received by the core, the busy signal that is an output to the other \ac{FPGA} logic, and a \ac{DMA} count to ensure we have all the data we expect. We also have a signal that is a wire that is the full flag from the receive \ac{FIFO}.

We move past the begin keyword to the instantiated code. First we see the output declarations. As part of our coding style and guidelines we assign the outputs right after the begin to ensure other reviewers of our code can find the results for our core. Immediately after we assign some of the outputs we see an instantiation of a \ac{FIFO}. Here we need a dual clock \ac{FIFO} since the \ac{uPP} clock is different than the system clock. The dual clock \ac{FIFO} defines the reset of the outputs of the core. 

The port map of the \ac{FIFO} is separated into two section the write section and read section. The write section is controlled by a process which we will discuss later. The read section is of the \ac{FIFO} is controlled by logic external to this core. The read clock is the system clock and the read enable is an input to the core. The read data, read data valid, and empty signals are all outputs to inform the external user logic. 

We move to the receive data process. In this process we first have a synchronous reset. In the reset we clear the \ac{DMA} transfer count the receive data valid and \ac{uPP} busy signal. While not in reset we resister the \ac{uPP} start signal. Then we edge detect the \ac{uPP} start signal and if detect the rising edge of the start signal we assert the busy signal. Next we check the \ac{DMA} transfer count and if we are at the end of the transfer we de-assert the busy signal. 

The last if statement registers data of the \ac{uPP} data bus. If the transfer is enabled then we increment the \ac{DMA} count, register the data bus, and assert the write enable line for the \ac{FIFO}. If \ac{uPP} enable is low then we de-assert write enable. This process fills up the \ac{FIFO} and the external logic pulls data from the \ac{FIFO} but the \ac{FIFO} should be configured to be to be able to support the \ac{DMA} length. 

\begin{VHDLlisting}[tabsize=2]
-- upp_rx.vhd

library ieee;
	use ieee.std_logic_1164.all;
	use ieee.numeric_std.all;
	
entity upp_rx is
	generic(g_dma_size : integer := 4096)
	port(i_clk           : in    std_logic;
	     i_rst           : in    std_logic;
		 
		 -- uPP IO
		 i_upp_clk       : in    std_logic; 
		 i_upp_start     : in    std_logic;
		 i_upp_enable    : in    std_logic;
		 o_upp_wait      :   out std_logic;
		 
		 i_upp_data      : in    std_logic_vector(15 downto 0);
		 
		 -- App interface
		 o_upp_busy      :   out std_logic;
		 o_rx_data_empty :   out std_logic;
		 i_rx_data_rden  : in    std_logic;
		 o_rx_data_dv    :   out std_logic;
		 o_rx_data       :   out std_logic_vector(15 downto 0)	
	);
end entity upp_rx;


architecture rtl of upp_rx is
	
	signal f_rx_data_dv   : std_logic;
	signal f_rx_data      : std_logic_vector(15 downto 0);
	signal w_rx_fifo_full : std_logic;
	signal f_upp_busy     : std_logic;
	
	signal f_dma_count    : unsigned(log2(g_dma_size)-1 downto 0);
	
begin

	o_upp_wait <= w_rx_fifo_full;
	o_upp_busy <= f_upp_busy; 

	u_rx_fifo : entity work.rx_fifo
	port map(i_rst => i_rst,
			 wrclk => i_upp_clk, 
			 wren  => f_rx_fifo_wren,
			 din   => f_rx_fifo_data,
			 full  => w_rx_fifo_full,
	
			 rdclk => i_clk,
			 rden  => i_rx_data_rden,
			 dout  => o_rx_data,
			 rddv  => o_rx_data_dv,
			 empty => o_rx_data_empty	
	);

	p_rx_data : process(i_upp_clk, i_arst)
	begin
		if rising_edge(i_upp_clk) then
			if i_rst = '1' then
				f_dma_count <= (others => '0');
				f_rx_data_dv <= '0';
				f_upp_busy <= '0';
			else
				f_upp_start <= i_upp_start;
				if f_upp_start = '0' and i_upp_start = '1' then	
					-- Start DMA transfer
					f_upp_busy <= '1';
				end if;
				
				if f_dma_count = g_dma_size-1 then
					-- DMA Transfer Complete
					f_dma_count <= (others => '0');
					f_upp_busy <= '0';
				end if;
			
				if i_upp_enable = '1' then
					-- Load uPP data into rx fifo
					f_dma_count <= f_dma_count + 1;
					f_rx_fifo_data <= i_upp_data;
					f_rx_fifo_wren <= '1';
				else
					-- uPP Transfer Stalled.
					f_rx_fifo_wren <= '0';
				end if;
			end if;
		end if;
	end process;	
end rtl;
\end{VHDLlisting}

\section{Data Processing Projects}
	<TODO Section Data Processing Projects : PROOF READ>
	
We now transition to \ac{VHDL} examples that are processing data. The first example calculates the multiplication of two matrices. Matrix multiplication has applications in signal processing and machine learning. We next look at the knapsack problem. In this problem we are looking to maximize the reward which is applicable to \ac{FPGA} development itself. Finally we present some signal processing examples, a \ac{FIR} filter and an \ac{NCO}, both of which are used regularly in \ac{DSP} applications. 

\subsection{Strassen's Matrix Multiplication}
	<TODO Subsection Strassen's Matrix Multiplication : PROOF READ>
	
Matrix multiplication is a common operation is \ac{DSP}, machine learning, and graphics processing applications. Here we look at how to make matrix multiplication more efficient. Matrix multiplication in general is $\mathbf{O}(n^3)$. To make the algorithm more efficient in terms of \ac{FPGA} resource we will use the Strassen matrix multiplication. 

In essence the Strassen matrix multiplication reduces the number of multiplies required. Since the dedicated multiply resources are very advantageous we look to minimize the usage if we can. To multiply very large matrices we need to use the dedicated multiply blocks efficiently. The Strassen matrix multiply does this optimally for reducing input matrices to a $2 \times 2$.

We are going to calculate the matrix multiplication $\mathbf{A}\mathbf{B}=\mathbf{C}$. We assume that $\mathbf{A}$ and $\mathbf{B}$ have specific dimensions, namely, the number of rows and columns equate to a power of 2. If the matrices you want to multiply are not powers of $2$ you can zero pad each dimension. Each dimension is zero padded to $2^k$ where $k$ is an integer. If both $\mathbf{A}$ and $\mathbf{B}$ have dimensions that are $2^k$ then the matrix multiplication can be recursively reduced to a number of $2 \times 2$ matrices. 

To calculate a $2 \times 2$ matrix multiply we turn to the Strassen algorithm. The input to the Strassen algorithm are the two matrices $\mathbf{A}$ and $\mathbf{B}$. Each matrix is indexed into by a row and column;

\begin{equation}
\mathbf{A}=
\begin{bmatrix}
  a_{1,1} & a_{1,2} \\
  a_{2,1} & a_{2,2} \\
 \end{bmatrix},~~~~~~~~~~~~~
\mathbf{B}=
\begin{bmatrix}
  b_{1,1} & b_{1,2} \\
  b_{2,1} & b_{2,2} \\
 \end{bmatrix}.
\end{equation}

Then the Strassen algorithm defines some intermediate values as:

\begin{eqnarray}
\label{eq:capp}
P &=& (a_{1,1} + a_{2,2})(b_{1,1} + b_{2,2})\\
Q &=& (a_{2,1} + a_{2,2})b_{1,1}          \\
R &=& a_{1,1}(b_{1,2} - b_{2,2})\\
S &=& a_{2,2}(b_{2,1} - b_{1,1})\\
T &=& (a_{1,1} + a_{1,2})b_{2,2}  \\        
U &=& (a_{2,1} - a_{1,1})(b_{1,1} + b_{1,2})\\
V &=& (a_{1,2} - a_{2,2})(b_{2,1} + b_{2,2})
\label{eq:capv}
\end{eqnarray}

And finally with the intermediate values defined the elements of the result matrix, $\mathbf{C}$, are calculated,

\begin{eqnarray}
c_{1,1} &=& P + S - T + V\\
c_{1,2} &=& R + T\\
c_{2,1} &=& Q + S\\
c_{2,2} &=& P + R - Q + U
\label{eq:couts}
\end{eqnarray}

Now that the algorithm is defined we will spend the reset of this section discussing how we can implement the Strassen Matrix Multiplication algorithm in \ac{VHDL}. Before we start implementing the algorithm we first must think about what the expected data rate is going to be. Since the Strassen algorithm helps with reducing the number of multiplies required from eight to seven we won't see a huge savings in \ac{FPGA} resources with small matrix multiplies. Our target implementation is going to be handling very large matrices that will be parted down to many $2 \times 2$ matrix multiplication. For this implementation we expect to instantiate this block many times so that we can handle all the data as it is streaming in. 

So for this implementation we want to be able to handle the two input matrices in parallel and data can be valid every clock cycle. So after the latency of the core is over we will have resulting $\mathbf{C}$ matrices every clock cycle. This implementation will be able to handle the largest data rates possible. 

To get started with this implementation we start with an entity declaration,

\begin{VHDLlisting}[tabsize=2]
-- matrix_multiply_strassen.vhd

library ieee;
	use ieee.std_logic_1164.all;
	use ieee.numeric_std.all;
	
entity matrix_multiply_strassen is
	generic(g_bitwidth : integer)
	port(i_clk    : in    std_logic;
		 i_rst    : in    std_logic;
		 
		 i_dv     : in    std_logic;
		 i_a_11   : in    signed(g_bitwidth-1 downto 0);
		 i_a_12   : in    signed(g_bitwidth-1 downto 0);
		 i_a_21   : in    signed(g_bitwidth-1 downto 0);
		 i_a_22   : in    signed(g_bitwidth-1 downto 0);
		
		 i_b_11   : in    signed(g_bitwidth-1 downto 0);
		 i_b_12   : in    signed(g_bitwidth-1 downto 0);
		 i_b_21   : in    signed(g_bitwidth-1 downto 0);
		 i_b_22   : in    signed(g_bitwidth-1 downto 0);
		 
		 o_dv     :   out std_logic;
		 o_c_11   :   out signed(2*g_bitwidth-1 downto 0);
		 o_c_12   :   out signed(2*g_bitwidth-1 downto 0);
		 o_c_21   :   out signed(2*g_bitwidth-1 downto 0);
		 o_c_22   :   out signed(2*g_bitwidth-1 downto 0)	
	);
end entity matrix_multiply_strassen;
\end{VHDLlisting}

The entity declaration for \emph{matrix\_multiply\_strassen}has one generic which sets the bit width for the elements of the matrices. The port definitions start with the clock and reset, then the two matrix inputs, with four elements each along with a single data valid line for both matrices. This interface puts the responsibility of ensuring the data is ready and valid in a single clock cycle on the instantiating core. The last set of signals are the output, the four elements of the matrix $\mathbf{C}$ and a data valid line. 

Now we will move on the architecture block of the core. First we define a shift register for the data valid line. The shift register keeps track of the pipeline stage. If the data was valid in a clock cycle then the data calculation and the valid is registered. In this way the valid flag follows along with the data. 

\begin{VHDLlisting}[tabsize=2]
signal f_i_dv : std_logic_vector(5 downto 0);

-- Stage 1 : Register Inputs
signal f_a_11 : signed(g_bitwidth-1 downto 0);
signal f_a_12 : signed(g_bitwidth-1 downto 0);
signal f_a_21 : signed(g_bitwidth-1 downto 0);
signal f_a_22 : signed(g_bitwidth-1 downto 0);
signal f_b_11 : signed(g_bitwidth-1 downto 0);
signal f_b_12 : signed(g_bitwidth-1 downto 0);
signal f_b_21 : signed(g_bitwidth-1 downto 0);
signal f_b_22 : signed(g_bitwidth-1 downto 0);
\end{VHDLlisting}

We now look at the signals that are going to be used in each stage of calculation. In the signal declaration section we organize and comment the groups of signals so that as our \ac{VHDL} files get larger we can orgainize the file in a logical order which helps keeps the logic more manageable. 

The first stage of calculation is registering the inputs under the data valid line. It is a good habit to register the inputs and outputs of any block that you write. This way when you instantiate the core you won't have to worry about if another core registers it's outputs. If you are using other people's code that doesn't register outputs and if you don't register inputs then its likely that those paths will be the first to fail timing. 

If the code you are interfacing to does register the outputs and you register the inputs the synthesizer is reliable enough to remove one register. This practice is very helpful when you are trying to achieve the highest data rate possible. Once we have made this core fully pipelined then the only other way to increase the data rate is to increase the clock speed. To run this core at the highest possible clock speed we will be glad to have the extra registers in the code. 

The process that registers the inputs is shown here. 

\begin{VHDLlisting}[tabsize=2]
p_s1 : process(i_clk)
	begin
		if rising_edge(i_clk) then
			if i_rst = '1' then
				f_i_dv <= (others => '0');
			else
				f_i_dv <= f_i_dv(f_i_dv'high-1 downto 0) & i_dv;
				
				if i_dv = '1' then
					f_a_11  <= i_a_11;
				    f_a_12  <= i_a_12;
				    f_a_21  <= i_a_21;
				    f_a_22  <= i_a_22;
				    f_b_11  <= i_b_11;
				    f_b_12  <= i_b_12;
				    f_b_21  <= i_b_21;
				    f_b_22  <= i_b_22;
				end if;
			end if;
		end if;
	end process;
\end{VHDLlisting}

Under the rising edge clause of this process we first check the reset line. If the reset line is asserted we only need to clear the data valid shift register. Clearing these registers ensure no data is output from the core. Next we shift in the data valid input into our shift register. This signal is what will be used in the subsequent processes to know when data is valid. Finally we register all the inputs under the single data valid. 

\begin{VHDLlisting}[tabsize=2]
-- Stage 2 : Term Calculations
signal ff_p_t1 : signed(g_bitwidth-1 downto 0);
signal ff_p_t2 : signed(g_bitwidth-1 downto 0);
signal ff_q_t1 : signed(g_bitwidth-1 downto 0);
signal ff_q_t2 : signed(g_bitwidth-1 downto 0);
signal ff_r_t1 : signed(g_bitwidth-1 downto 0);
signal ff_r_t2 : signed(g_bitwidth-1 downto 0);
signal ff_s_t1 : signed(g_bitwidth-1 downto 0);
signal ff_s_t2 : signed(g_bitwidth-1 downto 0);
signal ff_t_t1 : signed(g_bitwidth-1 downto 0);
signal ff_t_t2 : signed(g_bitwidth-1 downto 0);
signal ff_u_t1 : signed(g_bitwidth-1 downto 0);
signal ff_u_t2 : signed(g_bitwidth-1 downto 0);
signal ff_v_t1 : signed(g_bitwidth-1 downto 0);
signal ff_v_t2 : signed(g_bitwidth-1 downto 0);
\end{VHDLlisting} 

In stage two we declare signals that are used in the addition or subtraction calculations in \eq{capp}-\eq{capv}. In the signal names we use the notation $t1$ or $t2$ to indicate the two operands of the multiplication in each equation. 

\begin{VHDLlisting}[tabsize=2]
p_s2 : process(i_clk)
	begin
		if rising_edge(i_clk) then
			if f_i_dv(0) = '1' then
				ff_p_t1 <= f_a_11 + f_a_22;
				ff_p_t2 <= f_b_11 + f_b_22;
				ff_q_t1 <= f_a_21 + f_a_22;
				ff_q_t2 <= f_b_11;
				ff_r_t1 <= f_a_11;
				ff_r_t2 <= f_b_12 - f_b_22;
				ff_s_t1 <= f_a_22;
				ff_s_t2 <= f_b_21 - f_b_11;
				ff_t_t1 <= f_a_11 + f_a_12;
				ff_t_t2 <= f_b_22;
				ff_u_t1 <= f_a_21 - f_a_11;
				ff_u_t2 <= f_b_11 + f_b_12;
				ff_v_t1 <= f_a_12 - f_a_22;
				ff_v_t2	<= f_b_21 + f_b_22;
			end if;
		end if;
	end process;
\end{VHDLlisting}

For each term we follow \eq{capp}-\eq{capv} and assign each term. To ensure all the data with the associated input matrices stay aligned we need to register intermediate values even if no calculation is done. For example, the second term of $Q$ is just $b_{1,1}$ so we register that value because we will need it in the next clock cycle and another input matrix may be in the pipeline next clock cycle overwriting our data. So we need to register it in this process.

\begin{VHDLlisting}[tabsize=2]
-- Stage 3 : P,Q,R,S,T,U,V Calculations
signal fff_p   : signed(2*g_bitwidth-1 downto 0);
signal fff_q   : signed(2*g_bitwidth-1 downto 0);
signal fff_r   : signed(2*g_bitwidth-1 downto 0);
signal fff_s   : signed(2*g_bitwidth-1 downto 0);
signal fff_t   : signed(2*g_bitwidth-1 downto 0);
signal fff_u   : signed(2*g_bitwidth-1 downto 0);
signal fff_v   : signed(2*g_bitwidth-1 downto 0);
\end{VHDLlisting}

The third stage calculates the intermediate variables in the Strassen algorithm. The size of the result of a multiply operation is bigger. Just like when we multiply $9 \times 9 = 81$ we need more bits to handle the range of the possible results. 

\begin{VHDLlisting}[tabsize=2]
p_s3 : process(i_clk)
begin
	if rising_edge(i_clk) then
		if f_i_dv(1) = '1' then
			fff_p <= ff_p_t1 * ff_p_t2;
			fff_q <= ff_q_t1 * ff_q_t2;
			fff_r <= ff_r_t1 * ff_r_t2;
			fff_s <= ff_s_t1 * ff_s_t2;
			fff_t <= ff_t_t1 * ff_t_t2;
			fff_u <= ff_u_t1 * ff_u_t2;
			fff_v <= ff_v_t1 * ff_v_t2;
		end if;
	end if;
end process;
\end{VHDLlisting}

The process that calculates the multiplication is shown here. Under the third pipeline stage data valid we calculate all the intermediate values $P$ - $V$. 

\begin{VHDLlisting}[tabsize=2]
-- Stage 4 : C term calculations
signal f4_c11_t1 : signed(2*g_bitwidth-1 downto 0);
signal f4_c11_t2 : signed(2*g_bitwidth-1 downto 0);
signal f4_c12_t1 : signed(2*g_bitwidth-1 downto 0);
signal f4_c21_t1 : signed(2*g_bitwidth-1 downto 0);
signal f4_c22_t1 : signed(2*g_bitwidth-1 downto 0);
signal f4_c22_t2 : signed(2*g_bitwidth-1 downto 0);
               
-- Stage 5 : Output calculations
signal f5_c11    : signed(2*g_bitwidth-1 downto 0);
signal f5_c12    : signed(2*g_bitwidth-1 downto 0);
signal f5_c21    : signed(2*g_bitwidth-1 downto 0);
signal f5_c22    : signed(2*g_bitwidth-1 downto 0);
\end{VHDLlisting}

The reset of the signal declarations have the same bit width as the multiplication result. We declare the two terms for the final output in stage four and the four element results are calculated in stage five. 

\begin{VHDLlisting}[tabsize]
p_s4 : process(i_clk)
begin
	if rising_edge(i_clk) then
		if f_i_dv(2) = '1' then
			f4_c11_t1 <= fff_p + fff_s;
		    f4_c11_t2 <= fff_t + fff_v;
		    f4_c12_t1 <= fff_r + fff_t;
		    f4_c21_t1 <= fff_q + fff_s;
		    f4_c22_t1 <= fff_p + fff_r;
		    f4_c22_t2 <= fff_q + fff_u;
		
		end if;
	end if;
end process;
\end{VHDLlisting}

In the calculation of stage four we calculate the terms in \eq{couts}. The additions are calculated under the data valid line for stage four and once complete they are ready for the final stage of calculation. 

\begin{VHDLlisting}[tabsize=2]
p_s5 : process(i_clk)
begin
	if rising_edge(i_clk) then
		if f_i_dv(3) = '1' then
			f5_c11 <= f4_c11_t1 - f4_c11_t2;
		    f5_c12 <= f4_c12_t1;
		    f5_c21 <= f4_c21_t1;
		    f5_c22 <= f4_c22_t1 - f4_c22_t2;
		end if;
	end if;
end process;
\end{VHDLlisting} 

The final stage, stage five, is calculated once again from \eq{couts} where for terms $c_{1,1}$ and $c_{2,2}$ a subtraction is needed but the other two elements are already final but we need to register them again to ensure data is aligned. 

\begin{VHDLlisting}[tabsize=2]
-- Assign Outputs
o_dv   <= f_i_dv(5);
o_c_11 <= f5_c11;
o_c_12 <= f5_c12;
o_c_21 <= f5_c21;
o_c_22 <= f5_c22;
\end{VHDLlisting}

After the final stage is calculated and registered we can use these registers as the output registers so now all we need to do is assign them as the output. At this point we need the fifth data valid in the shift register to denote that the output is valid. 

This shift register denotes the latency of the core. If we put in data at clock cycle zero. It will take six clock cycles to get valid data out. Note that index five in the data valid array is actually the sixth element since the array is zero indexed. 

\subsection{Knapsack Problem}
	<TODO Subsection Knapsack Problem : NOT DONE>

\subsection{Digital Signal Processing}
	<TODO Subsection Digital Signal Processing : PROOF READ>
	
A common application for \ac{FPGA}s is \ac{DSP}. In this section we discuss the implementation of common \ac{DSP} functions. Here will discuss the \ac{FIR} filter and an \ac{NCO}. Which are both vital in a \ac{DSP} application. 

\subsubsection{Finite Impulse Response Filter}
	<TODO Subsubsection  Finite Impulse Response Filter : PROOF READ>

There are many needs for \ac{FIR} filters in \ac{DSP}. Here we will present one type of \ac{FIR} filter that is quite useful. The type of filter we will be implementing in this section is the \emph{half-band \ac{FIR} \ac{LPF}}. Breaking down the name we first know what a \ac{FIR} filter is and does which depending on the impulse response attenuates some frequencies while keeping other frequencies of interest. 

The \emph{half-band} aspect of this filter refers to the ability of the filter to reduce the bandwidth of the filtered signal by half. The reduction of bandwidth of the filtered signal reduces the data rate of the system after the filter. You may be asking why would you have a higher data-rate coming into the \ac{FPGA} from the \ac{ADC} than you would need. The data-rate coming into the \ac{FPGA} is dependent on the \ac{IF} used in the \ac{RF} front-end and the \ac{ADC} sample rate. 

The \ac{IF} used in the front-end is usually standard and your selection of the \ac{IF} is fixed. What you do have control over is the \ac{ADC} sampling rate. The main input parameter used in deciding your sampling rate is the bandwidth of the signal you are interested in, since your sampling rate needs to be at least twice the bandwidth of the signal you are interested in. 

Once the \ac{ADC} sampling rate is chosen we can start interfacing the \ac{ADC} to the \ac{FPGA}. The data lines are routed to the \ac{FPGA} along with a clock if the \ac{ADC} is has a source-synchronous interface. In any case the data is registered on the \ac{FPGA} and now we have another design parameter to decide. What processing speed are we going to use. In general we can slow the clock down and process multiple samples in parallel or speed the clock up and have more clocks per sample. In either case we need to consider the algorithm in which we are implementing. 

With all that said we will assume moving forward all of those parameters have been worked out. What happens quite frequently is that the bandwidth of the received signal is twice as large as we are interested in so we are going to decimate by two. To decimate by two we need to filter out the aliases to avoid sampling interference. This filter is called an anti-alias filter. So the filter we are going to design here is going to be out anti-alias filter. The data is then sent out of the core where the data valid line is dropped every other sample to perform the decimation by two. 

\begin{VHDLlisting}[tabsize=2]
-- halfbandLPF.vhd

library ieee;
	use ieee.std_logic_1164.all;
	use ieee.numeric_std.all;
	
entity halfbandLPF is 
	generic(g_bitwidth : integer := 16;
	        g_ncoeffs  : integer := 35);
	port(
			i_clk      : in    std_logic;
			i_reset    : in    std_logic;
			i_data     : in    std_logic_vector(g_bitwidth-1 downto 0);
			i_dv       : in    std_logic;
			o_data     :   out std_logic_vector(2*g_bitwidth-1 downto 0);
			o_dv       :   out std_logic	
	);
end entity halfbandLPF;

architecture rtl of halfbandLPF is 
	type T_Sbw is array (natural range<>) of signed(g_bitwidth-1 downto 0);
	type T_S2bw is array (natural range<>) of signed(2*g_bitwidth-1 downto 0);
	signal f_x_mix    : T_Sbw(0 to g_ncoeffs-1) := (others => (others => '0'));
	signal f_dv_shift : std_logic_vector(0 to g_ncoeffs-1) := (others => '0');
	
	-- Length of this vector is Ceil((g_ncoeffs-1)/4)+1
	-- These coefficients need to be loaded externally
	signal f_h_uni      : T_Sbw(0 to 9) := (x"0026",x"FFA6",x"00BD",x"FEA0",x"0261",x"FC05",x"06BC",x"F350",x"2870",x"4000");

	signal ff_mix_add   : T_Sbw(0 to 9) := (others => (others => '0'));
	signal fff_add_mult : T_S2bw(0 to 9) := (others => (others => '0'));
	
	signal f4_sum_s1    : T_S2bw(0 to 4) := (others => (others => '0'));
	signal f5_sum_s2    : T_S2bw(0 to 2) := (others => (others => '0'));
	signal f6_sum_s3    : T_S2bw(0 to 1) := (others => (others => '0'));
	signal f7_sum_s4    : signed(2*g_bitwidth-1 downto 0) := (others => '0');
begin
	o_data <= std_logic_vector(f7_sum_s4);
	o_dv <= f_dv_shift(6);

	p_calc : process(i_clk) is
	begin
		if rising_edge(i_clk) then
			if i_reset = '1' then
				f_dv_shift <= (others => '0');
			elsif i_dv = '1' then
				-- Shift Register Data
				f_x_mix(34) <= signed(i_data);
				f_x_mix(0 to g_ncoeffs-2) <= f_x_mix(1 to g_ncoeffs-1);
				
				-- Shift Register Data Valid
				f_dv_shift(0) <= i_dv;
				f_dv_shift(1 to g_ncoeffs-1) <= f_dv_shift(0 to g_ncoeffs-2);
				
				-- Take advantage of symmetric filter. 
				ff_mix_add(0) <= f_x_mix(0) + f_x_mix(34);
				ff_mix_add(1) <= f_x_mix(2) + f_x_mix(32);
				ff_mix_add(2) <= f_x_mix(4) + f_x_mix(30);
				ff_mix_add(3) <= f_x_mix(6) + f_x_mix(28);
				ff_mix_add(4) <= f_x_mix(8) + f_x_mix(26);
				ff_mix_add(5) <= f_x_mix(10) + f_x_mix(24);
				ff_mix_add(6) <= f_x_mix(12) + f_x_mix(22);
				ff_mix_add(7) <= f_x_mix(14) + f_x_mix(20);
				ff_mix_add(8) <= f_x_mix(16) + f_x_mix(18);
				ff_mix_add(9) <= f_x_mix(17);
				
				-- Multiply Impulse response. 
				for i in 0 to 9 loop
					fff_add_mult(i) <= ff_mix_add(i) * f_h_uni(i);
				end loop;				
				
				-- Start of the adder tree.
				for i in 0 to 4 loop
					f4_sum_s1(i) <= fff_add_mult(2*i) + fff_add_mult(2*i+1);
				end loop;
				
				f5_sum_s2(2) <= f4_sum_s1(4);
				for i in 0 to 1 loop
					f5_sum_s2(i) <= f4_sum_s1(2*i) + f4_sum_s1(2*i+1);
				end loop;
				
				f6_sum_s3(0) <= f5_sum_s2(0) + f5_sum_s2(1);
				f6_sum_s3(1) <= f5_sum_s2(2);
								
				-- Result of adder tree.
				f7_sum_s4 <= f6_sum_s3(0) + f6_sum_s3(1);
			end if;			
		end if;
	end process;
end rtl;
\end{VHDLlisting}
	
In the entity declaration we have two generics defined. The first the is bit-width of the input samples. The second is the number of coefficients or taps in the \ac{FIR} filter. The port declarations have the clock, reset, input, and output data lines. We notice that the output is twice the bit width as the input since the \ac{FIR} taps are the same bit width as the input data and we multiply the data by the taps getting twice the bit-width in growth. 

In the architecture block before the \emph{begin} keyword we define two types. Each of which is an array one for the input data width and the other twice the input data width. The first signal that is declared is a shift register for the input data and a data valid shift register is defined after. 

Next we have signal that is initialized to a set of 10 coefficients. We first develop the \ac{FIR} filter with the taps already in place so that we know there are not any issues we loading the coefficients externally. Loading the coefficients externally would make this \ac{FIR} filter implementation the most flexible and very easy to reuse the the data-rates were to change in future projects. Lastly defined are signals for five stages of an adder tree.

After the \emph{begin} keyword we have the outputs being assigned. Next we move to a calculation process that has a synchronous reset. Then we have the shift register for the data. Then another shift register for the data valid. This implementation assumes that the input is directly from an \ac{ADC} so once the data valid is asserted the input is always valid. 

Next we take advantage of the symmetric filter taps used in this \ac{FIR} filter to reduce the number of multiples. To do this we need to add the data at various indices together. Then in the next stage the impulse response is multiplied by the summation in the previous step. After which the ten values are summed in an adder tree where the number of adders is $log_2$ of the number of elements left to add. 
	
\subsubsection{Numerically Controlled Oscillator}
	<TODO Subsubsection  Numerically Controlled Oscillator : PROOF READ>
	
An \ac{NCO} is used to generate \emph{sine} and \emph{cosine} waves on an \ac{FPGA}. The generation of sinusoidal waves is beneficial for digitally mixing signals. In this section we discuss how to make an \ac{NCO} in two parts. First a phase accumulator is used to generate an address which is used to index in to a \ac{BRAM} which is assigned as the output. 

\begin{VHDLlisting}[tabsize=2]
-- nco.vhd

library ieee;
	ieee.std_logic_1164.all;
	ieee.numeric_std.all;
	
entity nco is
generic(g_phinc_width : integer;
	    g_bitwidth    : integer;);
port(i_clk        : in    std_logic;
	 i_rst        : in    std_logic;
	 i_enable     : in    std_logic;
	 i_phz_init   : in    unsigned(g_phinc_width-1 downto 0);
	 i_phz_dv     : in    std_logic;
	 i_phinc_init : in    unsigned(g_phinc_width-1 downto 0);
	 i_phinc_dv   : in    std_logic;
	 o_nco_data   :   out unsigned(g_bitwidth-1 downto 0);
	 o_nco_dv     :   out std_logic
);
end 

architecture rtl of nco is
	type t_us_g is array (natural range <>) of unsigned(g_bitwidth-1 downto 0);
	signal f_bram       : t_us_g(0 to 2**g_phinc_width-1);
	signal f_phz_accum  : unsigned(g_phinc_width-1 downto 0);
	signal f_phinc      : unsigned(g_phinc_width-1 downto 0);
	signal f_nco_data   : unsigned(g_bitwidth-1 downto 0);
	signal f_nco_dv     : std_logic := '0';
	signal ff_nco_dv    : std_logic := '0';
begin

	o_nco_data <= f_nco_data;
	o_nco_dv   <= ff_nco_dv;
	
	p_init : process(i_clk)
	begin
		if rising_edge(i_clk) then
			if i_rst = '1' then
				f_phz_accum <= (others => '0');
				f_nco_dv <= '0';
			else
				if i_phz_dv = '1' then
					f_phz_accum <= i_phz_init;
				end if;
				
				if i_phinc_dv = '1' then
					f_phinc <= i_phinc_init;
				end if;
				
				f_nco_dv <= i_enable;
				ff_nco_dv <= f_nco_dv;
				f_nco_data <= f_bram(to_integer(f_phz_accum));
				if i_enable = '1' then
					f_phz_accum <= f_phz_accum + f_phinc;
				end if;				
			end if;
		end if;	
	end process;
end rtl;
\end{VHDLlisting}

The entity declaration starts with the generic declarations. We have a generic that determines the phase increment width, denoted \emph{phinc\_width}. We also define a generic that determines the bit width of the \ac{NCO} output. The difference here is that our frequency resolution in our \ac{NCO} is independent of our dynamic range of our sinusoidal wave. 

We move on to the port definitions of the core. We, of course, have the clock and reset. Next we have an enable line that activates the core in producing outputs. Then two data lines with corresponding valid flags for phase initial value and phase increment value, each of which are discussed later on. Finally the output of the \ac{NCO} with a valid line. 

In the architecture block we first define signals for internal use of the core. The core uses a \ac{BRAM} to store the \ac{LUT} for the sinusoid, for which we need a type definition. Next we have the phase accumulator. Then we have the registered phase increment. Finally registers for the outputs are defined.

After the \emph{begin} keyword, we first assign the outputs that are calculated later in the core. Next, we have the only process in the core. The process is a synchronous process that resets the phase accumulation register and the output data valid register. If not in reset then the output of the \ac{NCO} is calculated. 

While not in reset we perform three major tasks. First of which is registering the initial phase input. The initial phase allows the entity that instantiates the \ac{NCO} to determine the starting phase of the sinusoid. The second major task is the registration of the \emph{phase increment} once registered, the phase increment is used in the third major task.

The third major task has two parts. While the enable is high the phase accumulator is incremented by the phase increment. The phase increment is used as an address into a \ac{BRAM}. The \ac{BRAM} is initialized with a sinusoid that is sampled by the phase accumulator. As the phase accumulator rolls over in unsigned value the sinusoid \ac{BRAM} cycles over the next period of the wave being generated. The sinusoid is an output of the core. 

\section{Security in Hardware}
	<TODO Section Security in Hardware : PROOF READ>

There is a fundamental difference in programming a processor and designing hardware, as we have seen in this chapter. Consider the differences, when programming a processor there is an atomic instruction that is executed with references to register addresses. However, in an \ac{FPGA}, data is registered with no way of accessing the data by address and no way of changing the \ac{FPGA} configuration without erasing the data. 

Due to the trusted nature of hardware, which an \ac{FPGA} is configurable hardware, \ac{FPGA}s are finding a place as a front-line malicious packet detection. If a data packet comes over the network that is detected as malicious then appropriate action can be taken. 

\chapter{Data Transfer}
\section{Buttons}
\section{UART}
\section{SPI}
\section{I$^2$C}
\section{USB}
\section{Ethernet programing}

%input{chapter8.tex}
\chapter{Trouble Shooting and Debugging}
\section{Programming the Board (USB Device Driver)}
\section{Functional Verification}
\section{On-Chip Debugging Tools}

%input{chapter9.tex}
\chapter{Advanced topics}
\section{Hardware and Timer Interrupts}
\section{Data Logging with SD cards}
\section{Connecting to Internet}
\section{Audio Electronics}
\subsection{Equalization}
\subsection{Compression}
	
%input{chapter10.tex}
\chapter{Filters}
\section{Types of Digital Filters}
\section{Optimization for Digital Logic}
\section{Optimization for Micro-controllers}

%input{chapter11.tex}
\chapter{Case Study: Matrix Multiplication}
\section{Combinatorial Implementation}
\section{Sequential Implementation}

%\input{chapter12.tex}
\chapter{Case Study: Digital Receiver}
Is digital transceiver design city walks to the design of a orthogonal frequency-division multiplexing receiver first a section that describes how the what the major blocks of the receiver are is described next converting that to vhdl contesting East sub-module and then test by touching the receiver as a system and then finally putting the design synthesizing and placing Rock on xilinx fpga

\section{Digital Receiver Design - System Level}
In this section we discuss the system model for the receiver in this case that he will be focusing on the receiver however the transmitter is a design very similar however the receiver is is more difficult receiver has to deal with noise that the channel provides in the channel as into the system fever must be designed and weight snow ice handle this noise the first block in a little past the first block and a transceiver is a is a little pass filter is low pass filter prevent anti-aliasing colorado signal quality and ice block as a mask filter mSU filter is set to determine when the Preamble arrives the Preamble is a none sequence of deaths the Preamble is a known sequence of bits that is used to determine the synchronization between the transmitter and receiver on the synchronization is known then we can do my July the rest of the signal after the Preamble The Signal structure after the preamble the signal is an ofdm block ofdm block consists of a secret prefix andy cyclic prefix and a set up in superiors in the time to me the first step is to remove the cyclic prefix then transform to the frequency domain frequency domain is where the clam symbols are allocated once in the frequency domain the channel can be estimated dividing the pilot towns by dividing the pilot towns that were received 509 pilottown that were transmitted the resulting Channel estimates are at the sub cares where the pilot turns off where the pilot zones are located we use these pilot towns are we use these Channel estimates weezy's Channel estimates to extrapolate the channel estimates for the sub. On the date of tones 1 day fix tracker the channel estimates for all some carriers will I still receive signal remodeling and demodulate when's the beds have been estimated then for error correction or crippled graphic operations can occur on the bits that have been to modulated

The section talks about division multiplexing transmission transceivers in the section we Define the equations internet PGA to implement the transceiver whitley first take a look at orthogonal frequency-division multiplexing and how it compares to traditional single carrier we can take a look at how multipath is considered in single carrier and ofdm find a good look at simulation results for the system

\section{System Model - OFDM}
First we'll consider quad quadrature amplitude in particular will look at for 2 a.m.4q a.m. In the signaling scheme we have two levels for the ID channel and two levels of for the Q Channel can of this scheme lawson - 1 + 10 - I will use the clam constellation to Define our sub carriers in the frequency domain if we first take our string of bits and map them to cram symbols which symbol the bitstream will line up n dog carriers worth of Clan symbols well then take the inverse Fourier transform of them and get any samples in the time domain representing the end this is the signal that will be transmitted in our if we look at our model the if you look at the model for the Wireless Communications will have y equals convoluted with x plus in Texas it is the impulse response of the channel what is representing a measurement noise and why is the receive signal app at the receiver first we'll consider the situation where we have a frequency flat fading Channel in this case it is a single complex component all the notes are Channel length which equals one in this case nH of 0 cool aralie fading Channel nicole frequency flat fading complex component Within corresponding to the channel attenuation and a phase response elsa need to make sure we include multiple Mall imodium blocks in the transmitted signal can semen sample of that in this scenario and frequency flat fading we do not experience any interblock interference or we do not experience any interference where one of them block interferes with another of the empire when we increase the length of the channel making a sequel 10 for example and we look at the convolution of a transmitted signal with the impulse response what we have channel being multiplied the previous of the block being added to nation of the frequency response impulse response with a current of them block this energy is referred to as interblock or intersymbol interference the reader should also be aware of it inter carrier interference and this is where the zoo orthogonal to each other this happens all when a Doppler shift occurs for the the Doppler shift occurs for the minutes this is usually correct the through pilottown correction and it will not be discussed in this doppler shift is an issue in this to avoid intersymbol or inter block interference specifics of the last Eng Subs EP samples in the time to Maine from the end of the block from the block if I left him block him blocked out consists of a secret prefix and the end carrier transform into the time to me the first Eng Subs EP in the last and subsidies CP samples are the same they are just copied from the to the front forget the name this is where we get the name cyclic prefix now if we look at a graphic of the convolution of the impulse response with the transmitted signal are no where an impulse then the separate prefix length - 1 where the impulse response times northfield block the still occurs but it occurs in the cyclic prefix the prefix is simply thrown away consider it just a buffer also consider the reason why we use a cigarette prefix instead of just using Dead Space zero's at that location what would happen if the receiver is off by e90 face the receiver could trick her that phase difference frequency domain still be able to recover the transmitted symbols if the face of the receiver is off there is no information lost because well that just doesn't need any another we have the signal model we have X so multiple of 10 blocks basically facts pertaining to itself so if we have 10 of them blocks we have 10 ticket prefix list if we have 10 of them blocks 10 distinct sense of N Sub carriers in the time to Maine of those sets of samples we have 10 distinct that's of sickly prefixes does the prefix and time sample blocks interspersed rating the diagram in figure 94 then when we launch the signal into little experience now and I block interference go to demodulators the signal there of course is a preamble of the pretended to be transmitted signal this is what the receiver locks onto but for now we'll consider renovation step already complete at the receiver the first thing the receiver does is remove the sick like me fix this is usually just a flag hanging in FTP nope t block on the fpga is India started during the 66 recording time started at the first sample in the idiom block after the sickly prefix once in the frequency domain I need to go back and talk about pilot Townsend need to go back and talk about pilot Tums tones indiana towns and tones data and data tons now using the properties of convolution in the frequency domain is multiplication between a transmitted signal in the frequency domain and the channels frequency response in this case h plus n success bigger than hell - 1 estimate the channel estimate destiny the channel at the pilot y equals 8 times x equation and solve for H each sub carrier you calculate h equals y divided by X pilottown that we allocated at the transmitter known as a receiver why is the received signal at that particular subject we do this for all would you this for all of the Pilot Jones transmission so in general the number of Paula Times you should do this is the number of impulse response capstar it'll be fantastic if you know this ahead of time but in most Wireless channels you don't and the reason that the prefix length and the number of pilot owns should pretty close together should be the same because she went

\section{Module Design}
The first module in the receipt chain is a low pass filter low pass filter it prevents aliasing for the receiver it sets the bandwidth of the receiver find received this filter also impacts the signal-to-noise ratio the receiver if the low filter is too narrow signal is cut off and if it's too wide too much noise is let through this system a common lopes filter used in receivers are the square root for you to sign filter square-root race cosine filters use the transmitter and at the receiver and due to the nature of convolution overall filter used cosine filter in the sectional look at how the race goes on filter operates in in the RF energy Spectrum first look at the frequency response of the square is cosine filter oh also need to look into other types of low pass filters bear efficiency in hardware and software will touch up on these topics but they will also look at the hardware emergency can I talk to you then

The next section is the preamble or the magic filter for the preamble in the section will talk about the match filter for next filter and the Preamble for the attitude at 11 standard 802. 11 n standard will take a look at this point ensor probability of correlation under varying SNR conditions will look at a preamble that is randomly generated at the same length compare this to 1511 standard

The section discusses the Fourier transform of the receive signal the fast Fourier transform is a common algorithm for determining the Discrete Fourier transform of a signal the fast Fourier transform is of complexity login much more of this more efficient compared to the 848 transform which is N squared complexity in Hardware those other than has already developed for you and IP blocks from xilinx in Ontario he's a few blocks are generated specifically for the part that you're using this makes it very efficient in the hardware that you're using however it hardware agnostic HDL what is preferable so you can Target different species with the same code general you can generate the IP block for the next sounds of PGA xilinx generator will also like thanks for removing of the second Pacific prefix fix is used in ofdm signals to prevent interblock interference the longer the secret prefix the longer the impulse balls of the channel response of the channel the longer the cyclic prefix the longer the channel channels impulse response can be without experiencing interblock interference in an urban environment where there are many Reflections off buildings relatively long longer sick sick with prefix is needed to overcome these effects however in a line of sight application but there are no more components longer cyclic prefix effects of the data rate of the communication system so the secret prefix length is a trade-off between I lost connection and connection

After the Fourier transform we are in the frequency domain we are at the data tones and Pilot Jones are defined firstly use the pilot towns to estimate the Channel Inn receiver the frequency response of each of the Pilot Jones frequencies estimated by division the reason for the pilot turns being in the frequency that medicine that I can take advantage of the identity of the property of the Fourier transform or a convolution in the kind of meat is multiplication in the frequency domain we see that since the pilot towns are known at the receiver received pilot towns by the known transmetropolitan and get an estimate of the frequency response at that sub carrier we do this for each of the superior is in the the number of Pilot Jones allocated by the transmitter needs to at least be the number of degrees of freedom in the impulse response the second step for channel estimation is 2888 moon pilot own channel estimates the channel the frequency response for all the other times wind 888 by using the Discrete Fourier transform Matrix where the accounts are removed corresponding to the data tones the pilot towns are left taking the inverse Fourier transform of this matrix will receive impulse response estimate the impulse response of the system of the channel responsible Channel the channel is estimated then we can take the Fourier transform of the impulse response and get the frequency response of the estimated frequency response of the channel

Now that we have the estimated frequency response of the system of the channel equalize the received signal the estimated transmitted symbols this is similarly done too center Channel estimation where we knew the transmit a signal and the receive signal the signal in the channel and now estimating minutes the estimate to transmit a signal we simply divide the received signal by the channel estimate 90 model 8 bpsk are clam signals

\section{Data Rate Study}
\section{Test Benching - Module Level}
\section{Test Benching - System Level}
\section{FPGA Resource Usage}
























%\begin{acronym}
%    \acro{AOA}{Angle of Arrival}
%    \acro{CRLB}{Cramer-Rao Lower Bound}
%    \acro{FIM}{Fisher Information Matrix}
%    \acro{GPS}{Global Positioning System}
%    \acro{MCRLB}{Modified CRLB}
%    \acro{ML}{Maximum Likelihood}
%    \acro{MLE}{Maximum Likelihood Estimate}
%    \acro{MSE}{Mean Squared Error}
%    \acro{NP}{Neyman-Pearson}
%    \acro{PDF}{Probability Density Function}
%    \acro{RMSE}{Root Mean Squared Error}
%    \acro{PSD}{Power Spectral Density}
%    \acro{ROC}{Receiver Operating Characteristic}
%    \acro{RSS}{Received Signal Strength}
%    \acro{SNR}{Signal to Noise Ratio}
%    \acro{TDOA}{Time Difference of Arrival}
%    \acro{TOA}{Time of Arrival}
%    \acro{WSN}{Wireless Sensor Network}
%\end{acronym}

\begin{acronym}[WPAFB]
  \acro{4G-LTE}{Fourth Generation - Long Term Evolution}	
  \acro{ADC}{Analog to Digital Converter}
  \acro{AFRL}{Air Force Research Laboratory}
  \acro{AFIT}{Air Force Institute of Technology}
  \acro{AIS}{Automatic Identification System}
  \acro{ARM}{Advanced RISC Machine}
  \acro{ASCII}{American Standard Code for Information Interchange}
  \acro{ASIC}{Application Specific Integrated Circuit}
  \acro{AWGN}{Additive White Gaussian Noise}
  \acro{AXI}{Advanced Microcontroller Bus}
  \acro{AXI4}{Advanced Microcontroller Bus Version 4}
  \acro{BER}{Bit Error Rate}
  \acro{BPSK}{Binary Phase Shift Keying}
  \acro{BRAM}{Block Random Access Memory}
  \acro{CE}{Channel Estimate}
  \acro{CFO}{Carrier Frequency Offset}
  \acro{CLB}{Configurable Logic Block}
  \acro{CLI}{Command Line Interface}
  \acro{CP}{Cyclic Prefix}
  \acro{CPU}{Central Processing Unit}
  \acro{CS}{Chip Select}
  \acro{CSI}{Channel State Information}
  \acro{CSS}{Cascading Style Sheets}
  \acro{DC}{Direct Current}
  \acro{DDR}{Double Data Rate}
  \acro{DFT}{Discrete Fourier Transform}
  \acro{DMA}{Direct Memory Access}
  \acro{DSP}{Digital Signal Processing}
  \acro{DUT}{Device Under Test}
  \acro{EOF}{End of File}
  \acro{EOL}{End of Line}
  \acro{Ex}{Eavesdropping Receiver}
  \acro{FCC}{Federal Communications Commission}
  \acro{FDCE}{Frequency Domain Channel Estimation}
  \acro{FDE}{Frequency Domain Equalization}
  \acro{FEC}{Forward Error Correction}
  \acro{FFT}{Fast Fourier Transform}
  \acro{FIFO}{First-In First-Out}
  \acro{FIR}{Finite Impulse Reponse}
  \acro{FPGA}{Field Programmable Gate Array}
  \acro{FSK}{Frequency Shift Keying}
  \acro{GB}{Giga-Bytes}
  \acro{GHz}{Giga-Hertz}
  \acro{GCC}{GNU Compiler Collection}
  \acro{GMSK}{Gaussian Minimum Shift Keying}
  \acro{GPIO}{General Purpose Input Output}
  \acro{GPU}{Graphical Processing Unit}
  \acro{GSM}{Global System for Mobile Communications}
  \acro{GUI}{Graphical User Interface}
  \acro{HDD}{Hard Disk Drive}
  \acro{HDL}{Hardware Description Language}
  \acro{HTML}{Hyper-Text Markup Language}
  \acro{Hz}{Hertz}
  \acro{I}{In-Phase}
  \acro{IBI}{Inter-Block Interference}
  \acro{I2C}{Inter-Integrated Circuit}
  \acro{IC}{Integrated Circuit}
  \acro{ICI}{Inter-Carrier Interference}
  \acro{IEEE}{Institute of Electrical and Electronics Engineers}
  \acro{IF}{Intermediate Frequency}
  \acro{IFFT}{Inverse Fast Fourier Transform}
  \acro{i.i.d.}{Independent and Identically Distributed}
  \acro{IP}{Intellectual Property}
  \acro{IO}{Input Output}
  \acro{IoT}{Internet of Things}
  \acro{ISI}{Inter-Symbol Interference}
  \acro{ISM}{Industrial, Scientific and Medical}
  \acro{ISP}{Internet Service Provider}
  \acro{IQM}{IQ Mismatch}
  \acro{LAN}{Local Area Network}
  \acro{LED}{Light Emitting Diode}
  \acro{LFSR}{Linear Feedback Shift Register}
  \acro{LOS}{Line-of-Sight}
  \acro{LPD}{Low Probability of Detection}
  \acro{LPF}{Low Pass Filter}
  \acro{LS}{Least Squares}
  \acro{LSB}{Least Significant Bit}
  \acro{LTE}{Long-Term Evolution}
  \acro{LUT}{Look-Up Table}
  \acro{LVDS}{Low Voltage Differential Signaling}
  \acro{MAC}{Multiply Accumulate}
  \acro{MB}{Mega-Bytes}
  \acro{MHz}{Mega-Hertz}
  \acro{MIMO}{Multiple-Input Multiple-Output}
  \acro{MIPS}{Microprocessor without Interlocked Pipelined Stages}
  \acro{MISO}{Master-In Slave-Out}
  \acro{ML}{Machine Learning}
  \acro{MOSI}{Master-Out Slave-In}
  \acro{MSB}{Most Significant Bit}
  \acro{MSK}{Minimum Shift Keying}
  \acro{NCO}{Numerically Controller Oscillator}
  \acro{OFDM}{Orthogonal Frequency Division Multiplexing}
  \acro{OFDMA}{Orthogonal Frequency Division Multiple Access}
  \acro{OS}{Operating System}
  \acro{PAPR}{Peak-to-Average Power Ratio}
  \acro{PCB}{Printed Circuit Board}
  \acro{PCIe}{Peripheral Component Interconnect Express}
  \acro{PDF}{Probability Density Function}
  \acro{PIC}{Peripheral Interface Controller}
  \acro{PMF}{Probability Mass Function}
  \acro{PMI}{Precoding Matrix Index}
  \acro{PN}{Phase Noise}
  \acro{PPC}{Performance Optimization With Enhanced RISC – Performance Computing}
  \acro{PSD}{Power Spectral Density}
  \acro{Q}{Quadrature}
  \acro{QAM}{Quadrature-Amplitude Modulation}
  \acro{RAM}{Random Access Memory}
  \acro{RB}{Receiver Block}
  \acro{RF}{Radio Frequency}
  \acro{RISCV}{Reduced Instruction Set Computing Version 5}
  \acro{RMSE}{Root-Mean Squared Error}
  \acro{ROA}{Region of Activity}
  \acro{RPi}{Raspberry Pi}
  \acro{RTL}{Register-Transfer Level}
  \acro{Rx}{intended receiver}
  \acro{SATA}{Serial Advanced Technology Attachment}
  \acro{SBC}{Single Board Computer}
  \acro{SC-FDMA}{Single Carrier - Frequency Division Multiple Access}
  \acro{SD}{Secure Digital}
  \acro{SDR}{Software Defined Radio}
  \acro{SINR}{Signal to Interference plus Noise Ratio}
  \acro{SIMO}{Single-Input Multiple-Output}
  \acro{SISO}{Single-Input Single-Output}
  \acro{SNR}{Signal to Noise Ratio}
  \acro{SPI}{Serial Peripheral Interface}
  \acro{SSD}{Solid State Drive}
  \acro{STBC}{Space-Time Block Codes}
  \acro{SVD}{Singular Value Decomposition}
  \acro{TB}{Tera-Bytes}
  \acro{TCL}{Tool Command Language}
  \acro{TDCE}{Time Domain Channel Estimation}
  \acro{TI}{Texas Instruments}
  \acro{TI-DSP}{Texas Instruments-Digital Signal Processor}
  \acro{TOC}{Tactical Operation Centers}
  \acro{TOR}{Time Of Reference}
  \acro{TRB}{Transmitter-Receiver Block}
  \acro{UAVs}{Unmanned Aerial Vehicles}
  \acro{UART}{Universal Asyncronous Receiver Transmitter}
  \acro{uC}{micro-Controller}
  \acro{UI}{User Interface}
  \acro{uPP}{Universal Parallel Port}
  \acro{US}{United States}
  \acro{USB}{Universal Serial Bus}
  \acro{USRP}{Universal Software Radio Peripheral}
  \acro{VHDL}{VHSIC Hardware Description Language}
  \acro{VLSI}{Very Large Scale Integration}
  \acro{WARP}{Wireless open-Access Research Platform}
  \acro{WLAN}{Wireless Local Area Network}
  \acro{WPAFB}{Wright Patterson Air Force Base}
\end{acronym}

  % put this at end of paper
\end{document}
