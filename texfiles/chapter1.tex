\chapter{Overview of Computing Platforms}

\section{What Can an FPGA Do}

\subsection{History of FPGAs}
\subsubsection{SPLDs and CPLDs}
\subsubsection{Macrocells}

\subsection{Why Use an FPGA}
\subsubsection{Introduction to Data Rate}
A data rate in general is defined as the number of bits per unit time. For a system the data rate is how much data can the system handle per unit time. The overall data rate of the system is the minimum data rate for the individual components. In designing a system with a minimum data rate requirement it is important to be mindful of the data rates of the components.

To make the concept of determining data rate a little more concrete lets look at a simple architecture of a wireless receiver. To determine the data rate of a system the knowledge of the application and more importantly the architecture of the system needs to be known. Even for a simple receiver there are multiple ways to architect the design to enable higher or lower data rates. 

\subsubsection{Data Rate Study - Image and Video Processing}

\section{What Can a SBC Do}
\subsubsection{The Performance of Decisions}
\subsubsection{Data Rate Study - Web Hosting}

\section{FPGA vs. Processor}
\subsubsection{Balancing Data Rate and Decision Making}


