\chapter{Overview of Computing Platforms}
	<TODO Chapter Overview of Computing Platforms} : NOT DONE>

\section{What Can an FPGA Do}
	<TODO Section What Can an FPGA Do} : NOT DONE>

\subsection{History of FPGAs}
	<TODO Subsection History of FPGAs} : NOT DONE>

\subsubsection{SPLDs and CPLDs}
	<TODO Subsubsection  SPLDs and CPLDs} : NOT DONE>

\subsubsection{Macrocells}
	<TODO Subsubsection  Macrocells} : NOT DONE>

\subsection{Why Use an FPGA}
	<TODO Subsection Why Use an FPGA} : NOT DONE>

\subsubsection{Introduction to Data Rate}
	<TODO Subsubsection  Introduction to Data Rate} : NOT DONE>

A data rate in general is defined as the number of bits per unit time. For a system the data rate is how much data can the system handle per unit time. The overall data rate of the system is the minimum data rate for the individual components. In designing a system with a minimum data rate requirement it is important to be mindful of the data rates of the components.

To make the concept of determining data rate a little more concrete lets look at a simple architecture of a wireless receiver. To determine the data rate of a system the knowledge of the application and more importantly the architecture of the system needs to be known. Even for a simple receiver there are multiple ways to architect the design to enable higher or lower data rates. 

\subsubsection{Data Rate Study - Image and Video Processing}
	<TODO Subsubsection  Data Rate Study - Image and Video Processing} : NOT DONE>

\section{What Can a SBC Do}
	<TODO Section What Can a SBC Do} : NOT DONE>

\subsubsection{The Performance of Decisions}
	<TODO Subsubsection  The Performance of Decisions} : NOT DONE>

\subsubsection{Data Rate Study - Web Hosting}
	<TODO Subsubsection  Data Rate Study - Web Hosting} : NOT DONE>

\section{FPGA vs. Processor}
	<TODO Section FPGA vs. Processor} : NOT DONE>

\subsubsection{Balancing Data Rate and Decision Making}
	<TODO Subsubsection  Balancing Data Rate and Decision Making} : NOT DONE>
